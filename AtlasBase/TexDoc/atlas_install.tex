\documentclass[11pt]{article}

%\usepackage{psfig}
\usepackage{epsfig}


\newcommand{\Wskip}[1]{ }
\newcommand{\Wceil}[1]{\lceil #1 \rceil}
\newcommand{\Wfloor}[1]{\lfloor #1 \rfloor}

\newenvironment{routdef}[1]
{
   \begin{list}{BLANK}
   {
      \setlength{\parsep}{0in}
      \setlength{\itemsep}{.01in}
      \setlength{\partopsep}{0in}
      \setlength{\topsep}{0.1in}
      \setlength{\labelsep}{0in}
      \setlength{\labelwidth}{#1in}
      \setlength{\leftmargin}{#1in}
   }
} {\end{list}}
\newcommand{\rditem}[2]{\item[#1\hfill(~]#2 )}



\textwidth=6in
\textheight=8.7in
\hoffset = -.6in
\voffset = -.6in

\usepackage{fancyhdr}
\pagestyle{fancy}
\lfoot{}
\cfoot{}
\rfoot{}

\begin{document}
\headheight=25.3pt

\begin{titlepage}
\title{ATLAS Installation Guide
   \footnote{This work was supported in part by National Science
             Foundation CRI grant SNS-0551504}}
\vspace{.4in}
\author
{
 R. Clint Whaley \thanks { {\tt rwhaley@users.sourceforge.net}, 
 {\tt www.cs.utsa.edu/$\sim$whaley}}
}
\end{titlepage}
\maketitle

\fancypagestyle{plain}{}
%\lhead{\footnotesize \bf ATLAS Installation Guide}}
%\chead{\footnotesize \bf ATLAS Installation Guide}
%\rhead{\footnotesize CONTENTS}

%\fancypagestyle{plain}{\lhead{\bf University of Texas at San Antonio,
%    Department of Computer Science, Technical Report CS-TR-2008-002}\chead{}\rhead{}}
\begin{abstract}
This note provides a brief overview of ATLAS, and describes how to install
it.  It includes extensive discussion of common configure options, and
describes why they might be employed on various platforms.  In addition
to discussing how to configure and build the ATLAS package,
this note also describes how an installer can confirm that the resulting
libraries are producing correct answers and running efficiently.
Extensive examples are provided, including a full-length example showing
the installation of both ATLAS and LAPACK on an example architecture.

\end{abstract}

\newpage
\headheight=12.5pt
%\lhead{\footnotesize \bf UTSA/CS Technical Report}
%\chead{\footnotesize \bf CS-TR-2008-002}
\rhead{\footnotesize CONTENTS}
\begin{small}
\tableofcontents
\end{small}
%\listoftables
%\listoffigures

\newpage
\setcounter{page}{1}
%\lhead{\footnotesize \bf UTSA/CS Technical Report}
%\chead{\footnotesize \bf CS-TR-2008-002}
\rhead{{\footnotesize\bf Whaley}~~~~~~\thepage}
\section{Introduction}
This note provides a quick reference to installing and using 
ATLAS~\cite{atlas-hp,atlas_wn97,atlas_sc98,atlas_siam,WN147,whaley04}.  ATLAS
(Automatically Tuned Linear Algebra Software), is an empirical tuning
system that produces a BLAS~\cite{blas3,blas2a,blas2b,blas1a,blas1b}
(Basic Linear Algebra Subprograms) library which has been specifically optimized
for the platform you install ATLAS on.  The BLAS are a set of building block
routines which, when tuned well, allow more complicated Linear Algebra 
operations such as solving linear equations or finding eigenvalues to run
extremely efficiently (this is important, since these operations are
computationally intensive).  For a list of the BLAS routines, see the FORTRAN77
and C API quick references guides available in the ATLAS tarfile at:
\begin{verbatim}
   ATLAS/doc/cblasqref.pdf
   ATLAS/doc/f77blasqref.pdf
\end{verbatim}

ATLAS also natively provides a few routines from the LAPACK~\cite{lug}
(Linear Algebra PACKage).  LAPACK is an extremely comprehensive
FORTRAN77 package for solving the most commonly occurring problems in numerical
linear algebra.  LAPACK is available as an open source FORTRAN77 package from
netlib~\cite{lapack-hp}, and its size and complexity effectively rule out the
idea of ATLAS providing a full implementation.  Therefore, we add support
for particular LAPACK routines only when we believe that the potential
performance win we can offer make the extra development and maintenance
costs worthwhile.  Presently, ATLAS provides roughly 40 routines, all of
which derive from our improved LU and Cholesky factorizations, which use
recursive blocking.  The standard LAPACK routines use statically blocked
routines, which typically run slower than recursively blocked for all
problem sizes.  ATLAS's LU and Cholesky factorizations are based on
the work of~\cite{Toledo-LU,RecFred,gustavson98a,WN146,RecQR}.

In addition to providing the standard FORTRAN77 interface to LAPACK,
ATLAS also provides its own C interface, modeled after the official C interface
to the BLAS~\cite{Blast,blast-toms}, which includes support for row-major
storage in addition to the standard column-major implementations.  Note that
there is no official C interface to LAPACK, and so there is no general C
API that allows users to easily substitute one C-interface LAPACK for
another, as there is when one uses the standard FORTRAN77 API. 
For a list of the LAPACK routines that ATLAS natively supplies, see the 
FORTRAN77 and C API quick references guide available in the ATLAS tarfile at:
\begin{verbatim}
   ATLAS/doc/lapackqref.pdf
\end{verbatim}

Note that although ATLAS provides only a handful of LAPACK routines, it is
designed so that it can easily be combined with netlib LAPACK in order
to provide the complete library.  See Section~\ref{sec-lapack} for details.

\section{Overview of an ATLAS Installation}
\label{sec-install}

\subsection{Downloading the software and checking for known errors}
The main ATLAS homepage is at:
\vspace*{-0.1in}
\begin{verbatim}
   http://math-atlas.sourceforge.net/
\end{verbatim}

The software link off of this page allows for downloading the tarfile.
The explicit download link is:
\vspace*{-0.1in}
\begin{verbatim}
   https://sourceforge.net/project/showfiles.php?group_id=23725
\end{verbatim}

Once you have obtained the tarfile, you untar it in the directory where
you want to keep the ATLAS source directory.  The tarfile will create
a subdirectory called {\tt ATLAS}, which you may want to rename to make
less generic.  For instance, assuming I have saved the tarfile to
{\tt /home/whaley/dload}, and want to put the source in 
{\tt /home/whaley/numerics}, I could create ATLAS's source directory
({\tt SRCdir}) with the following commands:
\begin{verbatim}
   cd ~/numerics
   bunzip2 -c ~/dload/atlas3.8.0.tar.bz2 | tar xfm -
   mv ATLAS ATLAS3.8.0
\end{verbatim}

Before doing anything else, scope the ATLAS errata file for known 
errors/problems that you should fix/be aware of before installation:
\vspace*{-0.1in}
\begin{verbatim}
   http://math-atlas.sourceforge.net/errata.html
\end{verbatim}

This file contains not only all bugs found, but also all kinds
of platform-specific installation and tuning help.

\subsection{Turn off CPU throttling when installing ATLAS}
Most OSes (including Linux) now turn on CPU throttling for power management\\
\underline{even if you are using a desktop machine}.  CPU throttling makes
pretty much all timings completely random, and so any ATLAS install will be
junk.  Therefore, before installing ATLAS, turn off CPU throttling.  For most
PCs, you can switch it off in the BIOS (eg., on my Athlon-64 machine, I can say
"No" to "Cool and Quiet" under "Power Management").  Most OSes also provide a
way to switch off CPU throttling, but that varies from OS to OS.  Under
Fedora, at any rate, the following command seemed to work:
\begin{verbatim}
   /usr/bin/cpufreq-selector -g performance 
\end{verbatim}

On my Core2Duo, cpufreq-selector only
changes the parameters of the first CPU, regardless of which cpu you specify.
I suspect this is a bug, because on earlier systems, the remaining CPUs were
controlled via a logical link to /sys/devices/system/cpu/cpu0/.  In this case,
the only way I found to force the second processor to also run at its peak
frequency was to issue the following as root after setting CPU0 to performance:
\begin{verbatim} 
   cp /sys/devices/system/cpu/cpu0/cpufreq/scaling_governor \
      /sys/devices/system/cpu/cpu1/cpufreq/scaling_governor 
\end{verbatim}
Under MacOS or Windows,
you may be able to change this under the power settings.

ATLAS config tries to detect if CPU throttling is enabled, but it may not
always detect it, and sometimes may detect it after you have disabled it.
In the latter case, to force the {\tt configure} to continue regardless of the
results of the CPU throttling probe, pass this flag to {\tt configure}:
\begin{verbatim}
   -Si cputhrchk 0
\end{verbatim}

\subsection{Basic Steps of an ATLAS install}
An ATLAS install is performed in 5 steps, only the first two of which
are mandatory.
This install process is very similar to other free software installs,
particularly gnu, though the fact that ATLAS does an extremely complex
empirical tuning step can make the build step particularly long running.
There are two directories that we will refer to constantly in this note,
which indicate both the ATLAS source and build directories:
\begin{description}
\item[{\tt SRCdir}]: This handle should be replaced by the path to your
      ATLAS source directory (eg, {\tt /home/whaley/ATLAS3.8.0}).
\item[{\tt BLDdir}]: This handle should be replaced by the path to your
      ATLAS build directory (eg, {\tt /home/whaley/ATLAS3.8.0/Linux\_P4E64SSE3}).
\end{description}
Note that these two directories cannot be the same (i.e. you cannot build
the libraries directly in the source directory).  The examples in this
note show the {\tt BLDdir} being a subdirectory of the {\tt SRCdir}, but
this is not required (in fact, any directory to which the installer has
read/write permission other than {\tt SRCdir} can be used).

\vspace*{0.1in}
\noindent
The ATLAS install steps are:
\begin{enumerate}
\item {\bf configure} (\S\ref{sec-configure}): 
      Tell the ATLAS build harness where your {\tt SRCdir}
      and {\tt BLDdir} directories are, and allow ATLAS to probe the
      platform to create ATLAS's {\tt Make.inc} and {\tt BLDdir} directory
      tree.
\item {\bf build} (\S\ref{sec-build}): 
                  Tune ATLAS for your platform, and build the libraries.
\item {\bf check}\footnotemark[1] (\S\ref{sec-check}):
                  Run sanity tests to ensure your libraries are producing
                   correct answers.
\item {\bf time}\footnotemark[1] (\S\ref{sec-time}):
                  Run basic timing on various ATLAS kernels in order to
                  make sure the tuning done in the {\bf build} step
                  has resulted in efficient implementations.
\item {\bf install}\footnotemark[1] (\S\ref{sec-install-step}):
                  Copy ATLAS's libraries from the BLDdir to some
                  standard location.
\end{enumerate}
\footnotetext[1]{Optional step}

It is extremely important that you read Section~\ref{sec-configure} in 
particular, as most users will want to throw at least one flag during
the {\tt configure} step.  In particular, most installers will
want to set whether to build 32 or 64-bit libraries
(Section~\ref{sec-bitwidth}), and fine-tune the timer used, as discussed
in Section~\ref{sec-time-flags}.
However, for the impatient, here is the way a 
typical install might look (see \S\ref{sec-configure} for an explanation
of the {\tt configure} flags, since they will not work on all systems); 
note that the characters after the \verb+#+ character are comments,
and not meant to be typed in:
\begin{verbatim}
   bunzip2 -c atlas3.9.x.tar.bz2 | tar xfm -     # create SRCdir
   mv ATLAS ATLAS3.9.x                           # get unique dir name
   cd ATLAS3.9.x                                 # enter SRCdir
   mkdir Linux_C2D64SSE3                         # create BLDdir
   cd Linux_C2D64SSE3                            # enter BLDdir
   ../configure -b 64 -D c -DPentiumCPS=2400 \   # configure command
     --prefix=/home/whaley/lib/atlas \           # install dir
     --with-netlib-lapack-tarfile=/home/whaley/dload/lapack.tgz
   make build                                    # tune & build lib
   make check                                    # sanity check correct answer
   make time                                     # check if lib is fast
   make install                                  # copy libs to install dir
\end{verbatim}

\section{The ATLAS configure step}
\label{sec-configure}

In this step, ATLAS builds all the subdirectories of the {\tt BLDdir}, and
creates the {\tt make} include file used in all ATLAS's Makefiles 
({\tt Make.inc}).
In order to do this successfully, you inform ATLAS where your 
{\tt SRCdir} and {\tt BLDdir} are located, and pass flags which tell 
{\tt configure}
what type of install you want to do.  The basic way to do a configure step is:
\begin{verbatim}
   cd BLDdir ; SRCdir/configure [flags]
\end{verbatim}

A complete list of flags is beyond the
scope of this paper, but you can get a list of them by passing {\tt --help} to
{\tt configure}.  In this note, we will discuss some of the more important
flags only.  ATLAS takes two types of flags: flags that are consumed by the
initial {\tt configure} script itself begin with {\tt --}, and flags that are
passed by {\tt configure} to a later config step begin with only a 
single~{\tt -}.

We first discuss flags and steps for building a full netlib library
using netlib's LAPACK~(\S\ref{sec-lapack}), 
building a shared library~(\S\ref{sec-shared}),
changing the compilers~(\S\ref{sec-comps}), and 
a flag~(\S\ref{sec-nof77}) to indicate that you have no FORTRAN compiler 
(and thus don't need any FORTRAN APIs), and changing the way ATLAS
does timings~(\S\ref{sec-time-flags}). 
%and manually overriding various
%things configure detects~(\S\ref{sec-OS-override})
%such as OS type and assembly dialect.  
Finally, we consider a few
miscellaneous flags~(\S\ref{sec-flag-other}), including the flag telling
ATLAS whether the resulting libraries should assume a 64 or 32 bit
address space~(\S\ref{sec-bitwidth}).

\subsection{Building a full LAPACK library using ATLAS and netlib's LAPACK}
\label{sec-lapack}
ATLAS natively provides only a relative handful of the routines which comprise
LAPACK.  However, ATLAS is designed so that its routines can easily be
added to netlib's standard LAPACK in order to get a full LAPACK library.
If you want your final libraries to have all the LAPACK routines,
then you just need to pass the {\tt --with-netlib-lapack-tarfile} flag
to configure, along with the netlib tarfile that you have previously
downloaded.  For instance, assuming you have previously downloaded
the lapack tarfile to {\tt /home/whaley/dload/lapack.tgz}, you
would add the following to your {\tt configure} flags:
\begin{verbatim}
  --with-netlib-lapack-tarfile=/home/whaley/dload/lapack.tgz
\end{verbatim}

Configure then auto-builds a {\tt make.inc} for LAPACK to use,
and builds netlib LAPACK as part of the ATLAS install process.
Section~\ref{sec-lpdetails} provides details on how this works 
in case it breaks down for future releases of LAPACK (it has been
tested to work with netlib LAPACK 3.1.1 and 3.2.1).

\subsubsection{LAPACK APIs}
Note that there is no standard C API to LAPACK.  Therefore, when
you build the netlib LAPACK, you get only the Fortran77 API on
all platforms.  Various vendor libraries provide various C APIs.
ATLAS provides two types of LAPACK APIs for C.

\paragraph{ATLAS's clapack API:}
ATLAS's original C interface to the LU and Cholesky-related routines
is built from the {\tt ATLAS/interfaces/lapack/C/src/} directory,
and is documented in {\tt ATLAS/doc/cblasqref.pdf}.  This API
is like that of the {\tt cblas}, in that all routines take a
new argument that allows matrices to be either row- or column-major.
This API is difficult to extend to all of LAPACK, since the F77
LAPACK provided by netlib only handles column-major.  This API
uses the CBLAS enum types for F77's string arguments, and the
appropriate pass-by-value or pass-by-address.  This API prefixes
{\tt clapack\_} to the native lapack routine name.

\paragraph{ATLAS's C2F API:}
For routines where we depend on the Fortran LAPACK for implementation,
we have created a new C/LAPACK API, which can be found in \\
{\tt ATLAS/interfaces/lapack/C2F/src}.  This API is presently
undocumented.  It prefixes {\tt ATL\_C2F} to the native LAPACK
routine name.
This interface is designed to be quick to extend, and it's main
goal is to provide access to the F77 LAPACK without requiring the
caller to know the arcana of calling a Fortran77 routine from C
(which varies by compiler and platform).  Therefore, you get a 
nice C interface, using enums, with correct pass-by-address/value.
For every routine that takes a work argument, this API provides
two routines: one with the normal name which does \underline{not} take
a work, where the API does the {\tt malloc} for the user, and a second
routine with {\tt \_wrk} suffixed to the name that works exactly
like the F77 routine (user must pass in appropriate workspace).
So, for instance the lapack routine {\tt geqrf} takes work, and
so for double precision this API provides a routine {\tt ATL\_C2Fdgeqrf}
which takes no workspace-related parameters, and a 
{\tt ATL\_C2Fdgeqrf\_wrk} which takes workspace parameters.

The C2F API is built by simple wrappers around the F77 LAPACK,
and presently we have wrappers only for the QR-related routines.
It is extremely straightfoward to expand this API to cover all
of LAPACK, but I won't do so unless users ask for given routines,
or my own research requires them (no use producing code that
one using), so let me know if there are routines you'd like to
see ATLAS provide via this API.

\subsubsection{Details on ATLAS's LAPACK autobuild}
\label{sec-lpdetails}
The main thing that you need to install the netlib LAPACK is a correct
{\tt make.inc} which provides the correct compilers, flags, etc.  When
mixing ATLAS and netlib lapack, you need to be sure to use compatible
compiler flags.  ATLAS's lapack autobuild creates a {\tt make.inc} for
LAPACK to use by substituting the ATLAS options in {\tt make.inc.example}
that netlib LAPACK provides.  It auto-sets the macros
{\tt FORTRAN}, {\tt OPTS}, {\tt NOOPT}, {\tt LOADER}, {\tt LOADOPTS},
and {\tt TIMER}.  If in future releases of LAPACK they remove the
{\tt make.inc.example} file, rename the above macros, or provide
additional architecture-specific macros, then this autobuild process
may have to be modified.  In this case, you may be able to get things
working by creating a compatible {\tt make.inc} yourself after
the ATLAS configure step, but before the build step.  You can
see the Makefile targets for autobuilding netlib's ATLAS in
{\tt BLDdir/interfaces/lapack/C2F/Makefile}.
\Wskip{
Note that there is no standard C API to LAPACK, so this will get you
only the FORTRAN API.  As previously mentioned, ATLAS provides its own
C/F77 API only for those LAPACK routines it natively provides.  Therefore,
C users wishing to call an LAPACK routine not natively provided by ATLAS
will need to build the full LAPACK library as outlined in this section,
and then call the FORTRAN API.  Note that we are currently working on an
add-on package that provides a uniform C API for any FORTRAN77 API LAPACK,
but this work may not be available for some time (or indeed, ever).

Here are the steps to get a full FORTRAN77 API LAPACK which uses ATLAS's
improved routines when possible, and the standard netlib routines when not:
\begin{enumerate}
\item Download and install netlib's LAPACK (see~\S\ref{sec-netlib-lp} for
      help with this).  Assume this is installed in {\tt NLAPACKdir}
      (eg., my {\tt NLAPACKdir} might be {\tt /home/whaley/numerical/lapack-3.1.1}).
\item During configure of ATLAS, pass {\tt configure} the flag so that it will
      add your previously install netlib LAPACK to its own LAPACK routines
      by passing the following flags to your ATLAS configure line:
\vspace*{-0.1in}
\begin{verbatim}
      --with-netlib-lapack=NLAPACKdir/<your lapack library name>
\end{verbatim}
      (eg., 
      {\tt --with-netlib-lapack=/home/whaley/numerical/lapack-3.1.1/lapack\_linux.a}).
\item Do rest of ATLAS install as normal, full FORTRAN77 LAPACK API will
      be available in {\tt BLDdir/lib/liblapack.a}.
\end{enumerate}

These directions allow you to produce a full LAPACK when doing an ATLAS
install.  Section~\ref{sec-lapack-post} describes how to easily add netlib's
LAPACK to an already existing ATLAS build.
}

\subsubsection{Obtaining netlib's LAPACK}
\label{sec-netlib-lp}

You can download the LAPACK reference implementation from
{\tt www.netlib.org/lapack/}.  As of this writing, the newest LAPACK
tarfile was {\tt http://www.netlib.org/lapack/lapack.tgz}.

For more standard information on LAPACK, please scope the following URLs:
\vspace*{-0.1in}
\begin{itemize}
\item \verb|http://www.netlib.org/lapack/|
\vspace*{-0.1in}
\item \verb|http://www.netlib.org/lapack/lawn81/index.html|
\vspace*{-0.1in}
\item \verb|http://www.netlib.org/lapack/lawn41/index.html|
\vspace*{-0.1in}
\item \verb|http://www.netlib.org/lapack/release_notes.html|
\vspace*{-0.1in}
\item \verb|http://www.netlib.org/lapack/lug/index.html|
\end{itemize}

\subsection{Changing the compilers and flags that ATLAS uses for the build}
\label{sec-comps}

ATLAS defines eight different compilers and associated flag macros in 
its {\tt Make.inc} which are used to compile various files during the
install process.  ATLAS's {\tt configure} provides flags for changing both the
compiler and flags for each of these macros.  In the following list,
the macro name is given first, and the configure flag abbreviation is in
parentheses:
\begin{enumerate}
\item {\tt XCC} ({\tt xc}): C compiler used to compile ATLAS's build harness routines
                 (these never appear in any user-callable library)
\item {\tt GOODGCC} ({\tt gc}): gcc with any required architectural flags 
      (eg. {\tt -m64}), which will be used to assemble cpp-enabled assembly
      and to compile certain multiple implementation routines that specifically
      request {\tt gcc}
\item {\tt F77} ({\tt if}): FORTRAN compiler used to compile ATLAS's FORTRAN77
                API interface routines.
\item {\tt ICC} ({\tt ic}): C compiler used to compile ATLAS's C API interface
                           routines.
\item {\tt DMC} ({\tt dm}): C compiler used to compile ATLAS's generated
                double precision (real and complex) matmul kernels
\item {\tt SMC} ({\tt sm}): C compiler used to compile ATLAS's generated
                single precision (real and complex) matmul kernels
\item {\tt DKC} ({\tt dk}): C compiler used to compile all other double
                precision routines (mainly used for other kernels, thus the
                {\tt K})
\item {\tt SKC} ({\tt sk}): C compiler used to compile all other single
                precision routines (mainly used for other kernels,
                thus the {\tt K})
\end{enumerate}

It is almost \underline{never} a good idea to change {\tt DMC} or {\tt SMC},
and it is only very rarely a good idea to change {\tt DKC} or {\tt SKC}.
For ATLAS~3.8.0, all architectural defaults are set using gcc 4.2 only
(the one exception is MIPS/IRIX, where SGI's compiler is used).  In most
cases, switching these compilers will get you worse performance and accuracy,
even when you are \underline{\em absolutely} sure it is a better compiler
and flag combination!  In particular we tried the Intel compiler {\tt icc} 
(called {\tt icl} on Windows) on Intel x86 platforms,
and overall performance was lower than {\tt gcc}.  Even worse, 
from the documentation
{\tt icc} does not seem to have any firm IEEE floating point compliance unless
you want to run so slow that you could compute it by hand faster.  This means
that whenever {\tt icc} achieves reasonable performance, I
have no idea if the error will be bounded or not.  I could not obtain
access to {\tt icc} on the Itaniums, where {\tt icc} has historically been
much faster than {\tt gcc}, but I note that the performance of 
{\tt gcc4.2} is much better than {\tt gcc3} for most routines, so {\tt gcc}
may be the best compiler there now as well.

There is almost never a need to change {\tt XCC}, since it doesn't affect
the output libraries in any way, and we have seen that changing the kernel
compilers is a bad idea.  However, what if you yourself use a non-gnu
compiler, like Intel's {\tt icc} or {\tt ifort}, then what you need to
do is tell ATLAS to compile its interface routines with your compilers,
which is discussed in Section~\ref{sec-cc-iface}.  Another common problem
is that your OS has been built with an older {\tt gcc} whose libraries
are incompatible with {\tt gcc 4.2}.  In this case, creating an executable
with {\tt gcc4.2} can cause problems, and so what you want to do is keep
{\tt gcc3} as you default compiler (compiling ATLAS interface routines
with it, as well as using it for all linking) but
compile the ATLAS kernel routines with {\tt gcc4}.  This
case is discussed in Section~\ref{sec-cc-4v3}.
For those who insist on monkeying with
other compilers, Section~\ref{sec-cc-override} gives some guidance.
Finally installing ATLAS without a FORTRAN compiler
is discussed in Section~\ref{sec-nof77}.

\subsubsection{Changing ATLAS interface compilers to match your usage}
\label{sec-cc-iface}

As mentioned, ATLAS typically gets its best performance when compiled
with {\tt gcc} using the flags that ATLAS automatically picks for
your platform (this assumes you are installing on a system that ATLAS
provides architectural defaults for).  However, you can vary the interface
(API) compilers without affecting ATLAS's performance.  Since most compilers
are interoperable with {\tt gcc} this is what we recommend you do if you
are using a non-default compiler.  Note that almost all compilers can
interoperate with {\tt gcc}, though you may have to throw some special
flags (eg., {\tt /iface:cref} for MSVC++).

The configure flags to override the C interface compiler and flags are:
\vspace*{-0.1in}
\begin{verbatim}
   -C ic <C compiler> -F ic '<compiler flags>'
\end{verbatim}

The configure flags to override the FORTRAN interface compiler and flags are:
\vspace*{-0.1in}
\begin{verbatim}
   -C if <FORTRAN compiler> -F if '<compiler flags>'
\end{verbatim}

A few example will help here.  If I wanted to use Intel's FORTRAN and C
compilers under windows on a P4, I could issue:
\vspace*{-0.1in}
\begin{verbatim}
   -C if ifort -F if '-O2 -fltconsistency -nologo' \
   -C ic icl -F ic '-QxN -O3 -Qprec -fp:extended -fp:except -nologo -Oy'
\end{verbatim}

On the same system, if I wanted to use Intel for FORTRAN and MSVC++ for C:
\vspace*{-0.1in}
\begin{verbatim}
   -C if ifort -F if '-O2 -fltconsistency -nologo' \
   -C ic icl -F ic '-Oy -Ox -arch:SSE2 -nologo'
\end{verbatim}

For Windows, we can note a couple of things.  First, while these flags are
straight from the Windows compiler documentation, we have replaced the
Windows `{\tt /}' flag character with the Unix `{\tt -}' flag character.
This is because ATLAS doesn't call native Windows compilers directly, but
rather calls a wrapper routine that makes these compilers work with make
like a standard Unix compiler.  The second thing to notice is that we don't
have to say to use the {\tt /iface:cref} flag, because this same wrapper
always throws this flag (ATLAS does not work with the other rather bizarre
naming strategies).

For a non-Windows example, assume you use the Sun Workshop compilers available
under Solaris.  You can instruct {\tt configure} to use them for building the
APIs rather than the gnu compilers with something like:
\vspace*{-0.1in}
\begin{verbatim}
   -C if f77 -F if '-dalign -native -xO5' \
   -C ic cc -F ic '-dalign -fsingle -xO5 -native'
\end{verbatim}

\subsubsection{Compiling ATLAS with gcc 4.2 when your OS uses an
incompatible gcc}
\label{sec-cc-4v3}

As previously mentioned, {\tt gcc4.2} is what the architectural defaults
are built for, and previous versions are likely to hurt your performance.
For systems with {\tt gcc4.1} (the worst-performing {\tt gcc} for x86
machines), you can usually just install {\tt gcc4.2}, and change your
path so that {\tt gcc4.2} is your default compiler.  However, between
major releases the {\tt gcc} system libraries change too much for this
to work right.  Therefore, if your OS was built with {\tt gcc3}, for example,
what will often happen is that executables built with {\tt gcc4} will
not be able to run, unless you fiddle with your {\tt LD\_LIBRARY\_PATH} so
that the {\tt gcc4} libraries are found before those of {\tt gcc3}.
However, if you do this, then often {\tt gcc3}-built objects, which include
the majority of things you use every day (eg., editors), won't run because
they find the {\tt gcc4} libraries instead of the expected libs from {\tt gcc3}!

Therefore, you don't want to make {\tt gcc4.2} your default compiler, but
you want to have ATLAS use it to compile all the kernel routines, while
compiling interface routines and doing any linking with {\tt gcc3}.  To
do this, leave the system {\tt gcc} as the default one in your path, but
pass the following flag to {\tt configure}:
\vspace*{-0.1in}
\begin{verbatim}
   -Ss kern <path to gcc4.2>
\end{verbatim}

This tells ATLAS to use all non-kernel compilers as normal, but to change
all kernel compilers to the given compiler.  Therefore, if I have installed
{\tt gcc4.2} on my {\tt gcc3}-built OS in my own home area at
{\tt /home/whaley/local/gcc42}, I would add something like:
\vspace*{-0.1in}
\begin{verbatim}
   -Ss kern /home/whaley/local/gcc42/bin/gcc
\end{verbatim}

\subsubsection{Rough guide to overriding ATLAS's compiler choice/changing flags}
\label{sec-cc-override}
Previous sections have discussed the more useful cases of overriding
ATLAS's compiler and flags, which typically leave ATLAS's kernel compilers
alone.  Users often wish to add flags or change arbitrary compilers, however.
This is rarely a good idea, and almost always provides reduced performance.
However, you can do it.  You can find more details by passing 
{\tt --help} to {\tt configure}.  

As previously mentioned (\S\ref{sec-cc-iface}), you can specify what compiler
(flag setting) to
override by passing the appropriate abbreviation to the {\tt -C} ({\tt -F})
configure flags in order to change the compiler (compiler flags). For
example, you would pass {\tt -C if} to override interface FORTRAN compiler.
{\tt configure} also supports appending certain compiler flags, so that user
flags are simply added to the defaults that ATLAS uses.  This is done:
\vspace*{-0.1in}
\begin{verbatim}
   -Fa <abbr> '<comp flags to append>'
\end{verbatim}
where \verb|<abbr>| is one of:
\begin{itemize}
\item One of the already discussed compiler abbreviations 
      (eg, {\tt xc}, {\tt ic}, {\tt if}, {\tt sk}, {\tt dc}, {\tt sm} 
       or {\tt dm})
\item {\tt al}: all compilers (including FORTRAN) except {\tt GOODGCC}
\item {\tt alg} all compilers (including FORTRAN) including {\tt GOODGCC}
\item {\tt  ac}: all C compilers except {\tt GOODGCC}
\item {\tt acg}: all C compilers including {\tt GOODGCC}
\end{itemize}

Therefore, by passing the following to {\tt configure}:
\vspace*{-0.1in}
\begin{verbatim}
   -Fa acg '-DUsingDynamic -fPIC'
\end{verbatim}

We would have all C routines compiled with {\tt -fPIC}, and also have the
macro {\tt UsingDynamic} defined (ATLAS does not use this macro, this is
for example only).

The compiler overriding flag {-C} can also take the abbreviation {\tt ac}
which will override all C compilers except {\tt GOODGCC} with the given
C compiler.  There is currently no flag to override {\tt GOODGCC} on
the command line, so if you need to do this, you will need to edit the
output {\tt Make.inc} after configure.

As an example, if I want to use SunOS's {\tt f77} rather than {\tt gfortran},
I could pass the following compiler and flag override:
\vspace*{-0.1in}
\begin{verbatim}
   -C if f77    -F if 'dalign -native -xO5'
\end{verbatim}

{\bf IMPORTANT NOTE}: If you change the default flags in any way for
the kernel compilers
(even just appending flags), you may reduce performance.  Therefore once
your build is finished, you should make sure to compare your achieved
performance against what ATLAS's architectural defaults achieved.  See
Section~\ref{sec-time-nondefault} for details on how to do this.
If your compiler is a different version of {\tt gcc}, you may also
want to tell ATLAS not to use the architectural defaults, as described
in Section~\ref{sec-nodefaults}.

\subsubsection{Installing ATLAS when you don't have access to a FORTRAN
compiler}
\label{sec-nof77}

By default, ATLAS expects to find a FORTRAN compiler on your system.  If
you cannot install a FORTRAN compiler, you can still install ATLAS, but
ATLAS will be unable to build the FORTRAN77 APIs for both BLAS and LAPACK.
Further, certain tests will not be able to even compile, as their testers
are at least partially implemented in FORTRAN.  To tell ATLAS you wish
to install w/o a FORTRAN compiler, simply add the flag:
\vspace*{-0.1in}
\begin{verbatim}
   --nof77
\end{verbatim}
to your {\tt configure} command.

{\tt IMPORTANT NOTE}: When you install ATLAS w/o a FORTRAN compiler,
your build step will end with a bunch of {\tt make} errors about being
unable to compile some FORTRAN routines.  This is because the {\tt Makefile}s
always attempt to compile the FORTRAN APIs: they simply continue the install
if they don't succeed in building them.  So, just because you get a lot
of {\tt make} messages about FORTRAN, don't assume your library is messed
up.  As long as {\tt make check} and {\tt make time} say your {\tt -nof77}
install is OK, you should be fine.

\newpage
\subsubsection{Improving flag selection using mmflagsearch}
\label{sec-flagsearch}

If you are on a architecture or using a gcc for which configure does
not suggest flags, or if you believe the present set is out-of-date,
you can quickly search through a host of compiler flags to find the
best set for a given gemm kernel using the specialized routine
mmflagsearch.c.  To do this, you need a working install, typically
installed with your best guess at good flags.  Now, in your
{\tt BLDdir/tune/blas/gemm} directory, issue {\tt make xmmflagsearch}.

The idea behind this search is that it takes an ATLAS GEMM kernel description
file (output from one of the ATLAS searches), and then tries a series of
flags given in another file, and returns to you the best combination found.
The important flags are:
\begin{description}
\item [{\tt -p [s,d,c,z]}] : set type/precision prefix
\vspace*{-0.1in}
\item [{\tt -f $<$flagfile$>$}]: file containing all flags to try
\vspace*{-0.1in}
\item [{\tt -m $<$mmfile$>$}]: mmsearch output file describing kernel to time
\end{description}

The {\tt mmfile} is the matmul kernel that you wish to use to find the
best flags, and if this argument is omitted the search will automatically
read {\tt res/$<$pre$>$gMMRES.sum}, which is the best kernel found for the
during the prior install using the scalar ANSI~C generator.  If bad flags
have caused this search to generate a weird file, you can copy this file
to a new name, and then hand edit it to have the features you like.

In the {\tt flagfile}, any line beginning with `\verb+#+' is ignored.
This file has a special format that is more easily understood once 
you understand the method of the search.  The user provides one line
for any flags that should always appear (examples include things like
{\tt -fPIC}, {\tt -m64}, {\tt -mcpu=XXX}, etc.).  This is given on the
first line.

Now, the way the search is going to work is that first it will find
the appropriate optimization level and fundamental flag combination,
which will be searched by trying all combinations of these flags.
Once these baseline flags are determined, all remaining flags will
be tried one after the other using a greedy linear search.
With this in mind, the format of this file is:
\begin{verbatim}
Required flags for all cases (eg. -fPIC -m64 -msse3 -mfpmath=sse)
<N>      Number of optimization level lines
<lvlflagset1>
....
<lvlflagsetN>
<F>      Number of fundamental flag lines
<fundflagset1>
....
<fundflagsetF>
# Now list any number of modifier flag lines
flag set 1
flag set 2
...
flag set X
\end{verbatim}

So, the way this search is going to work is that we will first try all
$N \times (F+1)$ combinations of the levels and fundamental flags, and
choose a best-performing set.  We will then try adding every provided modifier
flag line to the best found combination.  The best performing list will be 
given.  

To create such a flag file one usually scopes the compiler documentation, 
and finds all performance-oriented flags.  For gcc, you can make mmflagsearch
give you a template that includes all non-architecture-specific optimization
flags (as found in the documentation for gcc 4.2)
by running {\tt ./xmmflagsearch -f gcc}.  This will create a directory
called {\tt gccflags.txt} in the current directory, which presently has
a format like:
\begin{verbatim}
REPLACE THIS LINE WT ARCH-DEP FLAGS ALWAYS USED (eg, -fPIC -m64 -msse3)
4
-O2
-O1
-O3
-Os
6
-fschedule-insns 
-fno-schedule-insns 
-fschedule-insns2
-fno-schedule-insns2
-fexpensive-optimizations
-fno-expensive-optimizations
# Flags to probe once optimization level is selected
...
whole boatload of flags
...
\end{verbatim}

Now lets see an example of this working on my ARM embedded machine.  The first
thing I do is replace the first line with my mandatory flags:
\begin{verbatim}
-mfpu=vfpv3 -mcpu=cortex-a8
\end{verbatim}
I then add two architecture-specific flags to the auto-generated general
flag list (might want to try a lot more, this is just an example), 
which in this case are:
\begin{verbatim}
-mtune=cortex-a8
-mno-thumb
\end{verbatim}

An extract of this search is shown in Figure~\ref{fig-flagsrch}.

\begin{figure}[p]
\begin{footnotesize}
\begin{verbatim}
FINDING BEST FLAGS USING MATMUL KERNEL:
ID=0 ROUT='dgmm.c' AUTH='Whaley/emit_mm' TA='T' TB='N' \
   MULADD=1 PREF=1 LAT=5 NFTCH=2 IFTCH=6 FFTCH=1 KBMAX=0 KBMIN=0 KU=1 NU=5  \
   MU=4 MB=80 NB=80 KB=80 L14NB=0 PFBCOLS=0 PFABLK=0 PFACOLS=0 STFLOAT=0  \
   LDFLOAT=0 AOUTER=0 LDAB=1 BETAN1=0 LDISKB=1 KUISKB=0 KRUNTIME=0 NRUNTIME=0  \
   MRUNTIME=0 LDCTOP=0 X87=0  \
   MFLOP=5.848105e+02
FINDING BEST FLAG SETTINGS FOR THIS MATMUL KERNEL:
...Trying optlvls using base flags: '-mfpu=vfpv3 -mcpu=cortex-a8'
   1. mf=557.83, flags='-O2'
    ---> Opt level '-O2' is better!
   2. mf=558.18, flags='-O2 -fschedule-insns'
   3. mf=564.07, flags='-O2 -fno-schedule-insns'
    ---> Opt combo '-O2 -fno-schedule-insns' is better!
   4. mf=564.53, flags='-O2 -fno-schedule-insns -fschedule-insns2'
   5. mf=572.04, flags='-O2 -fno-schedule-insns -fno-schedule-insns2'
    ---> Opt combo '-O2 -fno-schedule-insns -fno-schedule-insns2' is better!
   6. mf=572.11, flags='-O2 -fno-schedule-insns -fno-schedule-insns2 
                        -fexpensive-optimizations'
   7. mf=573.24, flags='-O2 -fno-schedule-insns -fno-schedule-insns2 
                        -fno-expensive-optimizations'
   8. mf=572.34, flags='-O1'
   ... Bunch of cases elided ...
  27. mf=562.53, flags='-Os -fno-schedule-insns -fno-schedule-insns2 
                        -fexpensive-optimizations'
  28. mf=563.81, flags='-Os -fno-schedule-insns -fno-schedule-insns2 
                        -fno-expensive-optimizations'
...All cases using flags: '-O2 -mfpu=vfpv3 -mcpu=cortex-a8 -fno-schedule-insns
                           -fno-schedule-insns2'
  29. mf=571.41, flags='-mtune=cortex-a8'
  30. mf=572.32, flags='-mno-thumb'
  31. mf=571.02, flags='-fno-cprop-registers'
   ... Bunch of cases elided  ...
  42. mf=574.69, flags='-fomit-frame-pointer'
  43. mf=572.39, flags='-foptimize-register-move'
  44. mf=571.79, flags='-fno-optimize-register-move'
  45. mf=592.68, flags='-fprefetch-loop-arrays'
    ---> Adding flag '-fprefetch-loop-arrays'!
  46. mf=572.65, flags='-fno-prefetch-loop-arrays'
   ... Bunch of cases elided  ...
  90. mf=594.61, flags='-falign-loops=8'
  91. mf=594.65, flags='-falign-loops=16'
  92. mf=594.37, flags='-falign-loops=32'
BEST FLAGS GIVE MFLOP=592.68 (6.25% improvement over first case):
   '-O2 -mfpu=vfpv3 -mcpu=cortex-a8 -fno-schedule-insns -fno-schedule-insns2 
    -fprefetch-loop-arrays'
\end{verbatim}
\end{footnotesize}
\caption{Result of {\tt ./xmmflagsearch -p d -f gccflags.txt} on ARM}
\label{fig-flagsrch}
\end{figure}

\clearpage
\subsection{Building dynamic/shared libraries}
\label{sec-shared}
ATLAS natively builds static libraries (i.e. libs that usually end in
`\texttt{.a}' under Unix and `\texttt{.lib}' under windows).  
ATLAS always builds such a library,
but it can also optionally be requested to build a dynamic/shared library
(typically ending in .so for Unix or .dll windows) as well.  In order to do so,
you must tell ATLAS up front to compile with the proper flags (the same is
true when building netlib's LAPACK, see \S\ref{sec-lapack} for more details).  
As long as you are using gnu compilers, all you need to add to your 
{\tt configure} command is:
\vspace*{-0.1in}
\begin{verbatim}
   --shared
\end{verbatim}

For any non-gnu compiler, you will additionally have to tell configure what
flags are
needed to tell the compiler to produce a shared library-compatible object
file (you can skip this step if the compiler does so by default.

ATLAS always builds the static libraries, but the {\tt --shared} command
adds an additional step to the install which also builds two shared libraries:
\begin{description}
\item [libsatlas.[so,dylib,dll]]: This library contains all serial APIs
      (serial lapack, serial BLAS), 
      and all ATLAS symbols needed to support them.
\item [libtatlas.[so,dylib,dll]]: This library contains all parallel
      APIs (parallel LAPACK and parallel BLAS) 
      and all ATLAS symbols needed to support them.
\end{description}

After your build is complete, you can {\tt cd} to your {\tt OBJdir/lib} 
directory, and
ask ATLAS to build the {\tt .so} you want.  If you want all libraries, including
the FORTRAN77 routines, the target choices are:
\begin{description}
\item [shared] : create shared versions of ATLAS's sequential libs
\item [ptshared] : create shared versions of ATLAS's threaded libs
\end{description}
If you want only C routines (eg., you don't have a FORTRAN compiler):
\begin{description}
\item [cshared] : create shared versions of ATLAS's sequential libs
\item [cptshared] : create shared versions of ATLAS's threaded libs
\end{description}

Note that this support for building dynamic libraries is new in this release,
and not well debugged or supported, and is much less likely to work for
non-gnu compilers.

\textbf{IMPORTANT NOTE}: Since gcc uses one less integer register when
      compiling with this flag, this
      could potentially impact performance of the architectural defaults,
      but we have not seen it so far.  Therefore, do not throw this flag
      unless you want dynamic libraries.  If you want both static and dynamic
      libs, the safest thing is probably to build ATLAS twice, once static
      and once dynamic, rather than getting both from a dynamic install.


\subsection{Changing the way ATLAS does timings}
\label{sec-time-flags}

By default ATLAS does all timings with a CPU timer, so that the install
can be done on a machine that is experiencing relatively heavy load.  However,
CPU time has very poor resolution, and so this makes the timings less repeatable
and provides for only a rough idea of overall performance.  Therefore, if
you are installing ATLAS on a machine which is not heavily loaded, you will
want to improve your install by instructing ATLAS to use one of its higher
resolution wall timers.

For x86 machines, ATLAS has access to a cycle accurate wall timer, assuming
you are using {\tt gcc} as your interface compiler (we use {\tt gcc}'s inline
assembly to enable this timer -- under Linux, Intel's {\tt icc} also supports
this form of inline
assembly).  ATLAS needs to be able to translate the cycle count returned by
this function into seconds, so you must pass your machine's clock rate to
ATLAS.  In order to do this, you add the following flags to your 
configure flags:
\vspace*{-0.1in}
\begin{verbatim}
   -D c -DPentiumCPS=<your Mhz>
\end{verbatim}
So, for my 2.4Ghz Core2Duo, I would pass:
\vspace*{-0.1in}
\begin{verbatim}
   -D c -DPentiumCPS=2400
\end{verbatim}

If you are not on an x86 machine, or if your interface compiler is not {\tt gcc}
(or {\tt icc} if on Linux), then you cannot use the above cycle-accurate
wall timer.  However, wall time is still much more accurate than CPU time,
so you can indicate ATLAS should use its wall timer for the install by passing
the flag:
\vspace*{-0.1in}
\begin{verbatim}
   -D c -DWALL
\end{verbatim}

Note that on Windows XP/NT/2000, this should still get you a cycle-accurate
walltime, since it calls some undocumented Windows APIs that purport to do so. 
For Solaris, the high resolution timer {\tt gethrtime} will be used.  For
all other OSes, this will call a standard wall timer such as {\tt gettimeofday},
which is still usually much more accurate than the CPU timer.

\Wskip{
\subsection{Manually setting OS, architecture, assembly dialect}
\label{sec-OS-override}

Configure's selection of operating system, architecture, assembly dialect
and SIMD vectorization type are all controlled by enumerated types.
Occasionally, configure will misdetect one of these values and so
configure provides
flags for overriding configures detecting of these features.

\subsubsection{Manually overriding the assembly dialect}
\subsubsection{Manually overriding SIMD vectorization type}
\label{sec-SIMD-override}
Also, the installer
may occasionally want to refrain from using a feature that this machine
possesses, but other machines that plan to use the same library don't.
For instance, the installation machine might have SSE3, but the installer
wants to use the same library on machines that possess only SSE2, and so
the installer wants to use only SSE2, even though this will make the library
sub-optimal on the machine possessing SSE3.  
}

\subsection{Various other flags}
\label{sec-flag-other}

\subsubsection{Changing pointer bitwidth (64 or 32 bits)}
\label{sec-bitwidth}
Most modern platforms allow for compiling libraries to handle either 32 or
64 bit address spaces.  On the x86, this selection strongly affects the ISA
used (eg., whether to use IA32 or x86-64).  The x86-64 ISA, with 16 rather
than 8 registers, is more
amenable to optimization than the IA32, so if the user has no preference,
64-bit pointers are recommended.
If ATLAS's guess is not correct,
you can tell configure what address space to build for.  In order to force
32-bit pointer width, pass the flag:
\vspace*{-0.1in}
\begin{verbatim}
   -b 32
\end{verbatim}
and in order to force 64 bit pointers, pass:
\vspace*{-0.1in}
\begin{verbatim}
   -b 64
\end{verbatim}
(the {\tt b} stands for bitwidth).

This tells ATLAS to throw the appropriate compiler flags for compilers it
knows about, as well as effecting various configure probes.  Therefore, if
you override ATLAS's compiler choices, be sure that you give the correct
flags to match this setting.

\subsubsection{Changing configure verbosity}
\label{sec-verb}
{\tt configure} does a series of architectural probes to figure out how to do
an install on your system.  Many of the probes that are run don't produce
output during the configure step.  You can tell {\tt configure} that you want
to see more output by cranking up the verbosity.  Presently, maximum 
verbosity is enabled by adding the flag:
\vspace*{-0.1in}
\begin{verbatim}
   -v 2
\end{verbatim}

\subsubsection{Controlling where ATLAS will move files to during install step}
ATLAS supplies some flags to control where ATLAS will move files to when
you do the {\tt make install} step (\S\ref{sec-install}).  These flags
are taken from gnu {\tt configure}, and they are:
\begin{itemize}
\item \verb|--prefix=<dirname>| : Top level installation directory.
      include files will be moved to \verb|<dirname>/include| and
      libraries will be moved to \verb|<dirname>/lib|.
      Default: {\tt /usr/local/atlas}
\item \verb|--incdir=<dirname>| : Installation directory for ATLAS's
      include files.  Default: {\tt /usr/local/atlas/include}.
\item \verb|--incdir=<dirname>| : Installation directory for ATLAS's
      libraries. \\ Default: {\tt /usr/local/atlas/lib}.
\end{itemize}

\subsubsection{Telling ATLAS to ignore architectural defaults}
\label{sec-nodefaults}
Architectural defaults are partial results of past searches when
the compiler and architecture are known.  They allow you skip the
full ATLAS search, which makes install time much quicker.  They also
ensure that you have good results, since they typically represent
several searches and/or user intervention into the usual search so
that maximum performance is found.  This doesn't typically mean a huge
performance difference, since the empirical search usually does an adequate
job, but it often provides a few extra percentage points of performance.  Also,
occasionally the empirical search will, due to machine load or other timing
problems, produce inadequate code, and using the architectural defaults
prevents this from happening.

By default, ATLAS automatically uses the architectural defaults anytime
it has results for the given architecture and compiler.  However, the
compiler detection is based on the compiler name, not version, and so
ATLAS's architectural defaults for gnu {\tt gcc4.2} might not be best
for {\tt gcc3} or apple's {\tt gcc}, etc, even though {\tt configure} would
use the architectural defaults in such cases.

So, there are times when you want to tell ATLAS to ignore any architectural
defaults it might have.  Common reasons include the fact that you have
overridden the compiler flags ATLAS uses, or are using an earlier version
of the supported compiler.  In these cases, the best idea is often to
install both with and without the architectural defaults, and compare timings.
If both your installs (homegrown-compiler/flags+archdef, 
homegrown-compiler/flags+search) are slower than the architectural defaults
using the default compiler, you should probably install the default 
compiler.  However, if your results are largely the same, you know your
changes haven't depressed performance and so it is OK to use the generated
libraries (see Section~\ref{sec-time} for details on timing an ATLAS
install).  If your timing results are substantially better, and you haven't
enabled IEEE-destroying flags, you should send your improved compiler and
flags to the ATLAS team!

To force ATLAS to ignore the architectural defaults (and thus to perform
a full ATLAS search), pass the following flags to {\tt configure}:
\begin{verbatim}
   -Si archdef 0
\end{verbatim}

\section{The ATLAS build step}
\label{sec-build}

This is the step where ATLAS performs all its empirical tuning, and then
uses the discovered kernels to build all required libraries.  It uses
the {\tt BLDdir} created by the configure step, and is invoked from
the {\tt BLDdir} with the {\tt make build} command, or simply by {\tt make}.
This step can be quite long, depending on your platform and whether or
not you use architectural defaults.  For a system like the Core2Duo with
architectural defaults, the build step may take 10 or 20 minutes, while in
order to complete a full ATLAS search on a slower platform (eg. MIPS) could
take anywhere between a couple of hours and a full day.

\section{The ATLAS check step}
\label{sec-check}
In this optional step, ATLAS runs various testers in order to make sure
that the generated library is not producing completely bogus results.
For each precision, ATLAS runs the standard BLAS testers (both C and F77
interface), and then various of ATLAS's homegrown testers that appear
in {\tt ATLAS/bin}.  If you have installed without a FORTRAN compiler,
then the standard BLAS testers cannot be run (the standard BLAS testers,
downloadable from netlib, require FORTRAN even to test the C interface),
and so your testing will be less comprehensive.

There are two possible targets, {\tt check} which tests ATLAS's serial
routines, and {\tt ptcheck} which check the parallel routines.  You cannot
run {\tt ptcheck} if you haven't installed the parallel libraries.
This step is invoked from {\tt BLDdir} by typing:
\begin{verbatim}
   make check      # test serial routines
   make ptcheck    # check parallel routines
\end{verbatim}

Both of these commands will first do a lot of compilation, and then they
will finish with results such as:
\vspace*{-0.1in}
\begin{verbatim}
core2.home.net. make check
...................................................
..... A WHOLE LOT OF COMPILATION AND RUNNING ......
...................................................
DONE BUILDING TESTERS, RUNNING:
SCOPING FOR FAILURES IN BIN TESTS:
fgrep -e fault -e FAULT -e error -e ERROR -e fail -e FAIL \
        bin/sanity.out
8 cases: 8 passed, 0 skipped, 0 failed
4 cases: 4 passed, 0 skipped, 0 failed
8 cases: 8 passed, 0 skipped, 0 failed
4 cases: 4 passed, 0 skipped, 0 failed
8 cases: 8 passed, 0 skipped, 0 failed
4 cases: 4 passed, 0 skipped, 0 failed
8 cases: 8 passed, 0 skipped, 0 failed
4 cases: 4 passed, 0 skipped, 0 failed
DONE
SCOPING FOR FAILURES IN CBLAS TESTS:
fgrep -e fault -e FAULT -e error -e ERROR -e fail -e FAIL \
        interfaces/blas/C/testing/sanity.out | \
                fgrep -v PASSED
make[1]: [sanity_test] Error 1 (ignored)
DONE
SCOPING FOR FAILURES IN F77BLAS TESTS:
fgrep -e fault -e FAULT -e error -e ERROR -e fail -e FAIL \
        interfaces/blas/F77/testing/sanity.out | \
                fgrep -v PASSED
make[1]: [sanity_test] Error 1 (ignored)
DONE
make[1]: Leaving directory `/home/whaley/TEST/ATLAS3.7.36.0/obj64'
\end{verbatim}

Notice that the {\tt Error 1 (ignored)} commands come from {\tt make}, and they
indicate that {\tt fgrep} is not finding any errors in the output files
(thus this {\tt make} output does not represent the finding of an error).
When true errors occur, the lines of the form
\vspace*{-0.1in}
\begin{verbatim}
8 cases: 8 passed, 0 skipped, 0 failed
\end{verbatim}

will have non-zero numbers for {\tt failed}, or you will see other tester
output discussing errors, such as the printing of large residuals.

As mentioned, this is really sanity checking, and it runs only a few tests
on a handful of problem sizes.  This is usually adequate to catch most
blatant problems (eg., compiler producing incorrect output).  More subtle
or rarely-occurring bugs may require running the LAPACK and/or full ATLAS
testers.  The ATLAS developer guide~\cite{atlas-devel} provides instructions
on how to use the full ATLAS tester, as well as help in diagnosing problems.
The developer guide is provided in the ATLAS tarfile as
{\tt ATLAS/doc/atlas\_devel.pdf}

\section{The ATLAS time step}
\label{sec-time}
In this optional step, ATLAS times certain kernel routines and reports their
performance as a percentage of clock rate.  Its purpose is to provide a
quick way to ensure that your install has resulted in a library that obtains
adequate performance.  If you are installing using architectural
defaults, this step will print a timing comparison against the performance
that the ATLAS maintainer got when creating the architectural defaults.
To invoke this step, issue the following command in your {\tt BLDdir}:
\vspace*{-0.1in}
\begin{verbatim}
   make time
\end{verbatim}

\begin{figure}[p]
\begin{footnotesize}
\begin{verbatim}
NAMING ABBREVIATIONS:
   kSelMM : selected matmul kernel (may be hand-tuned)
   kGenMM : generated matmul kernel
   kMM_NT : worst no-copy kernel
   kMM_TN : best no-copy kernel
   BIG_MM : large GEMM timing (usually N=1600); estimate of asymptotic peak
   kMV_N  : NoTranspose matvec kernel
   kMV_T  : Transpose matvec kernel
   kGER   : GER (rank-1 update) kernel
Kernel routines are not called by the user directly, and their
performance is often somewhat different than the total
algorithm (eg, dGER perf may differ from dkGER)


Reference clock rate=2394Mhz, new rate=2394Mhz
   Refrenc : % of clock rate achieved by reference install
   Present : % of clock rate achieved by present ATLAS install

                    single precision                  double precision
            ********************************   *******************************
                  real           complex           real           complex
            ---------------  ---------------  ---------------  ---------------
Benchmark   Refrenc Present  Refrenc Present  Refrenc Present  Refrenc Present
=========   ======= =======  ======= =======  ======= =======  ======= =======
  kSelMM      535.0   551.4    525.4   509.6    311.5   312.7    298.0   296.5
  kGenMM      175.5   174.0    175.5   173.6    160.5   159.7    165.4   166.9
  kMM_NT      145.2   143.7    149.3   150.7    135.3   131.0    132.3   134.3
  kMM_TN      163.2   158.0    161.1   164.6    148.7   144.8    146.0   155.4
  BIG_MM      510.1   544.5    504.0   545.9    307.7   301.5    293.0   304.9
   kMV_N      113.5   109.1    216.9   208.3     58.9    56.2     97.4    88.8
   kMV_T       89.9    85.9     94.6    96.4     47.2    44.4     74.1    77.1
    kGER      154.2   154.1    119.4   116.9     29.1    26.0     46.8    45.6
\end{verbatim}
\end{footnotesize}
\caption{Normal results for {\tt make time} on Core2Duo64SSE3}
\label{fig-tim}
\end{figure}
\begin{figure}[p]
\begin{footnotesize}
\begin{verbatim}
Reference clock rate=2200Mhz, new rate=1597Mhz
....
                   single precision                  double precision
            ********************************   *******************************
                  real           complex           real           complex
            ---------------  ---------------  ---------------  ---------------
Benchmark   Refrenc Present  Refrenc Present  Refrenc Present  Refrenc Present
=========   ======= =======  ======= =======  ======= =======  ======= =======
  kSelMM      335.5   338.8    329.4   331.6    178.9   180.8    180.3   178.7
  kGenMM      175.4   100.4    174.2   100.3    163.7    92.6    141.4    94.9
  kMM_NT      142.0    86.8    141.2    92.0    125.3    85.2    138.1    88.8
  kMM_TN      143.0    92.7    141.1    95.2    139.4    87.8    137.4    90.1
  BIG_MM      327.1   325.2    318.6   320.0    169.8   171.3    171.0   172.0
   kMV_N       61.4    35.5    139.3    98.9     47.2    30.7     71.9    74.2
   kMV_T       73.6    53.6     75.3    62.5     31.6    20.2     52.7    36.6
    kGER       43.6    28.8     91.8    65.1     23.7    18.3     46.8    40.3
\end{verbatim}
\end{footnotesize}
\caption{Timings results when architectural defaults are compiled with 
         substandard gcc4.1, rather than gcc4.2}
\label{fig-tim-gcc4}
\end{figure}

In Figure~\ref{fig-tim} we see a typical printout of a successful install,
in this case ran on my 2.4Ghz Core2Duo.
The {\tt Refrenc} 
columns provide the performance achieved by the architectural defaults when
they were
originally created, while the  {\tt Present} columns provide the results
obtained using the new ATLAS install we have just completed.  We see that
the {\tt Present} columns wins occasionally (eg. single precision real
{\tt kSelMM}), and loses sometimes (eg. single precision complex {\tt kSelMM}),
but that the timings are relatively similar across the board.  This tells
us that the install is OK from a performance angle.

As a general rule, performance for both data types of a particular precision
should be roughly comparable, but may vary dramatically between precisions
(due mainly to differing vector lengths in SIMD instructions).

The timings are normalized to the clock rate, which is why the clock rate
of both the reference and present install are printed.  It is expected that
as clock rates rise, performance as a percent of it may fall slightly
(since memory bus speeds do not usually rise in exact lockstep).  Therefore,
if I installed on a 3.2Ghz Core2Duo, I would not be surprised if the 
{\tt Present} install lost by a few percentage points in most cases.

True problems typically display a significant loss that occurs in a pattern.
The most common problem is from installing with a poor compiler, which will
lower the performance of most compiled kernels, without affecting the speed
of assembly kernels.  Figure~\ref{fig-tim-gcc4} shows 
such an example, where {\tt gcc 4.1} (a terrible
compiler for floating point arithmetic on x86 machines) has been used to
install ATLAS on an Opteron, rather than {\tt gcc 4.2}, which was the compiler
that was used to create the architectural defaults.
Here, we see that the present machine is actually slower than the machine
that was used to create the defaults, so if anything, we expect it to
achieve a greater percentage of clock rate.  Indeed, this is more or less
true of the first line, {\tt kSelMM}.  On this platform, {\tt kSelMM} is
written totally in assembly, and {\tt BIG\_MM} calls these kernels,
and so the {\tt Present} results are good for these rows.  All the other
rows show kernels that are written in C, and so we see that the use of
a bad compiler has markedly depressed performance across the board.  
Anytime you see a pattern such as this, the first thing you should check
is if you are using a recommended compiler, and if not, install and use
that compiler.

On the other hand, if only your {\tt BIG\_MM} column is depressed, it is
likely you have a bad setting for the {\tt CacheEdge} or the complex-to-real
crossover point (if the performance is depressed only for both complex types).

\subsection{Contrasting non-default install performance}
\label{sec-time-nondefault}
If you do not install using the architectural defaults, {\tt make time} will
only print out the {\tt Present} columns.  This gives you a good summary of
ATLAS's library performance, but it can be hard to tell what is good and
bad if you are not familiar with ATLAS on this hardware.  Sometimes, ATLAS
has architectural defaults for your platform, but your install doesn't use
them.  This is usually because the installer has specified the use of a
non-default compiler, or has explicitly asked that the architectural defaults
not be used, or has overridden the detection of the architecture, etc.  In
this case, {\tt make time} does not do the comparison
against the architectural defaults, and so only the {\tt Present} columns
are printed.

However, if you wish to ensure that your library is as good as one that
uses the architectural defaults, then you can manually tell the program
called by {\tt make time} ({\tt xatlbench} to do the comparison.  The most
common example would
be you have switched to an unsupported compiler (eg., the Intel compiler),
and now you want to see if the library you built using it is as fast or faster
than the one using the default {\tt gcc 4.2} compiler.  Another example would
be that you want to compare the performance of two closely related
architectures.  This is what we will do here, where we contrast the performance
of the 32 and 64 bit versions of the library on my Core2Duo.

In order to manually do a comparison between a present install and any of
the results stored in ATLAS's architectural defaults you'll need to
perform the following steps:
\begin{enumerate}
\item  {\tt make time} issued in the {\tt BLDdir} of your non-default install.
       This does the timings of the present build, and stores the results
       in {\tt BLDdir/bin/INSTALL\_LOG}.
\item {\tt cd SRCdir/CONFIG/ARCHS}, and find the tarfile containing the
      results you wish to compare against.  In our case, we choose
      {\tt Core2Duo32SSE3.tgz} to compare against our own Core2Duo64SSE
      results.
\item {\tt gunzip -c Core2Duo32SSE3.tgz | tar xvf -} untars the
      selected architectural results (replace {\tt Core2Duo32SSE3.tgz} with the
      tarfile you have selected in step\#2).
\item {\tt cd BLDdir}
\item \verb|./xatlbench -dp SRCdir/CONFIG/ARCHS/<ARCH> -dc BLDdir/bin/INSTALL_LOG|  \\
    {\tt xatlbench} is the program that compares two sets of results, with
    the {\tt -dp} pointing to the previous ({\tt Refrenc}) install result
    directory and {\tt -dc} pointing to the current ({\tt Present})
    install result directory.
\end{enumerate}

Figure~\ref{fig-tim-32v64} shows me doing this on my Core2Duo, with 
{\tt SRCdir} = {\tt /home/whaley/TEST/ATLAS3.7.36.0} and
{\tt BLDdir} = {\tt /home/whaley/TEST/ATLAS3.7.36.0/obj64}, where we compare
the present 64-bit install to the stored 32-bit install.
We see that the 64-bit install, which gets to use 16 rather than 8 registers,
is slightly faster for almost all kernels and precisions, as one might expect.

\begin{figure}[htbp]
\begin{footnotesize}
\begin{verbatim}
core2.home.net. cd /home/whaley/TEST/ATLAS3.7.36.0/obj64
core2.home.net. make time
..... lots of output .....
core2.home.net. pushd ~/TEST/ATLAS3.7.36.0/CONFIG/ARCHS/
core2.home.net. ls
BOZOL1.tgz          CreateTar.sh      MIPSICE964.tgz  POWER564.tgz
Core2Duo32SSE3/     HAMMER64SSE2.tgz  MIPSR1xK64.tgz  PPCG532AltiVec.tgz
Core2Duo32SSE3.tgz  HAMMER64SSE3.tgz  negflt.c        PPCG564AltiVec.tgz
Core2Duo64SSE3/     IA64Itan264.tgz   P432SSE2.tgz    USIV32.tgz
Core2Duo64SSE3.tgz  KillDirs.sh       P4E32SSE3.tgz   USIV64.tgz
CoreDuo32SSE3.tgz   Make.ext          P4E64SSE3.tgz
CreateDef.sh        Makefile          POWER432.tgz
CreateDirs.sh       MIPSICE932.tgz    POWER464.tgz
core2.home.net. gunzip -c Core2Duo32SSE3.tgz | tar xvf -
..... lots of output .....
core2.home.net. pushd
core2.home.net. ./xatlbench \
   -dp /home/whaley/TEST/ATLAS3.7.36.0/CONFIG/ARCHS/Core2Duo32SSE3 \
   -dc /home/whaley/TEST/ATLAS3.7.36.0/obj64/bin/INSTALL_LOG/
.....
Reference clock rate=2394Mhz, new rate=2394Mhz
.....
                   single precision                  double precision
            ********************************   *******************************
                  real           complex           real           complex
            ---------------  ---------------  ---------------  ---------------
Benchmark   Refrenc Present  Refrenc Present  Refrenc Present  Refrenc Present
=========   ======= =======  ======= =======  ======= =======  ======= =======
  kSelMM      539.0   551.4    496.5   509.6    299.4   312.7    289.0   296.5
  kGenMM      165.1   174.0    165.1   173.6    156.1   159.7    153.8   166.9
  kMM_NT      137.6   143.7    134.7   150.7    115.7   131.0    123.5   134.3
  kMM_TN      116.3   158.0    112.3   164.6    101.3   144.8    110.9   155.4
  BIG_MM      521.3   544.5    476.5   545.9    282.6   301.5    282.8   304.9
   kMV_N       69.0   109.1    206.9   208.3     56.3    56.2     69.4    88.8
   kMV_T       84.8    85.9    117.3    96.4     48.0    44.4     87.9    77.1
    kGER       90.1   154.1    114.2   116.9     27.9    26.0     41.5    45.6
\end{verbatim}
\end{footnotesize}
\caption{Comparing 32 and 64 bit libraries on a 2.4 Ghz Core2Duo}
\label{fig-tim-32v64}
\end{figure}

\subsection{Discussion of timing targets}
Presently, ATLAS times mostly kernel routines, which are used to build
higher level routines that then appear in the BLAS or LAPACK.
{\tt kSelMM} is the matrix multiply kernel that is being used for large
GEMM calls, which will be the best kernel found in the generator and
multiple implementation searches.  Therefore this kernel may be written
in assembly on some platforms.
{\tt kGenMM} is the fastest generated kernel that matches {\tt kSelMM},
and it may be used for some types of cleanup.  All generated kernels
are written in ANSI~C, and thus their peak performance will strongly
depend on the compiler being used.

{\tt kMM\_NT} and {\tt kMM\_TN} are two of the four generated kernels that
will be used for small-case GEMM when we cannot afford to copy the input
matrices.  The last two characters indicate the transpose settings.
The other two kernels' performance lies between these extremes:
{\tt NT} is typically the slowest kernel (all non-contiguous access),
and {\tt TN} is typically the fastest (all contiguous access).

{\tt BIG\_MM} is the only non-kernel timing we presently report, and it is
the speed found when doing a large GEMM call.  ``Large'' can vary by platform:
it is typically $M=N=K=1600$, except where we were unable to allocate that
much memory, where it will be less.  On many machines, this line gives you
a rough asymptotic bound on BLAS performance.

The next three lines report Level 2 BLAS kernel performance (the Level~2
BLAS' performance will follow these kernels in roughly the same way that
the Level~3 follow the GEMM kernels).

We should eventually supply an expanded timing comparison that would include
higher level timings, such as LAPACK routines and threaded performance,
but do not currently do so.

\section{The ATLAS install step}
\label{sec-install-step}
This final optional step instructs ATLAS to copy the created libraries
and include files into the appropriate directories, as specified in
the configure step.  This functionality is new, and so far
is not bullet-proof (for instance, it copies only static libraries,
and so presently fails to copy any dynamic libraries the user has built).
From your {\tt BLDdir}, it may be invoked by:
\vspace*{-0.1in}
\begin{verbatim}
   make install
\end{verbatim}

By default, this command will copy all the static libraries to
{\tt /usr/local/atlas/lib} and all the user-includable header files to
{\tt /usr/local/atlas/include}.  You may override this default directory
during the configure step using the gnu-like flags {\tt --prefix},
{\tt --incdir} and/or {\tt --libdir}.  Assuming you didn't issue
{\tt --incdir} or {\tt --libdir}, you can also override the prefix
directory at install time with the command:
\vspace*{-0.1in}
\begin{verbatim}
   make install DESTDIR=<prefix directory to install atlas in>
\end{verbatim}

\section{Example: Installing ATLAS with full LAPACK on Linux/AMD64}
\label{sec-example}
In this section, I show a complete ATLAS install, including installing
LAPACK.  We assume I have already downloaded the tarfiles 
{\tt atlas3.9.12.tar.bz2} and {\tt lapack.tgz} into the
{\tt /home/whaley/dload} directory.  
%We will assume we are installing
%both libraries in {\tt /home/whaley/numerics}.

\subsection{Figuring out configure flags}
\label{sec-ex-configure}
The system is a Fedora Core 8 system, which unfortunately uses the broken
{\tt gcc 4.1.2}, which would cripple ATLAS performance.  Therefore, prior to
installing ATLAS, I have installed gcc~4.2.1, with 
   {\tt --prefix=/home/whaley/local/gcc-4.2.1}
I therefore add the following lines to my {\tt .cshrc} so that ATLAS
will use this gcc (it is put first in the path), and will be able
to find the {\tt gcc 4.2} libraries:
\begin{footnotesize}
\vspace*{-0.1in}
\begin{verbatim}
set path = (/home/whaley/local/gcc-4.2.1/bin $path)
setenv LD_LIBRARY_PATH /home/whaley/local/gcc-4.2.1/lib64:/home/whaley/local/gcc-4.2.1/lib
\end{verbatim}
\end{footnotesize}

\noindent
I source the C shell startup file, and then check that I'm now getting the
correct compiler:
\vspace*{-0.1in}
\begin{verbatim}
etl-opt8>source ~/.cshrc
etl-opt8>gcc -v
Using built-in specs.
Target: x86_64-unknown-linux-gnu
Configured with: ../configure --prefix=/home/whaley/local/gcc-4.2.1 --enable-languages=c
Thread model: posix
gcc version 4.2.1
\end{verbatim}

Now, I don't need to pass a lot of flags to set what compiler to use,
since ATLAS will find {\tt gcc 4.2} as the first compiler, and it will
have the libraries it needs to work.  However, I want to build dynamic
libraries for this install, so I know I'll need to add the 
{\tt --shared} configure flag; config will automatically add the required
{\tt -fPIC} flag to all gnu compilers so they can build shared object code.

Now, I do a top on {\tt etl-opt8} (the machine name) and see that
I'm alone on the machine.  Therefore, I will want to
use the cycle-accurate x86-specific wall timer in order to improve the
accuracy of my install.  This requires me to figure out what the Mhz of
my machine is.
Under Linux, I can discover this with {\tt cat /proc/cpuinfo}, which tells
me {\tt cpu~MHz~:~2100.000}.  Therefore, I will throw
{\tt -D c -DPentiumCPS=2100}.

I want ATLAS to install the resulting libraries and header files in
the directory \\
{\tt /home/whaley/local/atlas}, so I'll pass
{\tt --prefix=/home/whaley/local/atlas} as well.

I want a 64 bit install, and to build a full LAPACK library, so I will also
want to throw {\tt -b 64} and \\
\verb|--with-netlib-lapack-tarfile=/home/whaley/dload/lapack.tgz|.

\Wskip{
\subsection{Creating source directories and building LAPACK}
\label{sec-ex-lapack}
Now that I know what ATLAS {\tt configure} flags I want to throw, I am ready to
begin installing LAPACK.  The first step is to create the {\tt numerics}
directory that will hold the libraries, and untar them:
\vspace*{-0.1in}
\begin{verbatim}
animal>cd ~/
animal>mkdir numerics
animal>cd numerics/
animal>bunzip2 -c ~/atlas3.8.0.tar.bz2 | tar xfm -
animal>mv ATLAS ATLAS3.8.0
animal>gunzip -c ~/dload/lapack-3.1.1.tgz | tar xfm -
animal>ls
ATLAS3.7.38/  lapack-3.1.1/
\end{verbatim}

Now, we are needing to set the LAPACK Make.inc appropriately.  First,
I go into the LAPACK directory, and copy the platform-specific {\tt make.inc}
to {\tt make.inc}.  In my case this is:
\vspace*{-0.1in}
\begin{verbatim}
animal>cd lapack-3.1.1/
animal>cp INSTALL/make.inc.LINUX make.inc
\end{verbatim}

I now edit the created {\tt make.inc} ({\tt vi make.inc}), and here are
the {\tt make} macros that I change:
\vspace*{-0.1in}
\begin{verbatim}
FORTRAN  = <want to set to ATLAS's F77 macro>
OPTS     = <want to set to ATLAS F77FLAGS macro>
DRVOPTS  = $(OPTS)
NOOPT    = <F77FLAGS w/o optimization>
LOADER   = $(FORTRAN)
LOADOPTS = $(OPTS)
TIMER    = <need to know what compiler I'm using to set>
\end{verbatim}

So far, I have only been able to fill in {\tt DRVOPTS}, {\tt LOADER} and
{\tt LOADOPTS},
which are defined in terms of the macros I've yet to fill in!  The reason
is that I want to use the same compiler and flags as ATLAS, so that
I'm sure my LAPACK library can interoperate with my ATLAS-tuned library.
I will set the {\tt FORTRAN} macro to the compiler indicated by ATLAS's
{\tt F77} macro, and {\tt OPTS} will be the same as {\tt F77FLAGS}.

So, I change to the ATLAS source directory, and produce a dry-run {\tt BLDdir}
in order to get this information by:
\vspace*{-0.1in}
\begin{verbatim}
animal>cd ../ATLAS3.7.38/
animal>mkdir bogus
animal>cd bogus/
animal>../configure -b 64 -D c -DPentiumCPS=2200 -Fa alg -fPIC
...................................................
............<A WHOLE LOT OF OUTPUT>................
...................................................
animal>fgrep "F77 =" Make.inc
   F77 = gfortran
animal>fgrep "F77FLAGS =" Make.inc
   F77FLAGS = -fomit-frame-pointer -mfpmath=387 -O2 -falign-loops=4 -fPIC -m64
\end{verbatim}

With this info in hand, I am ready to delete this
bogus directory, and go back and edit the LAPACK make.inc:
\vspace*{-0.1in}
\begin{verbatim}
animal>cd ..
animal>rm -rf bogus/
animal>cd ../lapack-3.1.1/
animal>vi make.inc
\end{verbatim}

I now fill in my {\tt make.inc} macros as:
\vspace*{-0.1in}
\begin{verbatim}
FORTRAN  = gfortran
OPTS     = -fomit-frame-pointer -mfpmath=387 -O2 -falign-loops=4 -fPIC -m64
DRVOPTS  = $(OPTS)
NOOPT    = -fomit-frame-pointer -mfpmath=387 -m64
LOADER   = $(FORTRAN)
LOADOPTS = $(OPTS)
TIMER    = INT_ETIME
\end{verbatim}

I chose the setting of {\tt TIMER} based on the fact that the example file's
comments said it is the correct setting when the compiler is {\tt gfortran}.

Now I perform the LAPACK install:
\vspace*{-0.1in}
\begin{footnotesize}
\begin{verbatim}
animal>make lib
  ./testdlamch; ./testsecond; ./testdsecnd; ./testversion )
make[1]: Entering directory `/home/whaley/numerics/lapack-3.1.1/INSTALL'
gfortran -fomit-frame-pointer -mfpmath=387 -O2 -falign-loops=4 -m64 -c lsame.f -o lsame.o
........................................................
.............<WHOLE LOT OF COMPILATION>.................
........................................................
ar cr ../../tmglib_LINUX.a slatms.o slatme.o slatmr.o slagge.o slagsy.o
slakf2.o slarge.o slaror.o slarot.o slatm2.o slatm3.o slatm5.o slatm6.o
clatms.o clatme.o clatmr.o clagge.o claghe.o clagsy.o clakf2.o clarge.o
claror.o clarot.o clatm1.o clarnd.o clatm2.o clatm3.o clatm5.o clatm6.o
slatm1.o slaran.o slarnd.o dlatms.o dlatme.o dlatmr.o dlagge.o dlagsy.o
dlakf2.o dlarge.o dlaror.o dlarot.o dlatm2.o dlatm3.o dlatm5.o dlatm6.o
zlatms.o zlatme.o zlatmr.o zlagge.o zlaghe.o zlagsy.o zlakf2.o zlarge.o
zlaror.o zlarot.o zlatm1.o zlarnd.o zlatm2.o zlatm3.o zlatm5.o zlatm6.o
dlatm1.o dlaran.o dlarnd.o
ranlib ../../tmglib_LINUX.a
make[1]: Leaving directory `/home/whaley/numerics/lapack-3.1.1/TESTING/MATGEN'
227.482u 20.093s 4:09.81 99.1%  0+0k 0+0io 12pf+0w
animal>
animal>ls
BLAS/    INSTALL/        make.inc          README    tmglib_LINUX.a
COPYING  lapack_LINUX.a  make.inc.example  SRC/
html/    Makefile        manpages/         TESTING/
\end{verbatim}
\end{footnotesize}

So, we have succesfully created the LAPACK library, and now we need to
install ATLAS and a complete LAPACK using it.
}

\subsection{Creating {\tt BLDdir} and installing ATLAS}
\label{sec-ex-install}
I'm ready to install ATLAS and LAPACK.  I just need to untar the
ATLAS tarfile, issue, create my {\tt BLDdir}, and issue the
previously selected flags to configure:
\vspace*{-0.1in}
\begin{footnotesize}
\begin{verbatim}
etl-opt8>bunzip2 -c ~/dload/atlas3.9.12.tar.bz2 | tar xfm -
etl-opt8>mv ATLAS ATLAS3.9.12.1   
etl-opt8>cd ATLAS3.9.12.1/
etl-opt8>mkdir obj64
etl-opt8>cd obj64/
etl-opt8>../configure -b 64 -D c -DPentiumCPS=2100 -Fa alg --shared \
  --prefix=/home/whaley/local/atlas \
  --with-netlib-lapack-tarfile=/home/whaley/dload/lapack.tgz   
...................................................
............<A WHOLE LOT OF OUTPUT>................
...................................................

etl-opt8>ls
ARCHS/       Makefile          xconfig*   xprobe_3dnow*      xprobe_OS*
atlcomp.txt  Make.inc          xctest*    xprobe_arch*       xprobe_pmake*
atlconf.txt  Make.top          xf2cint*   xprobe_asm*        xprobe_sse1*
bin/         src/              xf2cname*  xprobe_comp*       xprobe_sse2*
include/     tune/             xf2cstr*   xprobe_f2c*        xprobe_sse3*
interfaces/  xarchinfo_linux*  xf77test*  xprobe_gas_x8632*  xprobe_vec*
lib/         xarchinfo_x86*    xflibchk*  xprobe_gas_x8664*  xspew*

etl-opt8>make
.........................................................
............<A WHOLE WHOLE LOT OF OUTPUT>................
.........................................................
ATLAS install complete.  Examine
ATLAS/bin/<arch>/INSTALL_LOG/SUMMARY.LOG for details.
make[1]: Leaving directory `/home/whaley/TEST/ATLAS3.9.12.1/obj64'
make clean
make[1]: Entering directory `/home/whaley/TEST/ATLAS3.9.12.1/obj64'
rm -f *.o x* config?.out *core*
make[1]: Leaving directory `/home/whaley/TEST/ATLAS3.9.12.1/obj64'
1628.011u 153.212s 23:05.34 128.5%      0+0k 32+3325928io 0pf+0w
\end{verbatim}
\end{footnotesize}

OK, in a little over 20 minutes, we've got ATLAS and LAPACK built.  Now, we need
to see if it passes the sanity tests, which we do by:
\vspace*{-0.1in}
\begin{footnotesize}
\begin{verbatim}
etl-opt8>make check
........................................................
............<A WHOLE LOT OF COMPILATION>................
........................................................
DONE BUILDING TESTERS, RUNNING:
SCOPING FOR FAILURES IN BIN TESTS:
fgrep -e fault -e FAULT -e error -e ERROR -e fail -e FAIL \
                bin/sanity.out
8 cases: 8 passed, 0 skipped, 0 failed
4 cases: 4 passed, 0 skipped, 0 failed
8 cases: 8 passed, 0 skipped, 0 failed
4 cases: 4 passed, 0 skipped, 0 failed
8 cases: 8 passed, 0 skipped, 0 failed
4 cases: 4 passed, 0 skipped, 0 failed
8 cases: 8 passed, 0 skipped, 0 failed
4 cases: 4 passed, 0 skipped, 0 failed
DONE
SCOPING FOR FAILURES IN CBLAS TESTS:
fgrep -e fault -e FAULT -e error -e ERROR -e fail -e FAIL \
                interfaces/blas/C/testing/sanity.out | \
                fgrep -v PASSED
make[1]: [sanity_test] Error 1 (ignored)
DONE
SCOPING FOR FAILURES IN F77BLAS TESTS:
fgrep -e fault -e FAULT -e error -e ERROR -e fail -e FAIL \
                interfaces/blas/F77/testing/sanity.out | \
                fgrep -v PASSED
make[1]: [sanity_test] Error 1 (ignored)
DONE
make[1]: Leaving directory `/home/whaley/TEST/ATLAS3.9.12.1/obj64'
61.684u 6.485s 1:08.66 99.2%    0+0k 0+163768io 0pf+0w
\end{verbatim}
\end{footnotesize}

So, since we see no failures, we passed.  I get essentially the same
output when I check the parallel interfaces (my machine has eight processors)
via {\tt make ptcheck}.

Now, I am ready to make sure my libraries are getting the expected performance,
so I do:
\vspace*{-0.1in}
\begin{footnotesize}
{\samepage
\begin{verbatim}
etl-opt8>make time
........................................................
............<A WHOLE LOT OF COMPILATION>................
........................................................
                    single precision                  double precision
            ********************************   *******************************
                  real           complex           real           complex
            ---------------  ---------------  ---------------  ---------------
Benchmark   Refrenc Present  Refrenc Present  Refrenc Present  Refrenc Present
=========   ======= =======  ======= =======  ======= =======  ======= =======
  kSelMM      643.4   642.9    622.0   621.8    323.8   343.5    320.5   316.9
  kGenMM      191.4   192.1    161.8   174.1    178.3   164.3    172.9   172.4
  kMM_NT      140.0   138.5    127.4   129.3    137.4   136.1    126.4   131.8
  kMM_TN      165.2   165.3    159.8   157.0    163.0   161.6    158.0   155.2
  BIG_MM      604.1   617.0    601.8   599.8    311.3   332.3    309.2   292.1
   kMV_N       74.3    70.2    211.2   197.5     51.9    48.4    107.3    99.7
   kMV_T       82.2    79.8     97.2    95.3     46.4    43.9     77.6    73.3
    kGER       60.1    56.9    153.5   130.3     38.8    32.0     77.5    64.8
\end{verbatim}
}
\end{footnotesize}
We see that load and timer issues have made it so there is not an exact
match, but that neither install is worse overall, and so this install
looks good!  Now we are finally ready to install the libraries.  We can
do so, and then check what got installed by:
\begin{footnotesize}
\begin{verbatim}
etl-opt8>make install
...............................................
..............<A LOT OF OUTPUT>................
...............................................
etl-opt8>cd ~/local/atlas/
etl-opt8>ls
include/  lib/

etl-opt8>ls include/
atlas/  cblas.h  clapack.h

etl-opt8>ls include/atlas/
atlas_buildinfo.h      atlas_dr1kernels.h     atlas_strsmXover.h
atlas_cacheedge.h      atlas_dr1_L1.h         atlas_tcacheedge.h
atlas_cGetNB_gelqf.h   atlas_dr1_L2.h         atlas_trsmNB.h
atlas_cGetNB_geqlf.h   atlas_dsyr2.h          atlas_type.h
atlas_cGetNB_geqrf.h   atlas_dsyr.h           atlas_zdNKB.h
atlas_cGetNB_gerqf.h   atlas_dsyr_L1.h        atlas_zGetNB_gelqf.h
atlas_cmv.h            atlas_dsyr_L2.h        atlas_zGetNB_geqlf.h
atlas_cmvN.h           atlas_dsysinfo.h       atlas_zGetNB_geqrf.h
atlas_cmvS.h           atlas_dtGetNB_gelqf.h  atlas_zGetNB_gerqf.h
atlas_cmvT.h           atlas_dtGetNB_geqlf.h  atlas_zmv.h
atlas_cNCmm.h          atlas_dtGetNB_geqrf.h  atlas_zmvN.h
atlas_cr1.h            atlas_dtGetNB_gerqf.h  atlas_zmvS.h
atlas_cr1kernels.h     atlas_dtrsmXover.h     atlas_zmvT.h
atlas_cr1_L1.h         atlas_pthreads.h       atlas_zNCmm.h
atlas_cr1_L2.h         atlas_sGetNB_gelqf.h   atlas_zr1.h
atlas_csNKB.h          atlas_sGetNB_geqlf.h   atlas_zr1kernels.h
atlas_csyr2.h          atlas_sGetNB_geqrf.h   atlas_zr1_L1.h
atlas_csyr.h           atlas_sGetNB_gerqf.h   atlas_zr1_L2.h
atlas_csyr_L1.h        atlas_smv.h            atlas_zsyr2.h
atlas_csyr_L2.h        atlas_smvN.h           atlas_zsyr.h
atlas_csysinfo.h       atlas_smvS.h           atlas_zsyr_L1.h
atlas_ctGetNB_gelqf.h  atlas_smvT.h           atlas_zsyr_L2.h
atlas_ctGetNB_geqlf.h  atlas_sNCmm.h          atlas_zsysinfo.h
atlas_ctGetNB_geqrf.h  atlas_sr1.h            atlas_ztGetNB_gelqf.h
atlas_ctGetNB_gerqf.h  atlas_sr1kernels.h     atlas_ztGetNB_geqlf.h
atlas_ctrsmXover.h     atlas_sr1_L1.h         atlas_ztGetNB_geqrf.h
atlas_dGetNB_gelqf.h   atlas_sr1_L2.h         atlas_ztGetNB_gerqf.h
atlas_dGetNB_geqlf.h   atlas_ssyr2.h          atlas_ztrsmXover.h
atlas_dGetNB_geqrf.h   atlas_ssyr.h           cmm.h
atlas_dGetNB_gerqf.h   atlas_ssyr_L1.h        cXover.h
atlas_dmv.h            atlas_ssyr_L2.h        dmm.h
atlas_dmvN.h           atlas_ssysinfo.h       dXover.h
atlas_dmvS.h           atlas_stGetNB_gelqf.h  smm.h
atlas_dmvT.h           atlas_stGetNB_geqlf.h  sXover.h
atlas_dNCmm.h          atlas_stGetNB_geqrf.h  zmm.h
atlas_dr1.h            atlas_stGetNB_gerqf.h  zXover.h

etl-opt8>ls lib/
libatlas.a   libcblas.so    liblapack.a   libptcblas.so
libatlas.so  libf77blas.a   liblapack.so  libptf77blas.a
libcblas.a   libf77blas.so  libptcblas.a  libptf77blas.so
\end{verbatim}
\end{footnotesize}

The shared object support in ATLAS is still experimental, so we
can get some idea if our shared objects work by running an undocumented
tester.  To try a dynamically linked LU factorization, we:
\begin{footnotesize}
\begin{verbatim}
animal>cd ../bin
animal>make xdlutst_dyn
...............................................................
............<A WHOLE LOT OF UP-TO-DATE CHECKING>...............
...............................................................
make[1]: Leaving directory `/home/whaley/numerics/ATLAS3.7.38/animal64/bin'
gfortran -O -fPIC -m64 -o xdlutst_dyn dlutst.o \
   /home/whaley/numerics/ATLAS3.7.38/animal64/lib/libtstatlas.a \
   /home/whaley/numerics/ATLAS3.7.38/animal64/lib/liblapack.so \
   /home/whaley/numerics/ATLAS3.7.38/animal64/lib/libf77blas.so \
   /home/whaley/numerics/ATLAS3.7.38/animal64/lib/libcblas.so \
   /home/whaley/numerics/ATLAS3.7.38/animal64/lib/libatlas.so \
   -Wl,--rpath /home/whaley/numerics/ATLAS3.7.38/animal64/lib

animal>./xdlutst_dyn
NREPS  Major      M      N    lda  NPVTS      TIME     MFLOP     RESID
=====  =====  =====  =====  =====  =====  ========  ========  ========
    0  Col      100    100    100     95     0.001  1273.153 1.416e-02
    0  Col      200    200    200    194     0.002  2453.930 1.087e-02
    0  Col      300    300    300    295     0.007  2574.077 8.561e-03
    0  Col      400    400    400    394     0.017  2531.312 8.480e-03
    0  Col      500    500    500    490     0.031  2701.090 7.610e-03
    0  Col      600    600    600    594     0.051  2796.150 8.332e-03
    0  Col      700    700    700    693     0.081  2832.877 7.681e-03
    0  Col      800    800    800    793     0.116  2938.840 7.091e-03
    0  Col      900    900    900    893     0.161  3014.142 6.856e-03
    0  Col     1000   1000   1000    995     0.221  3019.330 7.097e-03

10 cases ran, 10 cases passed
\end{verbatim}
\end{footnotesize}
So, we appear to be good, and the install is complete!  Now we point our
users to the installed libs, and wait for the error reports to roll in.

\section{Special Instructions for some platforms}
\subsection{Special Instructions for 64 bit Windows (eg., Windows 7) users}
\label{sec-windows64}

If you have a 64-bit capable Windows like Windows~7, and it is possible
for your application to use the 64-bit libraries, it is highly recommended
to build the 64-bit versions rather than the 32-bit legacy versions.  The
32-bit library only has access to half as many registers, which will result
in strongly reduced performance.

ATLAS requires cygwin in order to install under Windows.  Cygwin
provides a Unix-style shell environment (including standard utilities such as
{\tt gcc} and {\tt make}) for Windows.  
Cygwin is free, and can be downloaded from {\tt www.cygwin.com}.  
Setup is usually as easy as running a install script selecting a mirror
site, and selecting the right packages.  If you have found you have
missed a package, just rerun the install script to add it.  The packages
that I install are pretty much everything that mentions gcc and gfortran.
You can find these packages by entering ``gcc'' in the search box.  Be sure
to select the gcc and gfortran packages that start with ``ming64''.
Once you finish the install and open up a cygwin window, you should
find the 64-bit compilers in \texttt{/usr/bin/}.  I used the gcc 4.x
series one.  In my install the full path was:
\begin{verbatim}
   /usr/bin/x86_64-w64-ming32-gcc-4.5.3.exe
   /usr/bin/x86_64-w64-ming32-gfortran.exe
\end{verbatim}

If you don't have files like this (though maybe different version numbers
on gcc, obviously), run cygwin setup again and select the right packages.

Once these packages exist in \texttt{/usr/bin} (any version number is
OK on the gcc executable, as is no number at all), then the next step
is to change your path so that cygwin can find the 64-bit MinGW dlls,
which you can do with:
\begin{verbatim}
   export PATH=/usr/i686-w64-mingw32/sys-root/mingw/bin:$PATH
\end{verbatim}

Now you should be able to config and install as normal.

\paragraph{Getting sshd enabled for remote login}
I also install Openssh, so that I can ssh into the box remotely from my
GNU/Linux workstation.  I found these directions helpful:
\begin{verbatim}
   http://www.acooke.org/cute/CygwinSSHS0.html
\end{verbatim}
To supplement those given in \texttt{/usr/share/doc/Cygwin/openssh.README}.
In the prompt for running {\tt /usr/bin/ssh-host-config}, I answered
No to ``privel sep'', yes to ``service'', left daemon blank and pretty
much took the defaults from there.  To test that the service is up,
just look up your IP address (``\texttt{IPv4 Address}'' from 
``\texttt{ipconfig}'' ran in cygwin window), and try to ssh to that IP
address from the machine itself.  If that works, {\tt sshd} is working.
If you can't login from a remote machine to it, then the most common
problem is that the Windows firewall is blocking it.  I had to change
to change my windows firewall settings under control panel, under advanced
settings, I added a rule to accept TCP port 22.

\subsection{Special Instructions for Windows users}
\label{sec-windows}

ATLAS presently requires cygwin in order to install under Windows.  Cygwin
provides a Unix-style shell environment (including standard utilities such as
{\tt gcc} and {\tt make}) for Windows.
Cygwin is free, and can be downloaded from {\tt www.cygwin.com}.  
We presently do
not support Interix (AKA Windows Services for Unix, etc.) as provided
by Microsoft, but a user has submitted code to help with this, and so
we hope to add support in the future.  We have had requests
to support MinGW (\verb|http://www.mingw.org/|), but no one has submitted
suggested code to help, and I have never successfully figured out how to
install and use it, so this is probably not coming soon unless something
changes.

Once cygwin is installed, you are ready to install ATLAS.  If you want
to call ATLAS from code using {\tt gcc} and {\tt gfortran}, then you
can just install as usual.  

If you want to call ATLAS from code compiled by native compilers such
as the Intel or Microsoft compilers, you must set up some environment
variables so that these compilers can be called from cygwin's shell.
Details on how do do this are available in the ATLAS errata file:
\vspace*{-0.1in}
\begin{verbatim}
   http://math-atlas.sourceforge.net/errata.html#WinComp
\end{verbatim}

If you want multithreaded (eg., shared-memory parallel) ATLAS libraries,
you must use {\tt gcc} to compile the main library, and if you use a native
compiler for interface compilation, manually link to the cygwin library.
This is because
ATLAS uses the POSIX threading standard, which of course Microsoft
does not support, and so you need the cygwin emulation layer to use
a decade-old standard.

\subsection{Special instructions for AIX}
Under AIX, it is critical that you define an envirnment variable
indicating whether you are building 64 or 32 bit libraries, and
this definition must match what you pass to {\tt configure} via the
{\tt -b} flag.  You need to define the environment variable
{\tt OBJECT\_MODE} to either 64 or 32, depending on which of
these you pass to {\tt configure} using the {\tt -b} flag.  So, if you
are building 64-bit libraries and you use a {\tt bash} derivative
shell, you would issue {\tt export OBJECT\_MODE=64} before starting
the ATLAS configure step.  On the other hand, if you use a {\tt csh}
derivative shell and want to build 32 but libaries, you would need
to issue {\tt setenv OBJECT\_MODE 32} before the build step.

\subsection{Special instructions for SunOS}
Solaris has its own version of the Unix utilities, which differ sharply
from the more common gnu tools.  In particular, SunOS offers two fgreps,
one of which works correctly for ATLAS's {\tt make check} step, and
one of which does not.  On my SunOS machine, I had to make sure
{\tt /usr/xpg4/bin} was in my path before {\tt /bin} in order
to get an {\tt fgrep} that can take multiple expression arguments
(as {\tt make check} requires).

Also, if gcc isn't compiled with with the correct gnu utilities, ATLAS
may fail to autodetect the assembly dialect of your machine.  This
will cause the build to fail since it can't assemble the UltraSPARC
assembly kernels, and you can see if it happened by examining your
{\tt Make.inc}'s ARCHDEF macro.  If this macro does not include
the definition {\tt -DATL\_GAS\_SPARC}, then this has happened to you.
On some systems, you can get the install to work by adding the flag
{\tt -s 3} to your {\tt configure} invocation.  If this still doesn't 
fix the problem, you'll need to get a better gcc install.  Note that
this error causes linking to assembled files to die with messages like:
\vspace{-0.05in}
\begin{verbatim}
ld: fatal: relocation error: R_SPARC_32: file /var/tmp//ccccPppx.o: 
    symbol <unknown>: offset 0xff061776 is non-aligned
\end{verbatim}

\section{Troubleshooting}
The first thing you need to do is scope the errata file to see if your
problem is already covered:
\vspace*{-0.1in}
\begin{verbatim}
   http://math-atlas.sourceforge.net/errata.html
\end{verbatim}

Probably the most common error is when ATLAS dies because its timings are
varying widely.  This can often be fixed with a simple restart, as described:
\vspace*{-0.1in}
\begin{verbatim}
   http://math-atlas.sourceforge.net/errata.html#tol
\end{verbatim}

If you are unable to find anything relevant in the errata file, you can
submit a support request to the ATLAS support tracker ({\bf not} the
bug tracker, which is for developer-confirmed bugs only):
\vspace*{-0.1in}
\begin{verbatim}
   https://sourceforge.net/tracker/?atid=379483&group_id=23725&func=browse
\end{verbatim}

When you create the support request, be sure to attach the error report.
It should appear as \verb|BLDdir/error_<arch>.tgz|.  If this file doesn't
exist, you can create it by typing \verb|make error_report| in your
{\tt BLDdir}.  More details on submitting support requests can be found
in the ATLAS FAQ at:
\vspace*{-0.1in}
\begin{verbatim}
   http://math-atlas.sourceforge.net/faq.html#help
\end{verbatim}



\bibliographystyle{plain}
\bibliography{atlas}
\newpage
\appendix

\section{Post-install Timing and Benchmarking}

This appendix describes how to use ATLAS's benchmarking tools, which are
built and run in {\tt OBJdir/results}.

\subsection{Setting up ploticus}
You will need to get ploticus installed.  On ubuntu, all you need is:
\begin{verbatim}
   sudo apt-get install ploticus
\end{verbatim}

For other OSes, download the software from the ploticus homepage:
\begin{verbatim}
   http://ploticus.sourceforge.net/doc/download.html
\end{verbatim}

Under ubuntu, the executable is called `ploticus'.  If your install
is called something else (eg., under Fedora it is called `pl'), then
you'll need to edit your {\tt BLDdir/results/Makefile} and change the
{\tt PLOT} macro to the correct name and path.  You may also need
to set where to find the ploticus prefab files.  For instance, under
Fedora I had to:
\begin{verbatim}
    export PLOTICUS_PREFABS=/usr/share/ploticus
\end{verbatim}

\subsection{Building the existing charts}
\label{sec-excharts}
This will continue to evolve, but as of right now, from your
{\tt BLDdir/results} directory, you can issue:
\begin{verbatim}
   make charts/<pre><area><usage><against>_<size>[_pt].eps
\end{verbatim}
Where the choices are:
\begin{enumerate}
\item \verb+pre+: precision/type prefix, choose `s', `d', `c', or `z'.
\item \verb+area+: choose `la' for lapack or `mm' for matmul
\item \verb+usage+: for lapack, choose `qr', `lu' or `lt' (Cholesky) \\
                    for GEMM, choose `sq' (square) or `rk' (rank-K update)
\item \verb+against+: choose `sys' or `at2' (compare two ATLAS installs)
\item \verb+size+: varies problem sizes charted, choose: `tiny', `med', `lrg'
\item \verb+_pt+: do threaded version rather than serial
\end{enumerate}

You can also get summary information that displays square GEMM and all
factorization performance on one chart with the commands:
\begin{verbatim}
   make charts/<pre>mmla<lib>[_pt].eps    # charts everything in raw MFLOP
   make charts/<pre>pcmm<lib>[_pt].eps    # charts everything as % of GEMM
\end{verbatim}
Where \verb+<lib>+ is one of `atl', `sys', `at2'.

To compare differing ATLAS installs, edit your {\tt BLDdir/results/Makefile},
and set {\tt AT2dir} to point to the other ATLAS install's lib directory.

{\samepage
To compare against a system LAPACK/BLAS, fill in the following macros in 
{\tt BLDdir/Make.inc} to point to the system libraries, rather than
the default F77BLAS:
\begin{verbatim}
   SBLASlib = $(FBLASlib)  # should be serial sysblas
   BLASlib = $(FBLASlib)   # should be parallel sysblas
   SLAPACKlib =   # set to parallel system lapack
   SSLAPACKlib =  # set to serial system lapack
\end{verbatim}
}

{\samepage
For instance, here's how to set them to use ACML:
\begin{verbatim}
   SBLASlib = /opt/acml4.4.0/gfortran64/lib/libacml.a
   BLASlib = /opt/acml4.4.0/gfortran64_mp/lib/libacml_mp.a -fopenmp
   SLAPACKlib = /opt/acml4.4.0/gfortran64_mp/lib/libacml_mp.a
   SSLAPACKlib =  $(SBLASlib)
\end{verbatim}
}

After the above setup, I can compare ATLAS's medium-sized Cholesky performance
to that of ACML by issuing:
\begin{verbatim}
   make charts/dlaltsys_med_pt.eps
   gv   charts/dlaltsys_med_pt.eps &
\end{verbatim}

\subsection{A guide to the tools (to build your own)}
I have written a set of generic tools for manipulating the output of
ATLAS's timers, and you can use and extend these tools if you want to
autotime fancier/different things.  All tools give usage information
if you pass {\tt --help} on the commandline.  All tools default to
taking input from {\tt stdin} and output to {\tt stdout}, so you can
pipe them into each other.  Each tool does a very simple thing, and
the idea is you build a pipe of them to do useful work.

To make building your own tools easy, examine 
{\tt SRCdir/include/atlas\_tvec.h}
which contains a host of prewritten routines and data structures to 
make tool building easy.

All the tools I have written allow you to choose to keep only certain
vectors of data (corresponding to columns of output in the timer output).
To give an example, say we ran the following line:
\begin{verbatim}
   c2d>./xdmmtst_atl -F 120 -N 10 100 10 -T 0 -# 3 > timer.out
\end{verbatim}

This will use {\tt gemmtst.c} to time all square problems between 10 and
100 in steps of 10, without doing any testing, forcing at least 120MFLOPS
of computation for accuracy, with three repititionsa.

Here's the tools I have written so far:
\begin{description}
\item [{\tt xatl2tvec}]: Reads in the output of a timer file, and produces
   a standard timing vectors file that can be read by routines provided
   in {\tt atlas\_tvec.h} and all downstream tools.  Example usage:
\begin{verbatim}
   c2d>./xatl2tvec -# 3
\end{verbatim}
\item [{\tt xreducetvec}]: take tvec file with repitition timings, and
      reduce them to single timings and adding simple statistics
      like min, max, and average.
\item [{\tt xcattvecs}]: take multiple vector files and combine them into
      one file for later comparison.  Renames vectors as necessary by
      adding \verb+_#+ to repeated names coming from later files.  Can
      specify for some vectors to get this statistical treatment, and
      other vectors to just use the first one found.
\item [{\tt xtvec2plp}]: Take a standard tvec file and produce a standard
      {\tt ploticus} prefab data file from it.
\item [{\tt xmergetvecs}]: Take two standar tvecs that contain separate runs
      of the same data with non-overlapping data, and combine them into
      one vector.  Eg., you do one run with $N=100,200, 300$ and a second
      with $N=1000 5000 8000$.  This routine will allow you to combine 
      these $N$ ranges into one for charting all results in one graph.
      This can be done repeatedly to merge any number of runs together.
\end{description}

To see how these tools can be used together, you can trace the dependence
chain of any of the charts that are autobuilt, as explained in
\S\ref{sec-excharts}.

\end{document}
