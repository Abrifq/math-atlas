\documentclass[11pt]{article}

%\usepackage{psfig}
\usepackage{epsfig}

\newcommand{\Wskip}[1]{ }
\newcommand{\Wceil}[1]{\lceil #1 \rceil}
\newcommand{\Wfloor}[1]{\lfloor #1 \rfloor}

\newenvironment{routdef}[1]
{
   \begin{list}{BLANK}
   {
      \setlength{\parsep}{0in}
      \setlength{\itemsep}{.01in}
      \setlength{\partopsep}{0in}
      \setlength{\topsep}{0.1in}
      \setlength{\labelsep}{0in}
      \setlength{\labelwidth}{#1in}
      \setlength{\leftmargin}{#1in}
   }
} {\end{list}}
\newcommand{\rditem}[2]{\item[#1\hfill(~]#2 )}
\newcommand{\kernk}[1]{{\tt #1\_k}}



\textwidth=6in
\textheight=8.7in
\hoffset = -.6in
\voffset = -.6in

\begin{document}

\begin{titlepage}
\title{User contribution to ATLAS}
\vspace{.4in}
\author
{
 R. Clint Whaley
}
\end{titlepage}
\maketitle

\begin{abstract}
This paper describes the method by which users can speed up ATLAS for
themselves, as well as contribute any such speedup to the ATLAS project.
It's written to get you started, in a highly informal (read sloppy)
fashion.  There's a lot of material that optimally should be covered
in detail, which is only hinted at in this document.  Since no real
attempt has been made to make the document sheerly backward referential,
it is recommended that the user at least skim the entire section before
attempting to understand and/or apply information from a given
subsection.
\end{abstract}

\newpage
\tableofcontents
%\listoftables
%\listoffigures

\newpage

\section{Introduction}

This paper describes the method by which users can speed up ATLAS
(Automatically Tuned Linear Algebra Software) for themselves, as well
as contribute any such speedup to the ATLAS project.

ATLAS is an implementation of a new style of high performance software
production/maintenance called Automated Empirical Optimization of Software
(AEOS).  In an AEOS-enabled library, many different ways of performing
a given kernel operation are supplied, and timers are used to empirically
determine which implementation is best for a given architectural platform.
ATLAS uses two techniques for supplying different implementations of kernel
operations: {\em multiple implementation} and {\em code generation}.

In code generation, a highly-parameterized program is written that can
generate many different kernel implementations.  The matrix multiply
code generator is an example of this.  The second method is multiple
implementation, and this, as its name suggests, is simply supplying
various hand-written implementations of the same kernel.

ATLAS provides a standard way for users to help with multiple
implementation.  ATLAS is designed such that several kernel routines
supply performance for the entire library.  The entire Level 3 BLAS
may be speeded up by improving the one simple kernel, which we will
refer to as {\it gemmK}, to distinguish it from a full GEMM routine.
The Level~2 routines make similarly be sped up by providing GER and
GEMV kernels (there are several of these, as discussed later).
ATLAS has standard
timers which can call user-programmed versions of these kernels, and
automatically use them throughout the library when they are superior
to the ATLAS-produced versions.

One thing to consider when getting started is to take the best ATLAS kernel
found, and, for instance, add some prefetch instructions, and see if you
can get noticeable improvements.

\subsection{Contributing your improvement to ATLAS}

If you produce a substantially better kernel than ATLAS presently has, we hope
that you will contribute it to the ATLAS project.  Code submissions and 
discussions of this type should take place on the 
{\tt math-atlas-devel@lists.sourceforge.net}
mailing list.  Anyone can sign up for this list, and we hope that interested
parties will correspond using it.  In this way someone interested in supplying
a particular enhancement to ATLAS can find if someone else is already working
on it, find interested collaborators, get coding help from a wider pool than
with the standard atlas mailing list, etc.

\subsubsection{Signing up for the ATLAS mailing list}
We have set up a mailing list that we hope contributers and interested
parties will subscribe to.  It's e-mail address is 
{\tt math-atlas-devel@lists.sourceforge.net}.  For details on signing up, see
{\verb+http://math-atlas.sourceforge.net/faq.html#lists+}.
\Wskip
{
We have set up a mailing list that we hope contributers and interested
parties will subscribe to, {\tt atlas-comm@cs.utk.edu}.  The regular
ATLAS mailing list, {\tt atlas@cs.utk.edu}, is a fixed mailing list, 
whose recipients are limited to the UTK contingent of the ATLAS project.
On the other hand, {\tt atlas-comm@cs.utk.edu}, is a dynamic list, that
users can subscribe and unsubscribe to at will.

To subscribe to this mailing list, send mail to {\tt majordomo@cs.utk.edu},
and include the line {\tt subscribe atlas-comm} as the only body to the 
message.  Unsubscribing is accomplished by sending {\tt unsubscribe atlas-comm}
instead.
}

\subsubsection{License issues}

ATLAS uses a BSD-style license for its distribution, and thus any code
that you wish to contribute must have a compatible license.  This effectively
rules out GPL and even LGPL licenses, since if the ATLAS group redistributed
any routines with these licenses, the conditions of their licenses would
effectively change the entire ATLAS library to GPL or LGPL.
%{\it {www.opensource.org} seems to indicate this is not the case.  If
%GPL does not destroy the BSD-ness of the rest of the library, ATLAS
%will explore excepting GPL contributions directly}.

Note that you, the author, retain the copyright, and thus you can relicense
your code in any way you wish.  So, anything released as part of the ATLAS
tarfile will need a BSD-compatible license, but you could then issue the
code yourself under a different license.  Alternatively, you can provide
your codes as ``patches'' to standard ATLAS, and thus avoid the license
issue altogether, at the cost of not being part of the standard tarfile.
Even if you decide not to license your contribution in a way that the
ATLAS group can redistribute, submitting it to the mailing list is still
recommended.  Other people can see your work, and we can point to it if
we think it is cool enough.

\subsubsection{Due Credit}

We have set in place some infrastructure in order to give contributers
credit for their work.  Firstly, if your code is included in the standard
tarfile, your contribution will be noted in {\tt ATLAS/doc/AtlasCredits.txt}.
As the author, you retain the copyright, and thus your name is in the
contributed code.
We have also modified the installation logger to print the author name
for user-provided routines (the exception is the Level 1 kernels, which are
simply too numerous for logging).  We will make a good faith effort to give
due credit for work, but in general, we can't guarantee anything.

As an aside, if fame and fortune are your major motivations, you may 
want to consider contributing to something else anyway.  As the founder
of the ATLAS project, I'm still waiting for my first interview with CNN,
NBC, etc., or indeed, anyone other then my relatives (and all those
interviews go like, "You're in computers, can't you get those wordperfect
guys to make it easier to use?").

\subsubsection{Inclusion in the ATLAS tarfile}

If you submit an improvement to {\tt math-atlas-devel@lists.sourceforge.net},
we hope that
other ATLAS users will try it out, and give comments.  Ultimately, 
someone in the ATLAS group (currently, that would be me) will decide 
whether to include it in the standard tarfile or not.  Such a decision
will be based on how hard it is to get to work with our distribution
(this pain should be minimal if the instructions in this note are followed),
how much of an improvement it is, etc.  Obviously, there are speed improvements
that some users may utilize that the ATLAS group can't indulge in (examples
include speedups that possibly lose accuracy, such as, for instance, using
Strassen's algorithm for matrix multiply).  If there appears to be a great
demand for such lossy optimizations, we may consider ways of allowing
users to select if they want to risk them.

Note that even if we are ungrateful morons who do not release your
submission, you, and anyone who wants to use your kernel routines,
can use the ATLAS infrastructure to get a complete, fast BLAS implementation.

\subsection{Directory terminology}
Using ATLAS's configure, you can build the library in any directory.  We
will call the path to your top-level directory where you choose to build
ATLAS {\tt BLDdir}.  The ATLAS tarfile produces the source directory,
which we will indicate by either {\tt SRCdir} or simply {\tt ATLAS/....}.

\subsection{Coding conventions for contributed code}
The following rules are mandatory:
\begin{itemize}
\item All externally addressable symbols used in the code (eg, routine names,
global variables, etc) will be prefaced by the prefix {\tt ATLU\_}.
\item All user-contributed include files will be found in
{\tt ATLAS/include/contrib}.
\end{itemize}

The ATLAS team encourages:
\begin{itemize}
\item Using the standard BLAS names for operands (i.e., for GEMM, using
the names {\tt A}, {\tt B}, and {\tt C} for the input matrices, for
GEMV, using the names {\tt A}, {\tt X}, and {\tt Y}, etc.).
\end{itemize}


\subsection{A note about ATLAS kernels}
All of ATLAS's user-suppliable kernels are used to speed up a wide range
of codes (i.e., GEMM speeds up all level 3 BLAS, etc), which means it is
possible to write a good GEMM, for instance, that is still not a good
GEMM {\em kernel}.  The unmodified testers and timers described in this
note time these kernels in their most-used states, so if you develop a
kernel using these techniques, everything will likely be OK.  However,
if you first write a full-blown GEMV, for instance, and then attempt
to adapt it, there is more opportunity for mismatch.  At the end of
each kernel section I give a few kernel notes to give you an idea of
how ATLAS uses the kernel.

\section{Speeding up the Level 3 BLAS using block-major GEMM}

The performance kernel for the entire Level 3 BLAS is matrix multiply.
Matrix multiply is written in terms of a lower-level building block that
we call {\it gemmK}.  {\it gemmK} is a special matrix multiply where
the input dimensions are fixed at $M = N = K = N_B$, where the blocking
factor $N_B$ is chosen in order to maximize L1 cache reuse, for a loose
enough definition of L1 cache (typically, we use it to mean the first 
level of cache accessible by the FPU, which may be the L2 cache on
some systems).

ATLAS actually has two different classes of GEMM kernels: one for copied
matrices ({\it gemmK}), and one that operates directly on the user's matrices
without a copy.  For matrices of
sufficient size, ATLAS copies the input matrix into {\em block-major} storage.
In block-major storage, the $N_B \times N_B$ blocks operated on by the
{\it gemmK} are actually contiguous.  This optimization prevents unnecessary
cache misses, cache conflicts, and TLB problems.  However, for sufficiently
small matrices, the cost of this data copy is prohibitively expensive,
and thus ATLAS has kernels that operate on non-copied data.  However,
without the copy to simplify the process, there are multiple non-copy
kernels (differing kernels for differing transpose settings, for instance).
Since the non-copy kernels are typically only used for very small problems,
and they are much more complex, ATLAS presently accepts contributed code
only for the copy matmul kernel.  For most problems, well over 98\% of ATLAS
time is spent in the copy matmul kernel, so this should not be much of 
a problem.

\subsection{Building the General Matrix Multiply From {\it gemmK}}
\label{sec-buildMM}

This section describes the code necessary to build the BLAS's general
matrix-matrix multiply using an L1 cache-contained
matmul (hereafter referred to as {\it gemmK}).

For our present discussion, it is enough to know
that ATLAS has at its disposal highly optimized routines for doing matrix
multiplies whose dimensions are chosen such that cache blocking is not
required (i.e., the hand-written code discussed in this section deals
with cache blocking; the generated/user supplied kernel assumes things fit
into cache).

When the user calls GEMM, ATLAS must decide whether the problem
is large enough to tolerate copying the input matrices $A$ and $B$.
If the matrices are large enough to support this $O(N^2)$ overhead, ATLAS will
copy $A$ and $B$
into block-major format.  ATLAS's block-major format breaks up the input
matrices into contiguous blocks of a fixed size $N_B$, where $N_B$ is chosen
in order to maximize L1 cache
reuse.  Once in block-major format, the blocks are contiguous, which
eliminates TLB problems, minimizes cache thrashing and maximizes cache line
use.  It also allows ATLAS to apply alpha (if alpha is not already one) to the
smaller of $A$ or $B$, thus minimizing this cost as well.  Finally, the
package can use the copy to transform the problem to a particular transpose
setting, which for load and indexing optimization, is set so
A is copied to transposed form, and B is in normal (non-transposed) form.
This means our L1-cache contained code is of the form
$C \leftarrow A^T B$, $C \leftarrow A^T B + C$,
and $C \leftarrow A^T B + \beta C$,
{\em where all dimensions, including the non-contiguous stride, are known
to be $N_B$}.  Knowing all of the dimensions of the loops allows for
arbitrary unrollings (i.e., if the instruction cache could support it, ATLAS
could unroll all loops completely, so that the L1 cache-contained multiply
had no loops at all).  Further, when the code generator knows leading dimension
of the matrices (i.e., the row stride), all indexing can be done up front,
without the need for expensive integer or pointer computations.

If the matrices are too small, the $O(N^2)$ data copy cost can actually
dominate the algorithm cost, even though the computation cost is $O(N^3)$.
For these matrices, ATLAS will call an gemm kernel which operates on non-copied
matrices (i.e. directly on the user's operands).  The non-copy matmul kernels
will generally not be as efficient as the even the generated copy {\it gemmK};
at this problem
size the main drawback is the additional pointer arithmetic required in order
to support the user-supplied leading dimension and its affect on the cost of
the memory load (which varies according to transpose settings, as well as
architectural features).

The choice of when a copy is dictated and when it is prohibitively expensive
is an AEOS parameter; it turns out that this crossover point depends strongly
both on the particular architecture, and the shape of the operands
(matrix shape effectively sets limits on which matrix dimensions can
enjoy cache reuse).  To handle this problem, ATLAS simply compares the
speed of the copy and non-copy matmul kernels for variously shaped matrices,
varying the problem size until the copy code provides a speedup (on some
platforms, and with some shapes, this point is never reached).  These
crossover points are determined at install time, and then used to make
this decision at runtime.  Because it is the dominant case, this paper
describes only the copied matmul algorithm in detail.

Figure~\ref{fig-GemmPanel} shows the necessary steps for computing a
$N_B \times N_B$ section of $C$ using {\it gemmK}.

\begin{figure}[hbtp]
\begin{picture}(430,100)(-20,0)
\put(20,15){\framebox(20,20){$C_{3,2}$}}
\put(150,15){\framebox(20,20){$A_{3,1}$}}
\put(170,15){\framebox(20,20){$A_{3,2}$}}
\put(190,15){\framebox(10,20){}}
\put(-20,37){$M$}
\put(50,80){$N$}
\put(0,0){\framebox(100,75){\bf C}}
\put(120,37){$\leftarrow$}
\put(135,37){$M$}
\put(173,80){$K$}
\put(150,0){\framebox(50,75){\bf A}}
\put(280,65){$N$}
\put(345,37){$K$}
\put(210,36){$\times$}
\put(230,10){\framebox(100,50){\bf B}}
\put(250,40){\framebox(20,20){$B_{1,2}$}}
\put(250,20){\framebox(20,20){$B_{2,2}$}}
\put(250,10){\framebox(20,10){$B_{3,2}$}}
\end{picture}
\caption{One step of matrix-matrix multiply\label{fig-GemmPanel}}
\end{figure}

More formally, the following actions are performed in order to
calculate the $N_B \times N_B$ block $C_{i,j}$, where $i$ and $j$ are in
the range $0 \leq i < \Wceil{M/N_B}$, $0 \leq j < \Wceil{N/N_B}$:

\begin{enumerate}
 \item Call {\it gemmK} of the correct form based on user-defined $\beta$
       (eg. if $\beta == 0$, use $C \leftarrow A B$) to multiply block $0$
       of the row panel $i$ of $A$ with block $0$ of the column panel $j$
       of $B$.
 \item Call {\it gemmK} of form $C \leftarrow A B + C$
       to multiply block $k$ of the row panel $i$ of
       $A$ with block $k$ of the column panel $j$ of $B$,
       $\forall k, 1 \leq k < \Wceil{K/N_B}$.
\end{enumerate}

As this example demonstrates, if a given dimension is not a multiple of
the L1 blocking factor $N_B$, partial blocks results.  ATLAS has special
routines that handle cases where one or more dimension is less than $N_B$;
these routines are referred to as {\em cleanup} codes.

\subsection{The Main GEMM Kernel, {\it gemmK}}
So, there are actually three {\it gemmK} kernels (corresponding to different
$\beta$ values), and perform the operations:
$C \leftarrow A^T B$,
$C \leftarrow A^T B + C$,
$C \leftarrow A^T B + \beta C$.  All input arrays ($A, B, C$) are
column-major (they are still used as performance kernels for row-major
BLAS as well, so don't worry).  Additionally, $A^T$ and $B$ are in block-major
format, such that $lda = ldb = M = N = K = N_B$.

\subsubsection{{\it gemmK}  macro definitions}
In order to make writing a {\it gemmK} easier, ATLAS defines several 
cpp macros for programmer use.  Examples in subsequent sections should 
illustrate the use of these macros, so we merely define them here.

First, ATLAS defines the macro {\tt ATL\_USERMM}
to the appropriate ATLAS internal kernel name.  Second, it defines one
of  {\tt SREAL}, {\tt DREAL}, {\tt SCPLX}, {\tt DCPLX}, according to
the data type being compiled (single precision real, double precision
real, single precision complex, double precision complex, respectively).

Similarly, ATLAS defines a macro indicating the $\beta$ case being
compiled, {\tt BETA1} ($\beta$ should be assumed to be 1.0), {\tt BETA0}
($\beta$ should be assumed to be 0.0), and {\tt BETAX}
($\beta$ neither zero or one, and should be handled as an input parameter).

Finally, the fixed blocking factors for each dimension are defined
{\tt MB}, {\tt NB}, {\tt KB}.  Note that for our {\it gemmK},
{\tt MB} = {\tt NB} = {\tt KB} = $N_B$; they are separated out for
support of the cleanup codes, where they can be different.  ATLAS also
defines the macros {\tt MB2}, {\tt NB2}, {\tt KB2}, which are simply
two times the appropriate blocking factor. 

\subsubsection{{\it gemmK} API}
The {\it gemmK} API may be summarized as:

\vspace*{.1in}

\noindent
\verb+#if defined(SREAL) || defined(SCPLX)+
\begin{routdef}{1.8}
\rditem{~~~void ATL\_USERMM}
{const int M, const int N, const int K, const float alpha, 
 const float *A, const int lda, const float *B, const int ldb, 
 const float beta, float *C, const int ldc}
\end{routdef}

\noindent
\verb+#elif defined(DREAL) || defined(DCPLX)+
\begin{routdef}{1.8}
\rditem{~~~void ATL\_USERMM}
{const int M, const int N, const int K, const double alpha, 
 const double *A, const int lda, const double *B, const int ldb, 
 const double beta, double *C, const int ldc}
\end{routdef}

\noindent
\verb+#endif+

\subsubsection{{\it gemmK} description file}
In the install process, ATLAS first searches through the {\it gemmK}
implementations provided by the ATLAS matmul generator.  When the
best generated code is found, the user contributed codes are timed
to see if they can beat the generated code.  The {\tt gemmK} multiple
implementation search
script opens a description file for each precision 
({\tt scases.dsc}, {\tt dcases.dsc}, {\tt ccases.dsc}, {\tt zcases.dsc})
in the {\tt BLDdir/tune/blas/gemm/} directory,
to see what user-contributed codes are available.  This master index
file is actually generated based on several user-supplied files from
{\tt ATLAS/tune/blas/gemm/CASES} (see Section~\ref{sec-index} for the
names and definitions of these files).  The format for all these files
is the same, and is described in the following paragraphs.

The first line of each file is a comment line, and is ignored.  The next
line indicates the number of user-contributed codes to search, and
each subsequent line supplies information about a given user-supplied
{\it gemmK}.  The form of these lines is:\\
\verb+<ID> <flag> <mb> <nb> <kb> <muladd> <lat> <mu> <nu> <ku> <rout> "<author>"+

\verb+<rout>+ and \verb+<author>"+ are strings, and the rest of the
parameters are signed integers.  

The meaning of these parameters are:
\begin{itemize}
\item {\verb+ID+}: Strictly positive integer which uniquely identifies this
descriptor line.  ID must by unique only within a precision.
\item {\verb+<flag>+}: flag indicating special conditions.  See table below.

\item{\verb+<mb>+, \verb+<nb>+, \verb+<kb>+}: 
   Used to indicate restriction on the input parameter $M$ ($N$, $K$ resp.),
   and its associated blocking {\tt MB} ({\tt NB}, {\tt KB}, resp.).
   If the value is zero, the internal routine handles any $M$; i.e.
   the loop-limit is a runtime variable.  If the value is negative, then
   $M$ = {\tt MB} = -\verb+<mb>+ (i.e., the blocking factor cannot be
   varied using a macro).  If the value is positive, the blocking factor
   can be varied by setting the appropriate macro 
   ({\tt MB} {\tt NB}, {\tt KB}, resp.), but the blocking factor must be
   a multiple of the value.  Therefore, setting \verb+<mb>+ = 4, indicates
   that {\tt MB} must be a multiple of 4, while setting it to 1 indicates
   that {\tt MB} is an arbitrary compile-time constant.
\item{\verb+<muladd>+}:
   Set to zero if you are using separate multiply and add instructions, 1
   otherwise.  If you don't know the answer, put 1.
\item{\verb+<lat>+}:
   Set to the latency you use between floating point instructions.
   If you don't know the answer, put 1.
\item{\verb+<mu>+}:
   Unrolling you are using for the $M$ loop.
\item{\verb+<nu>+}:
   Unrolling you are using for the $N$ loop.
\item{\verb+<ku>+}:
   Unrolling you are using for the $K$ loop.
\item{\verb+<rout>+}:
   The filename of the user-contributed routine, relative to the path
   {\tt ATLAS/tune/blas/gemm/CASES}.  Maximum length 64 chars.
\item{\verb+<author>+}:
   The name of the author or authors, enclosed in quotes.  
   Maximum length 64 chars.
\end{itemize}

Table~\ref{tab-mmflag} summarizes the presently defined {\tt flag} values.\\
\begin{table}[h!]
\begin{center}
\begin{tabular}{||r|l||}\hline\hline
FLAG & MEANING \\\hline\hline
   0 & Normal \\\hline
   8 & Do not consider this kernel for cleanup\\\hline
  16 & Consider this kernel for cleanup {\em only}\\\hline 
  32 & lda and ldb are not restricted to {\tt KB} \\\hline 
  64 & {\tt mb} provides run-time constraint, not compile-time\\\hline 
 128 & {\tt nb} provides run-time constraint, not compile-time\\\hline 
 256 & {\tt kb} provides run-time constraint, not compile-time\\\hline 
 512 & This kernel needs $4 N_b \le cacheelts$\\\hline
  \hline
\end{tabular}
\end{center}
\caption{Matmul index routine flag variables\label{tab-mmflag}}
\end{table}

Here's an example:
\begin{verbatim}
<ID> <flag> <mb> <nb> <kb> <muladd> <lat> <mu> <nu> <ku> <rout> "<Contributer>"
3
 1 0 0 0 0 1 1 1 1 1 ATL_mm1x1x1.c "R. Clint Whaley"
 2 0 1 1 1 1 1 1 1 1 ATL_mm1x1x1b.c "R. Clint Whaley"
 3 0 1 1 8 1 1 1 1 4 ATL_mm2.c "R. Clint Whaley"
\end{verbatim}

So, we have 3 user-supplied routines, all written by me.  The first loops
over $M$, $N$, and $K$, but the following two routines loop over the cpp
macros {\tt MB}, {\tt NB}, {\tt KB}.  The third routine insists that
{\tt KB} be a multiple of 8.  The first two routines don't unroll
any of the loops, while the third unrolls the K loop to a depth of 4.
They all use a combined muladd style of programming, and don't worry
about latency.

\subsubsection{Index filenames}
\label{sec-index}
As previously mentioned, ATLAS builds a system and type dependent index
file from user-supplied files in {\tt ATLAS/tune/blas/gemm/CASES}.  This
is done so that the all routines do not need to be run on all machines
(i.e., no need to waste time trying to run SSE-enabled assembly routines
when on a Dec ev56).  Here is a list of description files presently queried
by ATLAS when building the full search index:
\begin{enumerate}
 \item {\bf [s,d,c,z]cases.0}:  Any user-contributed kernel which is system
       independent (i.e. doesn't require a particular compiler, etc)
       \begin{itemize}
       \item Convention is to choose IDs in range: $0 < ID < 100$.
       \end{itemize}
 \item {\bf [s,d,c,z]cases.flg}: Any user-contributed kernel requiring specific
       compiler and/or flags
       \begin{itemize}
       \item Convention is to choose IDs in range: $300 \leq ID < 400$.
       \end{itemize}
 \item {\bf [s,c]cases.3DN}: Kernels requiring 3DNow! to run.
       \begin{itemize}
       \item Convention is to choose IDs in range: $100 \leq ID < 200$.
       \end{itemize}
 \item {\bf [s,c]cases.SSE}: Kernels requiring SSE1 to run.
       \begin{itemize}
       \item Convention is to choose IDs in range: $200 \leq ID < 300$.
       \end{itemize}
\end{enumerate}

\subsection{Putting it together with some examples}

Let's say we decide to cover the basics, the classical 3 do loop
implementation of matmul would be:
\begin{verbatim}
void ATL_USERMM
   (const int M, const int N, const int K, const double alpha,
    const double *A, const int lda, const double *B, const int ldb,
    const double beta, double *C, const int ldc)
{
   int i, j, k;
   register double c00;

   for (j=0; j < N; j++)
   {
      for (i=0; i < M; i++)
      {
         #ifdef BETA0
            c00 = 0.0;
         #elif defined(BETA1)
            c00 = C[i+j*ldc];
         #else
            c00 = C[i+j*ldc] * beta;
         #endif
         for (k=0; k < K; k++) c00 += A[k+i*lda] * B[k+j*ldb];
         C[i+j*ldc] = c00;
      }
   }
}
\end{verbatim}

We then save this paragon of performance to 
{\tt ATLAS/tune/blas/gemm/CASES/ATL\_mm1x1x1.c}.  
From {\tt BLDdir/tune/blas/gemm/}, we can test that it
gets the right answer by:
\begin{verbatim}
   make mmutstcase pre=d nb=40 mmrout=CASES/ATL_mm1x1x1.c beta=0
   make mmutstcase pre=d nb=40 mmrout=CASES/ATL_mm1x1x1.c beta=1
   make mmutstcase pre=d nb=40 mmrout=CASES/ATL_mm1x1x1.c beta=7
\end{verbatim}

We pass four arguments to mmutstcase, a precision specifier
({\tt d} : double precision real; {\tt s} : single precision real; 
 {\tt z} : double precision complex; {\tt c} : single precision complex),
the size of the blocking parameter $N_B$, the beta value to test
(0, 1, and other), and finally, the filename to test.

If these messages pass the test, we can then see what kind of performance
we get by (this is the actual output on my 266Mhz PII):
\begin{verbatim}
make ummcase pre=d nb=40 mmrout=CASES/ATL_mm1x1x1.c beta=1
dNB=40, ldc=40, mu=4, nu=4, ku=1, lat=4: time=1.820000, mflop=53.731868
dNB=40, ldc=40, mu=4, nu=4, ku=1, lat=4: time=1.810000, mflop=54.028729
dNB=40, ldc=40, mu=4, nu=4, ku=1, lat=4: time=1.830000, mflop=53.438251
\end{verbatim}

This is the same timing repeated three times (this just tries to ensure
timings are repeatable), and the only output of real interest is the
MFLOP rate at the end.  The values the timer prints (mu, nu, ku, lat) are
all defaults because we didn't specify them; specifying them has no
effect when the timer is used in this way, so don't worry about them.

Now we can trivially improve the implementation by using the macro
constants in order to let the compiler unroll the loops:
\begin{verbatim}
void ATL_USERMM
   (const int M, const int N, const int K, const double alpha,
    const double *A, const int lda, const double *B, const int ldb,
    const double beta, double *C, const int ldc)
{
   int i, j, k;
   register double c00;

   for (j=0; j < NB; j++)
   {
      for (i=0; i < MB; i++)
      {
         #ifdef BETA0
            c00 = 0.0;
         #elif defined(BETA1)
            c00 = C[i+j*ldc];
         #else
            c00 = C[i+j*ldc] * beta;
         #endif
         for (k=0; k < KB; k++) c00 += A[k+i*KB] * B[k+j*KB];
         C[i+j*ldc] = c00;
      }
   }
}
\end{verbatim}

We save this to {\tt ATL\_mm1x1x1b.c}, and then time:
\begin{verbatim}
make ummcase pre=d nb=40 mmrout=CASES/ATL_mm1x1x1b.c beta=1
dNB=40, ldc=40, mu=4, nu=4, ku=1, lat=4: time=1.670000, mflop=58.558084
dNB=40, ldc=40, mu=4, nu=4, ku=1, lat=4: time=1.660000, mflop=58.910843
dNB=40, ldc=40, mu=4, nu=4, ku=1, lat=4: time=1.670000, mflop=58.558084
\end{verbatim}

OK, maybe a little explicit loop unrolling will make things work better:
\begin{verbatim}
void ATL_USERMM
   (const int M, const int N, const int K, const double alpha, const double *A, const int lda, const double *B, const int ldb, const double beta, double *C, const int ldc)
{
   int i, j, k;
   register double c00, c10, b0;
   const double *pA0, *pB=B;

#if ( (KB / 8)*8 != KB ) || (MB / 2)*2 != MB
   create syntax error!$@@&
#endif
   for (j=0; j < NB; j++, pB += KB)
   {
      pA0 = A;
      for (i=0; i < MB; i += 2, pA0 += KB2)
      {
         #ifdef BETA0
            c00 = c10 = 0.0;
         #elif defined(BETA1)
            c00 = C[i+j*ldc];
            c10 = C[i+1+j*ldc];
         #else
            c00 = beta*C[i+j*ldc];
            c10 = beta*C[i+1+j*ldc];
         #endif
         for (k=0; k < KB; k += 8)
         {
            b0 = pB[k];
            c00 += pA0[k] * b0;
            c10 += pA0[KB+k] * b0;
            b0 = pB[k+1];
            c00 += pA0[k+1] * b0;
            c10 += pA0[KB+k+1] * b0;
            b0 = pB[k+2];
            c00 += pA0[k+2] * b0;
            c10 += pA0[KB+k+2] * b0;
            b0   =  pB[k+3];
            c00 += pA0[k+3] * b0;
            c10 += pA0[KB+k+3] * b0;
            b0   =  pB[k+4];
            c00 += pA0[k+4] * b0;
            c10 += pA0[KB+k+4] * b0;
            b0   =  pB[k+5];
            c00 += pA0[k+5] * b0;
            c10 += pA0[KB+k+5] * b0;
            b0   =  pB[k+6];
            c00 += pA0[k+6] * b0;
            c10 += pA0[KB+k+6] * b0;
            b0   =  pB[k+7];
            c00 += pA0[k+7] * b0;
            c10 += pA0[KB+k+7] * b0;
         }
         C[i+j*ldc] = c00;
         C[i+1+j*ldc] = c10;
      }
   }
}
\end{verbatim}

And with this ode to beauty and elegance we get (after checking that it
still gets the right answer, of course):
\begin{verbatim}
make ummcase pre=d nb=40 mmrout=CASES/ATL_mm2x1x8.c beta=1
dNB=40, ldc=40, mu=4, nu=4, ku=1, lat=4: time=0.720000, mflop=135.822222
dNB=40, ldc=40, mu=4, nu=4, ku=1, lat=4: time=0.710000, mflop=137.735211
dNB=40, ldc=40, mu=4, nu=4, ku=1, lat=4: time=0.710000, mflop=137.735211
\end{verbatim}

Its interesting to see the effects of differing $\beta$ on the code:
\begin{verbatim}
make ummcase pre=d nb=40 mmrout=CASES/ATL_mm2x1x8a.c beta=0
dNB=40, ldc=40, mu=4, nu=4, ku=1, lat=4: time=0.700000, mflop=139.702857
dNB=40, ldc=40, mu=4, nu=4, ku=1, lat=4: time=0.700000, mflop=139.702857
dNB=40, ldc=40, mu=4, nu=4, ku=1, lat=4: time=0.700000, mflop=139.702857

make ummcase pre=d nb=40 mmrout=CASES/ATL_mm2x1x8a.c beta=7
dNB=40, ldc=40, mu=4, nu=4, ku=1, lat=4: time=0.720000, mflop=135.822222
dNB=40, ldc=40, mu=4, nu=4, ku=1, lat=4: time=0.730000, mflop=133.961644
dNB=40, ldc=40, mu=4, nu=4, ku=1, lat=4: time=0.720000, mflop=135.822222
\end{verbatim}

As well as differing blocking factors:
\begin{verbatim}
make ummcase pre=d mmrout=CASES/ATL_mm2x1x8a.c beta=1 nb=16
dNB=16, ldc=16, mu=4, nu=4, ku=1, lat=4: time=0.850000, mflop=115.112056
dNB=16, ldc=16, mu=4, nu=4, ku=1, lat=4: time=0.860000, mflop=113.773544
dNB=16, ldc=16, mu=4, nu=4, ku=1, lat=4: time=0.850000, mflop=115.112056

make ummcase pre=d mmrout=CASES/ATL_mm2x1x8a.c beta=1 nb=32
dNB=32, ldc=32, mu=4, nu=4, ku=1, lat=4: time=0.730000, mflop=134.034586
dNB=32, ldc=32, mu=4, nu=4, ku=1, lat=4: time=0.740000, mflop=132.223308
dNB=32, ldc=32, mu=4, nu=4, ku=1, lat=4: time=0.730000, mflop=134.034586

make ummcase pre=d mmrout=CASES/ATL_mm2x1x8a.c beta=1 nb=48
dNB=48, ldc=48, mu=4, nu=4, ku=1, lat=4: time=0.820000, mflop=119.223571
dNB=48, ldc=48, mu=4, nu=4, ku=1, lat=4: time=0.820000, mflop=119.223571
dNB=48, ldc=48, mu=4, nu=4, ku=1, lat=4: time=0.820000, mflop=119.223571
\end{verbatim}

If we wanted to have ATLAS try these crappy implementations during the
ATLAS search, we would have the following
{\tt ATLAS/tune/blas/gemm/CASES/dcases.dsc}:
\begin{verbatim}
<ID> <flag> <mb> <nb> <kb> <muladd> <lat> <mu> <nu> <ku> <rout> "<Contributer>"
3
1 0 0 0 0 1 1 1 1 1 ATL_mm1x1x1.c    "R. Clint Whaley"
2 0 1 1 1 1 1 1 1 1 ATL_mm1x1x1b.c   "R. Clint Whaley"
3 0 2 1 8 1 1 2 1 8 ATL_mm2x1x8a.c   "R. Clint Whaley"
\end{verbatim}

\subsection{More timing info}
So maybe you wonder how our big hand-tuned guy stacks up against the ATLAS
code generator?  When ATLAS completed its search on my PII, it stored its
best case in {\tt ATLAS/tune/blas/gemm/LINUX\_PII/res/dMMRES}:
\begin{verbatim}
speedy. cat res/dMMRES 
MULADD  LAT  NB  MU  NU  KU  FFTCH  IFTCH  NFTCH    MFLOP
     0    5  40   2   1  40      0      2      1   197.94
16
\end{verbatim}

We generate and time this case by:
\begin{verbatim}
make mmcase muladd=0 lat=5 nb=40 mu=2 nu=1 ku=40 beta=1
dNB=40, ldc=40, mu=2, nu=1, ku=40, lat=5: time=0.490000, mflop=199.575510
dNB=40, ldc=40, mu=2, nu=1, ku=40, lat=5: time=0.500000, mflop=195.584000
dNB=40, ldc=40, mu=2, nu=1, ku=40, lat=5: time=0.490000, mflop=199.575510
\end{verbatim}

We test that the generator isn't out of its mind by:
\begin{verbatim}
make mmtstcase muladd=0 lat=5 nb=40 mu=2 nu=1 ku=40 beta=1
\end{verbatim}

Note that when timing/testing the generator, varying the parameters such
as mu, nu, ku, beta, etc., generates different codes, and thus different
performance numbers:
\begin{verbatim}
make mmcase muladd=0 lat=4 nb=40 mu=2 nu=1 ku=4 beta=1
dNB=40, ldc=40, mu=2, nu=1, ku=4, lat=4: time=0.760000, mflop=128.673684
dNB=40, ldc=40, mu=2, nu=1, ku=4, lat=4: time=0.770000, mflop=127.002597
dNB=40, ldc=40, mu=2, nu=1, ku=4, lat=4: time=0.770000, mflop=127.002597

make mmcase muladd=0 lat=4 nb=40 mu=2 nu=1 ku=8 beta=1
dNB=40, ldc=40, mu=2, nu=1, ku=8, lat=4: time=0.640000, mflop=152.800000
dNB=40, ldc=40, mu=2, nu=1, ku=8, lat=4: time=0.630000, mflop=155.225397
dNB=40, ldc=40, mu=2, nu=1, ku=8, lat=4: time=0.630000, mflop=155.225397

make mmcase muladd=0 lat=4 nb=40 mu=2 nu=2 ku=40 beta=1
dNB=40, ldc=40, mu=2, nu=2, ku=40, lat=4: time=2.550000, mflop=38.349804
dNB=40, ldc=40, mu=2, nu=2, ku=40, lat=4: time=2.560000, mflop=38.200000
dNB=40, ldc=40, mu=2, nu=2, ku=40, lat=4: time=2.550000, mflop=38.349804
\end{verbatim}

\subsection{Complex {\it gemmK}}
Vainly hoping we were done, eh?  Nope, complex codes are special.  
ATLAS actually uses the real matrix multiply generator in order to
do complex multiplication.  It needs some tricks to do this, obviously.
The first thing to note is that if $X_r$ denotes the real elements of $X$,
and $X_i$ indicates the imaginary components, then complex matrix-matrix
multiply of the form $C \leftarrow A B + \beta C$ may be accomplished by 
the following four real matrix multiplies:
{\samepage
\begin{enumerate}
\item $C_r \leftarrow A_i B_i - \beta C_r$
\item $C_i \leftarrow A_i B_r + \beta C_i$
\item $C_r \leftarrow A_r B_r - C_r$
\item $C_i \leftarrow A_r B_i + C_i$
\end{enumerate}
}

This works fine assuming $\beta$ is real, otherwise $\beta$ must be applied
explicitly as a complex scalar, and then set $\beta=1$ in the above outline.

Therefore, in order to use this trick, upon copying $A$ and $B$ to block-major
storage, ATLAS also splits the arrays into real and imaginary components.
The only matrix not expressed as two real matrices is then $C$, and to
fix this problem, ATLAS demands that the complex {\it gemmK} stride $C$ by
2.  An example will solidify the confusion.

A simple 3-loop implementation of an ATLAS complex {\it gemmK} is:
\begin{verbatim}
void ATL_USERMM
   (const int M, const int N, const int K, const double alpha,
    const double *A, const int lda, const double *B, const int ldb,
    const double beta, double *C, const int ldc)
{
   int i, j, k;
   register double c00;

   for (j=0; j < N; j++)
   {
      for (i=0; i < M; i++)
      {
         #ifdef BETA0
            c00 = 0.0;
         #else
            c00 = C[2*(i+j*ldc)];
            #ifdef BETAX
               c00 *= beta;
            #endif
         #endif
         for (k=0; k < K; k++) c00 += A[k+i*lda] * B[k+j*ldb];
         C[2*(i+j*ldc)] = c00;
      }
   }
}
\end{verbatim}

First we test that it produces the right answer:
\begin{verbatim}
make cmmutstcase pre=z nb=40 mmrout=CASES/ATL_cmm1x1x1.c beta=0
make cmmutstcase pre=z nb=40 mmrout=CASES/ATL_cmm1x1x1.c beta=1
make cmmutstcase pre=z nb=40 mmrout=CASES/ATL_cmm1x1x1.c beta=8
\end{verbatim}

Then we scope the awesome performance:
\begin{verbatim}
make cmmucase pre=z nb=40 mmrout=CASES/ATL_cmm1x1x1.c beta=1
zNB=40, ldc=40, mu=4, nu=4, ku=1, lat=4: time=1.830000, mflop=53.438251
zNB=40, ldc=40, mu=4, nu=4, ku=1, lat=4: time=1.830000, mflop=53.438251
zNB=40, ldc=40, mu=4, nu=4, ku=1, lat=4: time=1.830000, mflop=53.438251
\end{verbatim}

Now, since we are clearly gluttons for punishment, we compare our
masterwork to ATLAS's generated kernel:
\begin{verbatim}
speedy. cat res/zMMRES 
MULADD  LAT  NB  MU  NU  KU  FFTCH  IFTCH  NFTCH    MFLOP
     0    5  36   2   1  36      0      2      1   186.42
     10

speedy. make mmcase muladd=0 lat=5 nb=36 mu=2 nu=1 ku=36 beta=1
dNB=36, ldc=36, mu=2, nu=1, ku=36, lat=5: time=0.500000, mflop=195.581952
dNB=36, ldc=36, mu=2, nu=1, ku=36, lat=5: time=0.490000, mflop=199.573420
dNB=36, ldc=36, mu=2, nu=1, ku=36, lat=5: time=0.490000, mflop=199.573420
\end{verbatim}

\subsection{What to do if you are writing in assembler}
If your kernel is written in gas assembler, you can tell the tester
and timer that by setting the appropriate compiler and flag macro on
the command line.  For single precision types, these macros for {\it gemmK}
are called {\tt SMC} and {\tt SMCFLAGS}, respectively, and they are
{\tt DMC} and {\tt DMCFLAGS} for double precision routines.
For instance, to test a {\tt DGEMM} code written in assembler, requiring
a 16 blocking factor, you'd issue:
\begin{verbatim}
   make ummcase pre=d mmrout=CASES/myassembler.c nb=16 \
        DMC=gcc DMCFLAGS="-x assembler-with-cpp"
\end{verbatim}

\subsection{Providing ATLAS with kernel cleanup code}
As mentioned in Section~\ref{sec-buildMM}, when any problem dimension
(eg., {\tt M}, {\tt N}, or {\tt K}) is not a multiple of $N_B$, ATLAS
must call cleanup code to handle the remainder.  When the user-contributed
kernel is only modestly faster than ATLAS's generated kernel, letting
the generated code handle cleanup will probably be an adequate solution.
However, when the user-contributed kernel is much faster than the generated
code, using the generated cleanup may represent a significant performance
drop for many problem sizes (see Section~\ref{sec-cleancost} for an
analysis of the cost of cleanup), and thus it becomes necessary for the user
to supply ATLAS with cleanup code as well.  In order to understand how
this is done, it is necessary to discuss how ATLAS does cleanup.

\subsubsection{ATLAS and cleanup}
As we have seen, ATLAS's main performance kernel is an L1 cache contained
matmul with fixed dimension $N_B$, and when a given problem dimension of
the general matmul is not a multiple of $N_B$, cleanup code must be
called.  It should be apparent that generating codes with all dimensions
fixed at compile time, as we do with the full kernel, is not a good idea
for cleanup, since it would result in roughly ${N_B}^3$ cleanup routines.
Not only would this make the average executable huge, but it would also
probably result in performance degredation due to constant instruction load.

ATLAS therefore normally generates a variable number of cleanup cases,
with the number of generated codes minimally being $7$, and the maximum
number being $6 + N_B$.  The number of generated codes can vary because
the $K$ cleanup routines are special, sometimes requiring $N_B$ different
codes to handle efficiently, as we will see below.

ATLAS splits the generated cleanup into these categories
\begin{enumerate}
\item {\bf M-cleanup} $M < N_B$ \&\& $N = K = N_B$:
   3 routines, corresponding to {\tt BETA} = 0, 1 and arbitrary
\item {\bf N-cleanup} $N < N_B$ \&\& $M = K = N_B$:
   3 routines, corresponding to {\tt BETA} = 0, 1 and arbitrary
\item {\bf K-cleanup} $K <= N_B$ \&\& $M \leq N_B$ \&\& $N \leq N_B$:
   Only one {\tt BETA} case (arbitrary), but may compile special case for
   each possible $K$ value, resulting in at least 1, and at most $N_B$
   K-cleanup routines
\end{enumerate}

So we see that K-cleanup is special in several ways.  First, it is the
most general cleanup routine, since it can handle multiple dimensions not
being less than $N_B$, whereas the M- and N-cleanup routines can only have
their respective dimensions less then $N_B$.  The second thing to note is
that we compile only the most general {\tt BETA} case for K-cleanup; this
is due to the fact that we may need $N_B$ different routines to handle
K-cleanup efficiently, and multiplying this number of routines by three seems
counterproductive.

The final difference in the K-cleanup is the fact that it potentially requires
$N_B$ different routines to support.  This is due to several factors.
Firstly, in ATLAS, the innermost loop in gemm is the K-loop, making it
very important for performance.  On systems without good loop handling,
such as the x86, heavy K unrollings are critical.
Secondly, the leading dimensions of the $A$ and $B$ matrices are fixed
to {\tt KB} due to the data copy, which allows for more efficient indexing
of these matrices.  If a routine takes run-time $K$ (rather than compile-time,
as when the dimension is fixed to {\tt KB}), it must also take run-time 
{\tt lda} and {\tt ldb}, and this extra indexing is too costly on
many architectures.

\subsubsection{User supplied cleanup}
Users can supply cleanup code for the following three cases only, all of
which come in the three {\tt BETA} variants:
\begin{enumerate}
\item {\bf M-cleanup} $M < N_B$ \&\& $N = K = N_B$
\item {\bf N-cleanup} $N < N_B$ \&\& $M = K = N_B$
\item {\bf K-cleanup} $K < N_B$ \&\& $M = N = N_B$
\end{enumerate}

The generated code handles all cleanup where more
than one dimension is less than the blocking factor.  This simplification
allows ATLAS to avoid having to test ${N_B}^3$ cases when selecting user
cleanup.  Once the matrices in question are larger than $N_B$, cleanup
with more than one dimension less than $N_B$ rapidly stops being a 
performance factor.  Small matrices where this cleanup is a factor are
almost certainly going to be handled by ATLAS's small-case code anyway,
so it seems unlikely that this simplification will hurt performance in
practice.  Section~\ref{sec-cleancost} shows this in a more formal way.

Users need to be very careful when supplying cleanup, because if the user
indicates that a dimension must be a compile-time variable, rather than
a runtime variable, ATLAS will generate up to $N_B$ routines to handle
user cleanup, and since user routines are compiled with all {\tt BETA}
variants, it is possible to generate $9 N_B$ cleanup cases, in addition
to ATLAS's generated cases.  It is therefore recommended that the user
supply cleanup that uses run-time arguments whenever possible, and indicate
kernels taking compile-time dimensions as not to be used for cleanup.

\subsubsection{Indicating cleanup in the index file}
Any routine that does not throw the flag value of 8 will be evaluated by
the install as a cleanup possibility.  Flag values are very important
for indicating opportunities for cleanup.  Here is an example from the
release:
\begin{verbatim}
  1 480 4 4 1 1 1 4 4 2 ATL_mm4x4x2US.c "V. Nguyen & P. Strazdins"
\end{verbatim}

OK, as always, we can read this to see that {\tt MB} and {\tt NB}
must be multiples of 4, and that {\tt KB} can be any value.  With
no flag modifiers, if we wanted to use the routine for K cleanup,
we would have to compile it into $N_B$ different routines, since
loop dimensions are compile-time parameters by default.  However,
this routine is modified by a flag value of 480.  What does this mean?
Consulting table~\ref{tab-mmflag}, we see that $32 + 64 + 128 + 256 = 480$,
which means lda and ldb are not restricted to {\tt KB} (i.e., they are
run-time parameters to the routine), the M-loop is controlled by a run-time
variable, the N-loop is controlled by a run-time variable, and the K-loop
is controlled by a run-time variable.  We therefore know that we can
use this routine for all cleanups (M-, N-, and K-cleanup), and we need only
one routine to do so (i.e., we do not have to compile $N_B$ routines to handle
all cases).  However, it can only be used for M- and N- cleanup cases where
the respective dimension is a multiple of 4.  Therefore, assuming this
kernel is superior to the generated code, it will be used for all K cleanup
routines.  However, for M and N cleanup, there will be something corresponding
to the following pseudocode:
\begin{verbatim}
   if (M % 4 == 0) call ATL_mm4x4x2US
   else call generated M cleanup
\end{verbatim}


It is clear that without overloading the flag value to an even more
ludicrous degree, that cleanup will eventually need to have it's own
index file.  For instance, it would be nice to be able to insist that
a particular K-cleanup code be used only when $K > 3$, for instance,
in addition to insisting it be a multiple of a particular value.  The fact
that cleanup does not already have such a seperate file simply represents
a design failure on my part; it was not until I had already produced the
system working as it does now that I saw its shortcomings, and then it
was too late to change for the release.  Subsequent developer releases
will probably address this shortcoming.

\subsubsection{Testing and timing cleanup}
Cleanup is tested in the same way as the normal kernel, but you need to
supply additional parameters.  {\tt M}, {\tt N}, and {\tt K} are the problem
dimensions, and {\tt MB}, {\tt NB}, {\tt KB} are the blocking factors.
If the blocking factors are set to zero, that means they are run-time
parameters to the routine.  {\tt lda}, {\tt ldb}, {\tt ldc} are the leading
dimensions of the operand arrays, and they default to {\tt KB}, {\tt KB},
and zero, respectively.

Here is an example of testing an M-cleanup routine, insisting that
M is a run-time argument:
\begin{verbatim}
make mmutstcase mmrout=CASES/ATL_mm1x1x1.c pre=d M=17 N=40 K=40 \
     mb=0 nb=40 kb=40
\end{verbatim}

Here is timing the same routine, but insisting that the M-loop is fixed at
compile time:
\begin{verbatim}
make ummcase mmrout=CASES/ATL_mm1x1x1.c pre=d M=17 N=40 K=40 \
     mb=17 nb=40 kb=40
\end{verbatim}

Here's testing a K-cleanup routine, taking run-time K and leading dimensions:
\begin{verbatim}
make mmutstcase mmrout=CASES/ATL_mm1x1x1.c pre=d M=40 N=40 K=27 \
     mb=40 nb=40 kb=0 lda=0 ldb=0
\end{verbatim}

The same test taking compile-time K and leading dimensions:
\begin{verbatim}
make mmutstcase mmrout=CASES/ATL_mm1x1x1.c pre=d M=40 N=40 K=27 \
     mb=40 nb=40 kb=27 lda=27 ldb=27
\end{verbatim}

\subsubsection{Importance of cleanup}
\label{sec-cleancost}
In analyzing the importance of good cleanup for performance, it is necessary
to recognize the various types that can occur.  The cleanup that user's can
supply to ATLAS is {\em one dimensional cleanup}, i.e., only one of the three
possible dimensions is less than $N_B$.  There is also 2 and 3 dimensional
cleanup.  To give an idea of the relative importance of various catagories
of computation, it is roughly true that the matmul kernel is a cubic cost,
the one dimensional cleanup is a square cost, the two dimensional cleanup
is a linear cost, and the three dimensional cleanup is $O(1)$.

This is shown more formally below.  Define $M_r = M$ mod $N_B$, let 
$m, n, k$ be the dimensional arguments to the {\it gemmK} and/or cleanup,
and remember that matrix multiplication takes $2 M N K$ flops, and we see
that the flop count for each catagory is:
\begin{itemize}
\item {$\bf m=n=k=N_B$}: 
   $\Wfloor{\frac{M}{N_B}} \Wfloor{\frac{N}{N_B}} \Wfloor{\frac{K}{N_B}} 
   2 {N_B}^3 \Longrightarrow \Wfloor{\frac{N}{N_B}}^3 ~2 {N_B}^3$

\item $\bf m < N_B, n = k = N_B$:
   $\Wfloor{\frac{N}{N_B}} \Wfloor{\frac{K}{N_B}} 2 M_r {N_B}^2
    \Longrightarrow \Wfloor{\frac{N}{N_B}}^2 ~2 N_r {N_B}^2$
\item $\bf n < N_B, m = k = N_B$:
   $\Wfloor{\frac{M}{N_B}} \Wfloor{\frac{K}{N_B}} 2 N_r {N_B}^2
    \Longrightarrow \Wfloor{\frac{N}{N_B}}^2 ~2 N_r {N_B}^2$
\item $\bf k < N_B, m = n = N_B$:
   $\Wfloor{\frac{M}{N_B}} \Wfloor{\frac{N}{N_B}} 2 K_r {N_B}^2
    \Longrightarrow \Wfloor{\frac{N}{N_B}}^2 ~2 N_r {N_B}^2$

\item $\bf m < N_B, n < N_B, k = N_B$:
   $\Wfloor{\frac{K}{N_B}} 2 M_r N_r {N_B} 
    \Longrightarrow \Wfloor{\frac{N}{N_B}} 2 {N_r}^2 N_B$
\item $\bf m < N_B, k < N_B, n = N_B$:
   $\Wfloor{\frac{N}{N_B}} 2 M_r K_r {N_B}
    \Longrightarrow \Wfloor{\frac{N}{N_B}} 2 {N_r}^2 N_B$
\item $\bf n < N_B, k < N_B, m = N_B$:
   $\Wfloor{\frac{M}{N_B}} 2 N_r K_r {N_B}
    \Longrightarrow \Wfloor{\frac{N}{N_B}} 2 {N_r}^2 N_B$
\item $\bf m < N_B, n < N_B, k < N_B$:
   $2 M_r N_r K_r \Longrightarrow 2 {N_r}^3$
\end{itemize}

Note that the simplified equations to the right of $\Longrightarrow$ 
assume the square case, i.e. $M = K = N$.  The above analysis can now
be grouped into the catagories of interest as in:
\begin{itemize}
  \item {\bf kernel} : $\Wfloor{\frac{N}{N_B}}^3 ~2 {N_B}^3$
  \item {\bf 1D cleanup}: $3 \Wfloor{\frac{N}{N_B}}^2 ~2 N_r {N_B}^2
     ~~~~<~~~~ 3  \Wfloor{\frac{N}{N_B}}^2 ~2 {N_B}^3$
  \item {\bf 2D cleanup}: $\Wfloor{\frac{N}{N_B}} 2 {N_r}^2 N_B
     ~~~~~~~~<~~~~~ 3 \Wfloor{\frac{N}{N_B}} ~2 {N_B}^3$
  \item {\bf 3D cleanup}: $2 {N_r}^3 ~~~~~~~~~~~~~~~~~~~<~~~~~ 2 {N_B}^3$
\end{itemize}
The simplified equations to the right of the $<$ above provide a safe
upper bound on cleanup cost by setting $N_r = N_B$
(in reality, $0 < N_r < N_B$, of course).

With this analysis, we can easily see why it is not important for the
user to be able to contribute 2D and 3D cleanup cases: remember that
all of these kernels are for ATLAS's {\em large-case} gemm.  ATLAS has a 
seperate small-case gemm, which is invoked when the problem is so small
that the $2 N^2$ copy cost is significant compared to the $2 N^3$ computational
costs.  So, in the cases where the $O(N)$ 2D cleanup or $O(1)$ 3D cleanup
costs are prohibitive, this large-case gemm will probably not be used anyway.

\subsection{{\it gemmK} usage notes}
The assumptions behind this kernel are that the input operands are
loaded to L1 only one time (i.e., the blocking guarantees that all
of the matrix accessed in the inner loop plus the active panel of
the matrix in the outer loop fits in L1).  For very large caches, all
three operands may fit into cache, but this is typically not the
case.  Because this {\it gemmK} is called by routines that place $K$
as the innermost loop, the output operand $C$ will typically come
from the L2 cache (except, obviously, on the first of the 
$\frac{K}{N_B}$ such calls).  ATLAS uses the JIK loop variant of
on-chip multiply, and thus all of $A$ fits in cache, with nu columns
of $B$.  To take an example, say you are using mu = nu = 4, with
$N_B = 40$, then the idea is that the $40 \times 40$ piece of $A$,
along with the $40 \times 4$ piece of $B$ (the active panel of $B$),
and the $4 \times 4$ section of $C$ all fit into cache at once, with
enough room for the load of the next step, and any junk the algorithm
might have in L1.  That panel of $B$ is applied to all of $A$, and then
a new panel is loaded.  Since the panel has been applied to all $A$, it
will never be reloaded, and thus we see that $B$ is loaded to L1 only
one time.  Since all of $A$ fits in L1, and we keep it there across all
panels of $B$, it is also loaded to L1 only one time. 

If written appropriately, loading all of $B$ with a few rows
of $A$ should theoretically be just as efficient (i.e., the IJK variant
of matmul).  However, the variants where $K$ is not the innermost loop
are unlikely to work well in ATLAS, if for no other reason than the 
transpose settings we have chosen militate against it.

Note that the $\beta = 0$ case must not read $C$, since the memory may
legally be unitialized.

\subsection{Getting ATLAS to use your kernel}
OK, so let's say you've got a kernel that is faster than what ATLAS is
presently using, how do you get ATLAS to use it?  First, of course, you
put the source in the CASES directory, and update the appropriate
{\tt <pre>cases} index file.  Then, you take different steps depending on
how you wish to do the install, as discussed in the following sections.
In all of these discussions, {\tt <pre>} is replaced by your type/precision
modifier (one of {\tt s}, {\tt d}, {\tt c}, {\tt z}).

\subsubsection{Putting it in by hand}
In an already-installed ATLAS, you can make ATLAS reinstall just the
kernel.  From your {\tt BLDdir/tune/blas/gemm/} directory, issue
these commands:
\begin{verbatim}
   rm res/<pre>u*
   rm res/<pre>MMRES
   ./xmmsearch -p <pre>
   make <pre>install
\end{verbatim}

\subsubsection{With a fresh install}
\label{sec-archdefModify}
First, run config as usual.  Then, tail the created {\tt Make.inc}
file, and see if the macro {\tt INSTFLAGS} includes {\tt -a 1}.  If so,
ATLAS has some architectural defaults for your architecture (though
perhaps not for your compiler, if you have forced the use of a non-default
compiler), which won't include
your shiny new kernel.  So, you will need to remove the files indicating
the default {\tt gemmK} kernel for your precision.  To do this, scope
your {\tt ARCH} setting in your {\tt Make.inc}.  For the purposes of
this discussion, let us say it set to {\tt Core2Duo64SSE3} (i.e., in
the below example, substitute the definition of {\tt ARCH} for
{\tt Core2Duo64SSE3}).  Go to {\tt ATLAS/CONFIG/ARCHS}, and issue
the following commands:
%{\tt ARCHDEF} is set.  If so, cd to the
%directory pointed at by {\tt ARCHDEF}, and then issue
\begin{verbatim}
   gunzip -c Core2Duo64SSE3.tgz | tar xvf -
   rm Core2Duo64SSE3/gemm/gcc/<pre>u* 
   rm Core2Duo64SSE3/gemm/gcc/<pre>MMRES
   rm Core2Duo64SSE3.tgz
   tar cvf Core2Duo64SSE3.tar Core2Duo64SSE3
   gzip Core2Duo64SSE3.tar
   mv Core2Duo64SSE3.tar.gz Core2Duo64SSE3.tgz
\end{verbatim}

Now, continue install as normal, and your kernel should be used if it beats
what ATLAS is presently using.  Note that this assumes you are using {\tt gcc}
as the {\tt gemmK} compiler, which is the default on most systems.  If
you are using a different compiler, you would substitute its name instead
of {\tt gcc} in the above lines.  If there is no subdirectory with the
name of your compiler in the tarfile, ATLAS has no architectural defaults
for that compiler, and thus you need to make no changes to the tarfile.

\subsubsection{With an old install, but using full install command}
When rebuilding an old install, the main trap is to avoid having
architectural defaults make it so you don't time your new kernel.
Follow the instructions given in Section~\ref{sec-archdefModify},
but additionally make sure you delete any prexisting directory
that matches your {\tt ARCH} definiton.  Therefore, in the above
example, in the {\tt ATLAS/CONFIG/ARCHS} directory, you would
additionally issue:
\begin{verbatim}
   rm -rf Core2Duo64SSE3
\end{verbatim}
If such a subdirectory existed.

From your {\tt BLDdir} directory, then issue:
\begin{verbatim}
   rm bin/INSTALL_LOG/*
   rm tune/blas/gemm/res/<pre>u*
   rm tune/blas/gemm/res/<pre>MMRES
   make build
\end{verbatim}

\subsection{Contributing a complete GEMM implementation}
\textbf{
This feature has been temporarily disabled in 3.8, though it may be
re-enabled in the 3.9 series if there is user demand.  This section
is therefore kept around solely historical purposes, and will need
to be updated if the feature is added back in.
}

Contributing an L1 kernel is the prefered method of user contribution
for Level 3 BLAS speedup, but it is not the only one supported by ATLAS.
ATLAS also allows a user to contribute a complete system-specific
GEMM implementation.  This method of contribution is far less desirable
than kernel contribution, and thus the standards of acceptance are
correspondingly higher.

When only a kernel is contributed, it is only used when timings indicate
it is superior to the best ATLAS-supplied routine for a given architecture.
Because kernel routines are called in known ways by the ATLAS infrastructure,
the timer can be made to accurately reflect typical usage.  A full GEMM,
which is to all intents called directly by the user, has no ``typical''
usage, and the timer is thus not able to ensure that the user's full
GEMM is superior to that supplied by ATLAS in a system-independent way,
even if the additional installation time required to choose amoung full
GEMM implementations were allowed.  Thus, full GEMM implementations will
be used only when ATLAS's configuration detects a known architecture where
the ATLAS team has certified the full GEMM to be significantly better than
ATLAS's native GEMM, across the entire spectrum of problem shapes and sizes
(with the exception of those shapes and sizes handled by ATLAS's non-copy
code, as explained below).

As explained in Section~\ref{sec-buildMM}, ATLAS has both a small-case
matmul, which does not copy the user's input operands, and a large-case
code that does.  The user contributed GEMM replaces ATLAS's large-case
GEMM, and then timings are used as normal to determine the crossover points at
which the contributed GEMM outperforms ATLAS's small-case code.

\subsubsection{Supplying ATLAS with what it needs}
ATLAS expects that the contributed GEMM will have its own architecture-
specific subdirectory, just as with all other ATLAS source directories.
That directory is indicated to ATLAS by the {\tt UMMdir} macro set in
{\tt Make.inc}.  For instance, on the alpha platform, Mr. Goto's GEMM
is used by ATLAS, and {UMMdir} is therefore set to:
{\tt \$(TOPdir)/src/blas/gemm/GOTO/\$(ARCH)}.

In this directory, there must be a master makefile, called {\tt Makefile},
which minimally contains the following targets:
\begin{itemize}
\item {\tt susermm} : builds the single precision real gemm with
all its dependencies
\item {\tt dusermm} : builds the double precision real gemm with
all its dependencies
\item {\tt cusermm} : builds the single precision complex gemm with
all its dependencies
\item {\tt zusermm} : builds the double precision complex gemm with
all its dependencies
\item {\tt sclean} : deletes all non-library files created by {\tt susermm}
\item {\tt dclean} : deletes all non-library files created by {\tt dusermm}
\item {\tt cclean} : deletes all non-library files created by {\tt cusermm}
\item {\tt zclean} : deletes all non-library files created by {\tt zusermm}
\end{itemize}

For each precision, ATLAS calls the user's GEMM using this API:
\begin{verbatim}
int ATL_U<pre>usergemm(const enum ATLAS_TRANS TA, const enum ATLAS_TRANS TB,
                       const int M, const int N, const int K,
                       const SCALAR alpha, const TYPE *A, const int lda,
                       const TYPE *B, const int ldb, const SCALAR beta,
                       TYPE *C, const int ldc)
\end{verbatim}
where,\\
\begin{tabular}{||l|l|l|l|l||}\hline\hline
\verb+<pre>+ : & {\tt s} & {\tt d} & {\tt c} & {\tt z} \\\hline\hline
SCALAR & {\tt float} & {\tt double} & {\tt float*} & {\tt float*} \\\hline
TYPE   & {\tt float} & {\tt double} & {\tt float} & {\tt float} \\\hline\hline
\end{tabular}\\\\

This routine should return {\tt 0} upon successful invocation, and 
{\tt -1} if unable to malloc enough memory.  Other errors may be
signaled by returning a value of {\tt 2}.  On error in this routine,
ATLAS will call the no-copy code to get the answer, so a return value
of {\tt 1} indicates that ATLAS should do this.  If a fatal error
occurs, or if an error occurs after operands have been modified (i.e.,
calling the no-copy code will no longer produce the correct answer),
then execution should be halted.

ATLAS's interface routines have already done all required error checking,
so the user need not check the input arguments in this routine, or any
of the lower-level user contributed routines.

\subsubsection{What to do if you don't supply all precisions}

Remember that what ATLAS is doing is substituting your GEMM for its own
large-case GEMM.  However, ATLAS's large-case GEMM is still compiled in
the library, it is just not being used.  The following code will call
ATLAS's large case code for all precisions:
\begin{verbatim}
#include "atlas_misc.h"
#include "atlas_lvl3.h"

int Mjoin(Mjoin(ATL_U,PRE),usergemm)
   (const enum ATLAS_TRANS TA, const enum ATLAS_TRANS TB,
    const int M, const int N, const int K, const SCALAR alpha,
    const TYPE *A, const int lda, const TYPE *B, const int ldb,
    const SCALAR beta, TYPE *C, const int ldc)
{
   int ierr;

   if (N >= M)
   {
      if ( ierr = Mjoin(PATL,mmJIK)(TA, TB, M, N, K, alpha, A, lda, B, ldb,
                                    beta, C, ldc) )
         ierr = Mjoin(PATL,mmIJK)(TA, TB, M, N, K, alpha, A, lda, B, ldb,
                                  beta, C, ldc);
   }
   else
   {
      if ( ierr = Mjoin(PATL,mmIJK)(TA, TB, M, N, K, alpha, A, lda, B, ldb,
                                    beta, C, ldc) )
         ierr = Mjoin(PATL,mmJIK)(TA, TB, M, N, K, alpha, A, lda, B, ldb,
                                  beta, C, ldc);
   }
   return (ierr);
}
\end{verbatim}

So, you can use the above code to have ATLAS call it's normal routines
for those precisions you do not supply, and call your own otherwise.
In order to help understand this process, ATLAS includes a 
``user-supplied'' GEMM that simply calls ATLAS's own GEMM for all
precisions, in {\tt ATLAS/src/blas/gemm/UMMEXAMPLE}.  If, for
instance, you have a single precision real GEMM but nothing else,
you can take this directory as a starting point, adding your
own commands for single precision real in the Makefile, etc,
and leaving everything else alone.

In order to do this, you should do the following:
%(where {\tt <arch>} is
%replaced by the {\tt ARCH} string of your {\tt Make.<arch>}; eg, 
%{\tt Linux\_PII}, {\tt AIX\_POWER3}, {\tt OSF\_21264}, etc.):
\begin{enumerate}
\item In {\tt ATLAS/src/blas/gemm/UMMEXAMPLE}:
   \begin{itemize}
   \item {\tt mkdir <arch>}
   \item {\tt cp Makefile <arch>/.}
   \item {\tt ln -s ../../../../../Make.<arch> Make.inc}
   \item Modify {\tt <arch>/Makefile} to compile your single precision routine
   \item Follow the instructions given in Section~\ref{sec-force},
which discusses how to make ATLAS use a contributed GEMM that config
has not setup for you
   \end{itemize}
\end{enumerate}


\subsubsection{Forcing ATLAS to use your GEMM}
\label{sec-force}

If ATLAS detects you are on a platform where a contributed full GEMM is
superior to ATLAS's large-case GEMM, ATLAS will automatically handle
the details of making ATLAS call the user-contributed routine.  If,
however, you wish to force ATLAS to use your GEMM (for instance, you
are testing your code before contribution, or just want to utilize ATLAS
for complete BLAS coverage with your GEMM), you should take
the following steps, after creating the appropriate subdirectory and
API as previously described:
\begin{enumerate}
\item Edit your {\tt Make.inc} file: 
 \begin{itemize}
 \item Change {\tt UMMdir} to point to your full GEMM's architecture
       subdirectory
 \item Add {\tt -DUSERGEMM} to the {\tt CDEFS} macro.
 \end{itemize}
\item In {\tt ATLAS/src/blas/gemm}, touch {\tt ATL\_gemmXX.c} and
{\tt ATL\_AgemmXX.c} to force recompilation
\item In {\tt BLDdir/src/blas/gemm/} type {\tt make lib}
\item In {\tt BLDdir/include/}, issue 
  \begin{itemize}
   \item {\tt rm ?Xover.h atlas\_cacheedge.h}
   \item {\tt touch altas\_cacheedge.h sXover.h dXover.h cXover.h zXover.h}
  \end{itemize}
\item In {\tt BLDdir/tune/blas/gemm/}, issue:
  \begin{itemize}
  \item {\tt rm res/?Xover.h res/atlas\_cacheedge.h}
  \item {\tt make res/atlas\_cacheedge.h}
  \item {\tt make res/sXover.h pre=s}
  \item {\tt make res/dXover.h pre=d}
  \item {\tt make res/cXover.h pre=c}
  \item {\tt make res/zXover.h pre=z}
  \end{itemize}
\end{enumerate}


\section{Speeding up the Level 3 BLAS using access-major GEMM}

The performance kernel for the entire Level 3 BLAS is matrix multiply.
Matrix multiply is written in terms of a lower-level building block that
we call {\it gemmK}.  {\it gemmK} is a special matrix multiply where
both the $M$ and $N$ dimensions are runtime variables known to be multiples
of their respective unrolling factors ($mu$ and $nu$).  The $K$ dimension
can be either a run-time variable or compile time constant, as specified
by the programmer.  This kernel is always called with all dimensions set
small enough to encourage cache blocking.

ATLAS actually has two different classes of GEMM kernels: one for copied
matrices ({\it gemmK}), and one that operates directly on the user's matrices
without a copy.  For matrices of
sufficient size, ATLAS copies the input matrix into {\em access-major} storage.
In access-major storage, the $N_B \times N_B$ blocks operated on by the
{\it gemmK} are actually contiguous, and the copy reorders them so that
{\it gemmK} accesses all operands in strict memory order.  This optimization
prevents unnecessary cache misses, cache conflicts, TLB problems, and ensures
that the kernel uses the minimal constant bus traffic.  However, for
sufficiently
small matrices, the cost of this data copy is prohibitively expensive,
and thus ATLAS has kernels that operate on non-copied data.  However,
without the copy to simplify the process, there are multiple non-copy
kernels (differing kernels for differing transpose settings, for instance).
Since the non-copy kernels are typically only used for very small problems,
and they are much more complex, ATLAS presently accepts contributed code
only for the copy matmul kernel.  For most problems, well over 98\% of ATLAS
time is spent in the copy matmul kernel, so this should not be much of 
a problem.

\subsection{Building the General Matrix Multiply From {\it gemmK}}
\label{sec-buildAMM}

This section describes the code necessary to build the BLAS's general
matrix-matrix multiply using an L1 cache-contained
matmul (hereafter referred to as {\it gemmK}).

For our present discussion, it is enough to know
that ATLAS has at its disposal highly optimized routines for doing matrix
multiplies whose dimensions are chosen such that cache blocking is not
required (i.e., the hand-written code discussed in this section deals
with cache blocking; the generated/user supplied kernel assumes things fit
into cache).

When the user calls GEMM, ATLAS must decide whether the problem
is large enough to tolerate copying the input matrices $A$ and $B$.
If the matrices are large enough to support this $O(N^2)$ overhead, ATLAS will
copy $A$ and $B$
into block-major format.  ATLAS's block-major format breaks up the input
matrices into contiguous blocks of a fixed size $N_B$, where $N_B$ is chosen
in order to maximize L1 cache
reuse.  Once in block-major format, the blocks are contiguous, which
eliminates TLB problems, minimizes cache thrashing and maximizes cache line
use.  It also allows ATLAS to apply alpha (if alpha is not already one) to the
smaller of $A$ or $B$, thus minimizing this cost as well.  Finally, the
package can use the copy to transform the problem to a particular transpose
setting, which for load and indexing optimization, is set so
A is copied to transposed form, and B is in normal (non-transposed) form.
This means our L1-cache contained code is of the form
$C \leftarrow A^T B$, $C \leftarrow A^T B + C$,
and $C \leftarrow A^T B + \beta C$,
{\em where all dimensions, including the non-contiguous stride, are known
to be $N_B$}.  Knowing all of the dimensions of the loops allows for
arbitrary unrollings (i.e., if the instruction cache could support it, ATLAS
could unroll all loops completely, so that the L1 cache-contained multiply
had no loops at all).  Further, when the code generator knows leading dimension
of the matrices (i.e., the row stride), all indexing can be done up front,
without the need for expensive integer or pointer computations.

If the matrices are too small, the $O(N^2)$ data copy cost can actually
dominate the algorithm cost, even though the computation cost is $O(N^3)$.
For these matrices, ATLAS will call an gemm kernel which operates on non-copied
matrices (i.e. directly on the user's operands).  The non-copy matmul kernels
will generally not be as efficient as the even the generated copy {\it gemmK};
at this problem
size the main drawback is the additional pointer arithmetic required in order
to support the user-supplied leading dimension and its affect on the cost of
the memory load (which varies according to transpose settings, as well as
architectural features).

The choice of when a copy is dictated and when it is prohibitively expensive
is an AEOS parameter; it turns out that this crossover point depends strongly
both on the particular architecture, and the shape of the operands
(matrix shape effectively sets limits on which matrix dimensions can
enjoy cache reuse).  To handle this problem, ATLAS simply compares the
speed of the copy and non-copy matmul kernels for variously shaped matrices,
varying the problem size until the copy code provides a speedup (on some
platforms, and with some shapes, this point is never reached).  These
crossover points are determined at install time, and then used to make
this decision at runtime.  Because it is the dominant case, this paper
describes only the copied matmul algorithm in detail.

Figure~\ref{fig-GemmPanel} shows the necessary steps for computing a
$N_B \times N_B$ section of $C$ using {\it gemmK}.

\begin{figure}[hbtp]
\begin{picture}(430,100)(-20,0)
\put(20,15){\framebox(20,20){$C_{3,2}$}}
\put(150,15){\framebox(20,20){$A_{3,1}$}}
\put(170,15){\framebox(20,20){$A_{3,2}$}}
\put(190,15){\framebox(10,20){}}
\put(-20,37){$M$}
\put(50,80){$N$}
\put(0,0){\framebox(100,75){\bf C}}
\put(120,37){$\leftarrow$}
\put(135,37){$M$}
\put(173,80){$K$}
\put(150,0){\framebox(50,75){\bf A}}
\put(280,65){$N$}
\put(345,37){$K$}
\put(210,36){$\times$}
\put(230,10){\framebox(100,50){\bf B}}
\put(250,40){\framebox(20,20){$B_{1,2}$}}
\put(250,20){\framebox(20,20){$B_{2,2}$}}
\put(250,10){\framebox(20,10){$B_{3,2}$}}
\end{picture}
\caption{One step of matrix-matrix multiply\label{fig-GemmPanel}}
\end{figure}

More formally, the following actions are performed in order to
calculate the $N_B \times N_B$ block $C_{i,j}$, where $i$ and $j$ are in
the range $0 \leq i < \Wceil{M/N_B}$, $0 \leq j < \Wceil{N/N_B}$:

\begin{enumerate}
 \item Call {\it gemmK} of the correct form based on user-defined $\beta$
       (eg. if $\beta == 0$, use $C \leftarrow A B$) to multiply block $0$
       of the row panel $i$ of $A$ with block $0$ of the column panel $j$
       of $B$.
 \item Call {\it gemmK} of form $C \leftarrow A B + C$
       to multiply block $k$ of the row panel $i$ of
       $A$ with block $k$ of the column panel $j$ of $B$,
       $\forall k, 1 \leq k < \Wceil{K/N_B}$.
\end{enumerate}

As this example demonstrates, if a given dimension is not a multiple of
the L1 blocking factor $N_B$, partial blocks results.  ATLAS has special
routines that handle cases where one or more dimension is less than $N_B$;
these routines are referred to as {\em cleanup} codes.

\subsection{The Main GEMM Kernel, {\it gemmK}}
So, there are actually three {\it gemmK} kernels (corresponding to different
$\beta$ values), and perform the operations:
$C \leftarrow A B$,
$C \leftarrow A B + C$,
$C \leftarrow A B - C$.  All input arrays ($A, B, C$) are
access-major. 

\subsubsection{{\it gemmK}  macro definitions}
In order to make writing a {\it gemmK} easier, ATLAS defines several 
cpp macros for programmer use.  Examples in subsequent sections should 
illustrate the use of these macros, so we merely define them here.

First, ATLAS defines the macro {\tt ATL\_USERMM}
to the appropriate ATLAS internal kernel name.  Second, it defines one
of  {\tt SREAL}, {\tt DREAL}, {\tt SCPLX}, {\tt DCPLX}, according to
the data type being compiled (single precision real, double precision
real, single precision complex, double precision complex, respectively).

Similarly, ATLAS defines a macro indicating the $\beta$ case being
compiled, {\tt BETA1} ($\beta$ should be assumed to be 1.0), {\tt BETA0}
($\beta$ should be assumed to be 0.0), and {\tt BETAN1}
($\beta$ should be assumed to be -1.0).

Finally, if the user has called for a compile-time $K$ dimension,
the macro {\tt KB} will be set to the required compile-time dimension.

\subsubsection{{\it gemmK} API}
The {\it gemmK} API may be summarized as:

\vspace*{.1in}

\noindent
\verb+#if defined(SREAL) || defined(SCPLX)+
\begin{routdef}{1.8}
\rditem{~~~void ATL\_USERMM}
{size\_t nmu, size\_t nnu, size\_t K, const TYPE *pA, const TYPE *pB, float *pC,
 const float *pAn, const float *pBn, const float *pCn}
\end{routdef}

\noindent
\verb+#elif defined(DREAL) || defined(DCPLX)+
\begin{routdef}{1.8}
\rditem{~~~void ATL\_USERMM}
{size\_t nmu, size\_t nnu, size\_t K, const double *pA, const double *pB, 
 double *pC, const double *pAn, const double *pBn, const double *pCn}
\end{routdef}

\noindent
\verb+#endif+

$nmu$ is the number of unrolled $M$ iterations, i.e. $\frac{M}{mu}$, and
$nnu$ is the same for the $N$ dimension.  The $K$ parameter is ignored if
the kernel uses a compile-time $K$, otherwise it specifies the $K$ dimension.
The $pA$, $pB$ and $pC$ pointers all point to access-major arrays.
Finally, the $pAn$, $pBn$, and $pCn$ kernels are pointers to be used for
prefetching succeeding blocks of the named arrays.

\subsubsection{{\it gemmK} description file}
In the install process, ATLAS first searches through the {\it gemmK}
implementations provided by the ATLAS matmul generator.  When the
best generated code is found, the user contributed codes are timed
to see if they can beat the generated code.  The {\tt gemmK} multiple
implementation search
script opens a description file for each precision 
({\tt scases.dsc}, {\tt dcases.dsc}, {\tt ccases.dsc}, {\tt zcases.dsc})
in the {\tt BLDdir/tune/blas/gemm/} directory,
to see what user-contributed codes are available.  This master index
file is actually generated based on several user-supplied files from
{\tt ATLAS/tune/blas/gemm/CASES} (see Section~\ref{sec-index} for the
names and definitions of these files).  The format for all these files
is the same, and is described in the following paragraphs.

The first line of each file is a comment line, and is ignored.  The next
line indicates the number of user-contributed codes to search, and
each subsequent line supplies information about a given user-supplied
{\it gemmK}.  The form of these lines is:\\
\verb+<ID> <flag> <mb> <nb> <kb> <muladd> <lat> <mu> <nu> <ku> <rout> "<author>"+

\verb+<rout>+ and \verb+<author>"+ are strings, and the rest of the
parameters are signed integers.  

The meaning of these parameters are:
\begin{itemize}
\item {\verb+ID+}: Strictly positive integer which uniquely identifies this
descriptor line.  ID must by unique only within a precision.
\item {\verb+<flag>+}: flag indicating special conditions.  See table below.

\item{\verb+<mb>+, \verb+<nb>+, \verb+<kb>+}: 
   Used to indicate restriction on the input parameter $M$ ($N$, $K$ resp.),
   and its associated blocking {\tt MB} ({\tt NB}, {\tt KB}, resp.).
   If the value is zero, the internal routine handles any $M$; i.e.
   the loop-limit is a runtime variable.  If the value is negative, then
   $M$ = {\tt MB} = -\verb+<mb>+ (i.e., the blocking factor cannot be
   varied using a macro).  If the value is positive, the blocking factor
   can be varied by setting the appropriate macro 
   ({\tt MB} {\tt NB}, {\tt KB}, resp.), but the blocking factor must be
   a multiple of the value.  Therefore, setting \verb+<mb>+ = 4, indicates
   that {\tt MB} must be a multiple of 4, while setting it to 1 indicates
   that {\tt MB} is an arbitrary compile-time constant.
\item{\verb+<muladd>+}:
   Set to zero if you are using separate multiply and add instructions, 1
   otherwise.  If you don't know the answer, put 1.
\item{\verb+<lat>+}:
   Set to the latency you use between floating point instructions.
   If you don't know the answer, put 1.
\item{\verb+<mu>+}:
   Unrolling you are using for the $M$ loop.
\item{\verb+<nu>+}:
   Unrolling you are using for the $N$ loop.
\item{\verb+<ku>+}:
   Unrolling you are using for the $K$ loop.
\item{\verb+<rout>+}:
   The filename of the user-contributed routine, relative to the path
   {\tt ATLAS/tune/blas/gemm/CASES}.  Maximum length 64 chars.
\item{\verb+<author>+}:
   The name of the author or authors, enclosed in quotes.  
   Maximum length 64 chars.
\end{itemize}

Table~\ref{tab-mmflag} summarizes the presently defined {\tt flag} values.\\
\begin{table}[h!]
\begin{center}
\begin{tabular}{||r|l||}\hline\hline
FLAG & MEANING \\\hline\hline
   0 & Normal \\\hline
   8 & Do not consider this kernel for cleanup\\\hline
  16 & Consider this kernel for cleanup {\em only}\\\hline 
  32 & lda and ldb are not restricted to {\tt KB} \\\hline 
  64 & {\tt mb} provides run-time constraint, not compile-time\\\hline 
 128 & {\tt nb} provides run-time constraint, not compile-time\\\hline 
 256 & {\tt kb} provides run-time constraint, not compile-time\\\hline 
 512 & This kernel needs $4 N_b \le cacheelts$\\\hline
  \hline
\end{tabular}
\end{center}
\caption{Matmul index routine flag variables\label{tab-mmflag}}
\end{table}

Here's an example:
\begin{verbatim}
<ID> <flag> <mb> <nb> <kb> <muladd> <lat> <mu> <nu> <ku> <rout> "<Contributer>"
3
 1 0 0 0 0 1 1 1 1 1 ATL_mm1x1x1.c "R. Clint Whaley"
 2 0 1 1 1 1 1 1 1 1 ATL_mm1x1x1b.c "R. Clint Whaley"
 3 0 1 1 8 1 1 1 1 4 ATL_mm2.c "R. Clint Whaley"
\end{verbatim}

So, we have 3 user-supplied routines, all written by me.  The first loops
over $M$, $N$, and $K$, but the following two routines loop over the cpp
macros {\tt MB}, {\tt NB}, {\tt KB}.  The third routine insists that
{\tt KB} be a multiple of 8.  The first two routines don't unroll
any of the loops, while the third unrolls the K loop to a depth of 4.
They all use a combined muladd style of programming, and don't worry
about latency.

\subsubsection{Index filenames}
\label{sec-index}
As previously mentioned, ATLAS builds a system and type dependent index
file from user-supplied files in {\tt ATLAS/tune/blas/gemm/CASES}.  This
is done so that the all routines do not need to be run on all machines
(i.e., no need to waste time trying to run SSE-enabled assembly routines
when on a Dec ev56).  Here is a list of description files presently queried
by ATLAS when building the full search index:
\begin{enumerate}
 \item {\bf [s,d,c,z]cases.0}:  Any user-contributed kernel which is system
       independent (i.e. doesn't require a particular compiler, etc)
       \begin{itemize}
       \item Convention is to choose IDs in range: $0 < ID < 100$.
       \end{itemize}
 \item {\bf [s,d,c,z]cases.flg}: Any user-contributed kernel requiring specific
       compiler and/or flags
       \begin{itemize}
       \item Convention is to choose IDs in range: $300 \leq ID < 400$.
       \end{itemize}
 \item {\bf [s,c]cases.3DN}: Kernels requiring 3DNow! to run.
       \begin{itemize}
       \item Convention is to choose IDs in range: $100 \leq ID < 200$.
       \end{itemize}
 \item {\bf [s,c]cases.SSE}: Kernels requiring SSE1 to run.
       \begin{itemize}
       \item Convention is to choose IDs in range: $200 \leq ID < 300$.
       \end{itemize}
\end{enumerate}

\subsection{Putting it together with some examples}

Let's say we decide to cover the basics, the classical 3 do loop
implementation of matmul would be:
\begin{verbatim}
void ATL_USERMM
   (const int M, const int N, const int K, const double alpha,
    const double *A, const int lda, const double *B, const int ldb,
    const double beta, double *C, const int ldc)
{
   int i, j, k;
   register double c00;

   for (j=0; j < N; j++)
   {
      for (i=0; i < M; i++)
      {
         #ifdef BETA0
            c00 = 0.0;
         #elif defined(BETA1)
            c00 = C[i+j*ldc];
         #else
            c00 = C[i+j*ldc] * beta;
         #endif
         for (k=0; k < K; k++) c00 += A[k+i*lda] * B[k+j*ldb];
         C[i+j*ldc] = c00;
      }
   }
}
\end{verbatim}

We then save this paragon of performance to 
{\tt ATLAS/tune/blas/gemm/CASES/ATL\_mm1x1x1.c}.  
From {\tt BLDdir/tune/blas/gemm/}, we can test that it
gets the right answer by:
\begin{verbatim}
   make mmutstcase pre=d nb=40 mmrout=CASES/ATL_mm1x1x1.c beta=0
   make mmutstcase pre=d nb=40 mmrout=CASES/ATL_mm1x1x1.c beta=1
   make mmutstcase pre=d nb=40 mmrout=CASES/ATL_mm1x1x1.c beta=7
\end{verbatim}

We pass four arguments to mmutstcase, a precision specifier
({\tt d} : double precision real; {\tt s} : single precision real; 
 {\tt z} : double precision complex; {\tt c} : single precision complex),
the size of the blocking parameter $N_B$, the beta value to test
(0, 1, and other), and finally, the filename to test.

If these messages pass the test, we can then see what kind of performance
we get by (this is the actual output on my 266Mhz PII):
\begin{verbatim}
make ummcase pre=d nb=40 mmrout=CASES/ATL_mm1x1x1.c beta=1
dNB=40, ldc=40, mu=4, nu=4, ku=1, lat=4: time=1.820000, mflop=53.731868
dNB=40, ldc=40, mu=4, nu=4, ku=1, lat=4: time=1.810000, mflop=54.028729
dNB=40, ldc=40, mu=4, nu=4, ku=1, lat=4: time=1.830000, mflop=53.438251
\end{verbatim}

This is the same timing repeated three times (this just tries to ensure
timings are repeatable), and the only output of real interest is the
MFLOP rate at the end.  The values the timer prints (mu, nu, ku, lat) are
all defaults because we didn't specify them; specifying them has no
effect when the timer is used in this way, so don't worry about them.

Now we can trivially improve the implementation by using the macro
constants in order to let the compiler unroll the loops:
\begin{verbatim}
void ATL_USERMM
   (const int M, const int N, const int K, const double alpha,
    const double *A, const int lda, const double *B, const int ldb,
    const double beta, double *C, const int ldc)
{
   int i, j, k;
   register double c00;

   for (j=0; j < NB; j++)
   {
      for (i=0; i < MB; i++)
      {
         #ifdef BETA0
            c00 = 0.0;
         #elif defined(BETA1)
            c00 = C[i+j*ldc];
         #else
            c00 = C[i+j*ldc] * beta;
         #endif
         for (k=0; k < KB; k++) c00 += A[k+i*KB] * B[k+j*KB];
         C[i+j*ldc] = c00;
      }
   }
}
\end{verbatim}

We save this to {\tt ATL\_mm1x1x1b.c}, and then time:
\begin{verbatim}
make ummcase pre=d nb=40 mmrout=CASES/ATL_mm1x1x1b.c beta=1
dNB=40, ldc=40, mu=4, nu=4, ku=1, lat=4: time=1.670000, mflop=58.558084
dNB=40, ldc=40, mu=4, nu=4, ku=1, lat=4: time=1.660000, mflop=58.910843
dNB=40, ldc=40, mu=4, nu=4, ku=1, lat=4: time=1.670000, mflop=58.558084
\end{verbatim}

OK, maybe a little explicit loop unrolling will make things work better:
\begin{verbatim}
void ATL_USERMM
   (const int M, const int N, const int K, const double alpha, const double *A, const int lda, const double *B, const int ldb, const double beta, double *C, const int ldc)
{
   int i, j, k;
   register double c00, c10, b0;
   const double *pA0, *pB=B;

#if ( (KB / 8)*8 != KB ) || (MB / 2)*2 != MB
   create syntax error!$@@&
#endif
   for (j=0; j < NB; j++, pB += KB)
   {
      pA0 = A;
      for (i=0; i < MB; i += 2, pA0 += KB2)
      {
         #ifdef BETA0
            c00 = c10 = 0.0;
         #elif defined(BETA1)
            c00 = C[i+j*ldc];
            c10 = C[i+1+j*ldc];
         #else
            c00 = beta*C[i+j*ldc];
            c10 = beta*C[i+1+j*ldc];
         #endif
         for (k=0; k < KB; k += 8)
         {
            b0 = pB[k];
            c00 += pA0[k] * b0;
            c10 += pA0[KB+k] * b0;
            b0 = pB[k+1];
            c00 += pA0[k+1] * b0;
            c10 += pA0[KB+k+1] * b0;
            b0 = pB[k+2];
            c00 += pA0[k+2] * b0;
            c10 += pA0[KB+k+2] * b0;
            b0   =  pB[k+3];
            c00 += pA0[k+3] * b0;
            c10 += pA0[KB+k+3] * b0;
            b0   =  pB[k+4];
            c00 += pA0[k+4] * b0;
            c10 += pA0[KB+k+4] * b0;
            b0   =  pB[k+5];
            c00 += pA0[k+5] * b0;
            c10 += pA0[KB+k+5] * b0;
            b0   =  pB[k+6];
            c00 += pA0[k+6] * b0;
            c10 += pA0[KB+k+6] * b0;
            b0   =  pB[k+7];
            c00 += pA0[k+7] * b0;
            c10 += pA0[KB+k+7] * b0;
         }
         C[i+j*ldc] = c00;
         C[i+1+j*ldc] = c10;
      }
   }
}
\end{verbatim}

And with this ode to beauty and elegance we get (after checking that it
still gets the right answer, of course):
\begin{verbatim}
make ummcase pre=d nb=40 mmrout=CASES/ATL_mm2x1x8.c beta=1
dNB=40, ldc=40, mu=4, nu=4, ku=1, lat=4: time=0.720000, mflop=135.822222
dNB=40, ldc=40, mu=4, nu=4, ku=1, lat=4: time=0.710000, mflop=137.735211
dNB=40, ldc=40, mu=4, nu=4, ku=1, lat=4: time=0.710000, mflop=137.735211
\end{verbatim}

Its interesting to see the effects of differing $\beta$ on the code:
\begin{verbatim}
make ummcase pre=d nb=40 mmrout=CASES/ATL_mm2x1x8a.c beta=0
dNB=40, ldc=40, mu=4, nu=4, ku=1, lat=4: time=0.700000, mflop=139.702857
dNB=40, ldc=40, mu=4, nu=4, ku=1, lat=4: time=0.700000, mflop=139.702857
dNB=40, ldc=40, mu=4, nu=4, ku=1, lat=4: time=0.700000, mflop=139.702857

make ummcase pre=d nb=40 mmrout=CASES/ATL_mm2x1x8a.c beta=7
dNB=40, ldc=40, mu=4, nu=4, ku=1, lat=4: time=0.720000, mflop=135.822222
dNB=40, ldc=40, mu=4, nu=4, ku=1, lat=4: time=0.730000, mflop=133.961644
dNB=40, ldc=40, mu=4, nu=4, ku=1, lat=4: time=0.720000, mflop=135.822222
\end{verbatim}

As well as differing blocking factors:
\begin{verbatim}
make ummcase pre=d mmrout=CASES/ATL_mm2x1x8a.c beta=1 nb=16
dNB=16, ldc=16, mu=4, nu=4, ku=1, lat=4: time=0.850000, mflop=115.112056
dNB=16, ldc=16, mu=4, nu=4, ku=1, lat=4: time=0.860000, mflop=113.773544
dNB=16, ldc=16, mu=4, nu=4, ku=1, lat=4: time=0.850000, mflop=115.112056

make ummcase pre=d mmrout=CASES/ATL_mm2x1x8a.c beta=1 nb=32
dNB=32, ldc=32, mu=4, nu=4, ku=1, lat=4: time=0.730000, mflop=134.034586
dNB=32, ldc=32, mu=4, nu=4, ku=1, lat=4: time=0.740000, mflop=132.223308
dNB=32, ldc=32, mu=4, nu=4, ku=1, lat=4: time=0.730000, mflop=134.034586

make ummcase pre=d mmrout=CASES/ATL_mm2x1x8a.c beta=1 nb=48
dNB=48, ldc=48, mu=4, nu=4, ku=1, lat=4: time=0.820000, mflop=119.223571
dNB=48, ldc=48, mu=4, nu=4, ku=1, lat=4: time=0.820000, mflop=119.223571
dNB=48, ldc=48, mu=4, nu=4, ku=1, lat=4: time=0.820000, mflop=119.223571
\end{verbatim}

If we wanted to have ATLAS try these crappy implementations during the
ATLAS search, we would have the following
{\tt ATLAS/tune/blas/gemm/CASES/dcases.dsc}:
\begin{verbatim}
<ID> <flag> <mb> <nb> <kb> <muladd> <lat> <mu> <nu> <ku> <rout> "<Contributer>"
3
1 0 0 0 0 1 1 1 1 1 ATL_mm1x1x1.c    "R. Clint Whaley"
2 0 1 1 1 1 1 1 1 1 ATL_mm1x1x1b.c   "R. Clint Whaley"
3 0 2 1 8 1 1 2 1 8 ATL_mm2x1x8a.c   "R. Clint Whaley"
\end{verbatim}

\subsection{More timing info}
So maybe you wonder how our big hand-tuned guy stacks up against the ATLAS
code generator?  When ATLAS completed its search on my PII, it stored its
best case in {\tt ATLAS/tune/blas/gemm/LINUX\_PII/res/dMMRES}:
\begin{verbatim}
speedy. cat res/dMMRES 
MULADD  LAT  NB  MU  NU  KU  FFTCH  IFTCH  NFTCH    MFLOP
     0    5  40   2   1  40      0      2      1   197.94
16
\end{verbatim}

We generate and time this case by:
\begin{verbatim}
make mmcase muladd=0 lat=5 nb=40 mu=2 nu=1 ku=40 beta=1
dNB=40, ldc=40, mu=2, nu=1, ku=40, lat=5: time=0.490000, mflop=199.575510
dNB=40, ldc=40, mu=2, nu=1, ku=40, lat=5: time=0.500000, mflop=195.584000
dNB=40, ldc=40, mu=2, nu=1, ku=40, lat=5: time=0.490000, mflop=199.575510
\end{verbatim}

We test that the generator isn't out of its mind by:
\begin{verbatim}
make mmtstcase muladd=0 lat=5 nb=40 mu=2 nu=1 ku=40 beta=1
\end{verbatim}

Note that when timing/testing the generator, varying the parameters such
as mu, nu, ku, beta, etc., generates different codes, and thus different
performance numbers:
\begin{verbatim}
make mmcase muladd=0 lat=4 nb=40 mu=2 nu=1 ku=4 beta=1
dNB=40, ldc=40, mu=2, nu=1, ku=4, lat=4: time=0.760000, mflop=128.673684
dNB=40, ldc=40, mu=2, nu=1, ku=4, lat=4: time=0.770000, mflop=127.002597
dNB=40, ldc=40, mu=2, nu=1, ku=4, lat=4: time=0.770000, mflop=127.002597

make mmcase muladd=0 lat=4 nb=40 mu=2 nu=1 ku=8 beta=1
dNB=40, ldc=40, mu=2, nu=1, ku=8, lat=4: time=0.640000, mflop=152.800000
dNB=40, ldc=40, mu=2, nu=1, ku=8, lat=4: time=0.630000, mflop=155.225397
dNB=40, ldc=40, mu=2, nu=1, ku=8, lat=4: time=0.630000, mflop=155.225397

make mmcase muladd=0 lat=4 nb=40 mu=2 nu=2 ku=40 beta=1
dNB=40, ldc=40, mu=2, nu=2, ku=40, lat=4: time=2.550000, mflop=38.349804
dNB=40, ldc=40, mu=2, nu=2, ku=40, lat=4: time=2.560000, mflop=38.200000
dNB=40, ldc=40, mu=2, nu=2, ku=40, lat=4: time=2.550000, mflop=38.349804
\end{verbatim}

\subsection{Complex {\it gemmK}}
Vainly hoping we were done, eh?  Nope, complex codes are special.  
ATLAS actually uses the real matrix multiply generator in order to
do complex multiplication.  It needs some tricks to do this, obviously.
The first thing to note is that if $X_r$ denotes the real elements of $X$,
and $X_i$ indicates the imaginary components, then complex matrix-matrix
multiply of the form $C \leftarrow A B + \beta C$ may be accomplished by 
the following four real matrix multiplies:
{\samepage
\begin{enumerate}
\item $C_r \leftarrow A_i B_i - \beta C_r$
\item $C_i \leftarrow A_i B_r + \beta C_i$
\item $C_r \leftarrow A_r B_r - C_r$
\item $C_i \leftarrow A_r B_i + C_i$
\end{enumerate}
}

This works fine assuming $\beta$ is real, otherwise $\beta$ must be applied
explicitly as a complex scalar, and then set $\beta=1$ in the above outline.

Therefore, in order to use this trick, upon copying $A$ and $B$ to block-major
storage, ATLAS also splits the arrays into real and imaginary components.
The only matrix not expressed as two real matrices is then $C$, and to
fix this problem, ATLAS demands that the complex {\it gemmK} stride $C$ by
2.  An example will solidify the confusion.

A simple 3-loop implementation of an ATLAS complex {\it gemmK} is:
\begin{verbatim}
void ATL_USERMM
   (const int M, const int N, const int K, const double alpha,
    const double *A, const int lda, const double *B, const int ldb,
    const double beta, double *C, const int ldc)
{
   int i, j, k;
   register double c00;

   for (j=0; j < N; j++)
   {
      for (i=0; i < M; i++)
      {
         #ifdef BETA0
            c00 = 0.0;
         #else
            c00 = C[2*(i+j*ldc)];
            #ifdef BETAX
               c00 *= beta;
            #endif
         #endif
         for (k=0; k < K; k++) c00 += A[k+i*lda] * B[k+j*ldb];
         C[2*(i+j*ldc)] = c00;
      }
   }
}
\end{verbatim}

First we test that it produces the right answer:
\begin{verbatim}
make cmmutstcase pre=z nb=40 mmrout=CASES/ATL_cmm1x1x1.c beta=0
make cmmutstcase pre=z nb=40 mmrout=CASES/ATL_cmm1x1x1.c beta=1
make cmmutstcase pre=z nb=40 mmrout=CASES/ATL_cmm1x1x1.c beta=8
\end{verbatim}

Then we scope the awesome performance:
\begin{verbatim}
make cmmucase pre=z nb=40 mmrout=CASES/ATL_cmm1x1x1.c beta=1
zNB=40, ldc=40, mu=4, nu=4, ku=1, lat=4: time=1.830000, mflop=53.438251
zNB=40, ldc=40, mu=4, nu=4, ku=1, lat=4: time=1.830000, mflop=53.438251
zNB=40, ldc=40, mu=4, nu=4, ku=1, lat=4: time=1.830000, mflop=53.438251
\end{verbatim}

Now, since we are clearly gluttons for punishment, we compare our
masterwork to ATLAS's generated kernel:
\begin{verbatim}
speedy. cat res/zMMRES 
MULADD  LAT  NB  MU  NU  KU  FFTCH  IFTCH  NFTCH    MFLOP
     0    5  36   2   1  36      0      2      1   186.42
     10

speedy. make mmcase muladd=0 lat=5 nb=36 mu=2 nu=1 ku=36 beta=1
dNB=36, ldc=36, mu=2, nu=1, ku=36, lat=5: time=0.500000, mflop=195.581952
dNB=36, ldc=36, mu=2, nu=1, ku=36, lat=5: time=0.490000, mflop=199.573420
dNB=36, ldc=36, mu=2, nu=1, ku=36, lat=5: time=0.490000, mflop=199.573420
\end{verbatim}

\subsection{What to do if you are writing in assembler}
If your kernel is written in gas assembler, you can tell the tester
and timer that by setting the appropriate compiler and flag macro on
the command line.  For single precision types, these macros for {\it gemmK}
are called {\tt SMC} and {\tt SMCFLAGS}, respectively, and they are
{\tt DMC} and {\tt DMCFLAGS} for double precision routines.
For instance, to test a {\tt DGEMM} code written in assembler, requiring
a 16 blocking factor, you'd issue:
\begin{verbatim}
   make ummcase pre=d mmrout=CASES/myassembler.c nb=16 \
        DMC=gcc DMCFLAGS="-x assembler-with-cpp"
\end{verbatim}

\subsection{Providing ATLAS with kernel cleanup code}
As mentioned in Section~\ref{sec-buildMM}, when any problem dimension
(eg., {\tt M}, {\tt N}, or {\tt K}) is not a multiple of $N_B$, ATLAS
must call cleanup code to handle the remainder.  When the user-contributed
kernel is only modestly faster than ATLAS's generated kernel, letting
the generated code handle cleanup will probably be an adequate solution.
However, when the user-contributed kernel is much faster than the generated
code, using the generated cleanup may represent a significant performance
drop for many problem sizes (see Section~\ref{sec-cleancost} for an
analysis of the cost of cleanup), and thus it becomes necessary for the user
to supply ATLAS with cleanup code as well.  In order to understand how
this is done, it is necessary to discuss how ATLAS does cleanup.

\subsubsection{ATLAS and cleanup}
As we have seen, ATLAS's main performance kernel is an L1 cache contained
matmul with fixed dimension $N_B$, and when a given problem dimension of
the general matmul is not a multiple of $N_B$, cleanup code must be
called.  It should be apparent that generating codes with all dimensions
fixed at compile time, as we do with the full kernel, is not a good idea
for cleanup, since it would result in roughly ${N_B}^3$ cleanup routines.
Not only would this make the average executable huge, but it would also
probably result in performance degredation due to constant instruction load.

ATLAS therefore normally generates a variable number of cleanup cases,
with the number of generated codes minimally being $7$, and the maximum
number being $6 + N_B$.  The number of generated codes can vary because
the $K$ cleanup routines are special, sometimes requiring $N_B$ different
codes to handle efficiently, as we will see below.

ATLAS splits the generated cleanup into these categories
\begin{enumerate}
\item {\bf M-cleanup} $M < N_B$ \&\& $N = K = N_B$:
   3 routines, corresponding to {\tt BETA} = 0, 1 and arbitrary
\item {\bf N-cleanup} $N < N_B$ \&\& $M = K = N_B$:
   3 routines, corresponding to {\tt BETA} = 0, 1 and arbitrary
\item {\bf K-cleanup} $K <= N_B$ \&\& $M \leq N_B$ \&\& $N \leq N_B$:
   Only one {\tt BETA} case (arbitrary), but may compile special case for
   each possible $K$ value, resulting in at least 1, and at most $N_B$
   K-cleanup routines
\end{enumerate}

So we see that K-cleanup is special in several ways.  First, it is the
most general cleanup routine, since it can handle multiple dimensions not
being less than $N_B$, whereas the M- and N-cleanup routines can only have
their respective dimensions less then $N_B$.  The second thing to note is
that we compile only the most general {\tt BETA} case for K-cleanup; this
is due to the fact that we may need $N_B$ different routines to handle
K-cleanup efficiently, and multiplying this number of routines by three seems
counterproductive.

The final difference in the K-cleanup is the fact that it potentially requires
$N_B$ different routines to support.  This is due to several factors.
Firstly, in ATLAS, the innermost loop in gemm is the K-loop, making it
very important for performance.  On systems without good loop handling,
such as the x86, heavy K unrollings are critical.
Secondly, the leading dimensions of the $A$ and $B$ matrices are fixed
to {\tt KB} due to the data copy, which allows for more efficient indexing
of these matrices.  If a routine takes run-time $K$ (rather than compile-time,
as when the dimension is fixed to {\tt KB}), it must also take run-time 
{\tt lda} and {\tt ldb}, and this extra indexing is too costly on
many architectures.

\subsubsection{User supplied cleanup}
Users can supply cleanup code for the following three cases only, all of
which come in the three {\tt BETA} variants:
\begin{enumerate}
\item {\bf M-cleanup} $M < N_B$ \&\& $N = K = N_B$
\item {\bf N-cleanup} $N < N_B$ \&\& $M = K = N_B$
\item {\bf K-cleanup} $K < N_B$ \&\& $M = N = N_B$
\end{enumerate}

The generated code handles all cleanup where more
than one dimension is less than the blocking factor.  This simplification
allows ATLAS to avoid having to test ${N_B}^3$ cases when selecting user
cleanup.  Once the matrices in question are larger than $N_B$, cleanup
with more than one dimension less than $N_B$ rapidly stops being a 
performance factor.  Small matrices where this cleanup is a factor are
almost certainly going to be handled by ATLAS's small-case code anyway,
so it seems unlikely that this simplification will hurt performance in
practice.  Section~\ref{sec-cleancost} shows this in a more formal way.

Users need to be very careful when supplying cleanup, because if the user
indicates that a dimension must be a compile-time variable, rather than
a runtime variable, ATLAS will generate up to $N_B$ routines to handle
user cleanup, and since user routines are compiled with all {\tt BETA}
variants, it is possible to generate $9 N_B$ cleanup cases, in addition
to ATLAS's generated cases.  It is therefore recommended that the user
supply cleanup that uses run-time arguments whenever possible, and indicate
kernels taking compile-time dimensions as not to be used for cleanup.

\subsubsection{Indicating cleanup in the index file}
Any routine that does not throw the flag value of 8 will be evaluated by
the install as a cleanup possibility.  Flag values are very important
for indicating opportunities for cleanup.  Here is an example from the
release:
\begin{verbatim}
  1 480 4 4 1 1 1 4 4 2 ATL_mm4x4x2US.c "V. Nguyen & P. Strazdins"
\end{verbatim}

OK, as always, we can read this to see that {\tt MB} and {\tt NB}
must be multiples of 4, and that {\tt KB} can be any value.  With
no flag modifiers, if we wanted to use the routine for K cleanup,
we would have to compile it into $N_B$ different routines, since
loop dimensions are compile-time parameters by default.  However,
this routine is modified by a flag value of 480.  What does this mean?
Consulting table~\ref{tab-mmflag}, we see that $32 + 64 + 128 + 256 = 480$,
which means lda and ldb are not restricted to {\tt KB} (i.e., they are
run-time parameters to the routine), the M-loop is controlled by a run-time
variable, the N-loop is controlled by a run-time variable, and the K-loop
is controlled by a run-time variable.  We therefore know that we can
use this routine for all cleanups (M-, N-, and K-cleanup), and we need only
one routine to do so (i.e., we do not have to compile $N_B$ routines to handle
all cases).  However, it can only be used for M- and N- cleanup cases where
the respective dimension is a multiple of 4.  Therefore, assuming this
kernel is superior to the generated code, it will be used for all K cleanup
routines.  However, for M and N cleanup, there will be something corresponding
to the following pseudocode:
\begin{verbatim}
   if (M % 4 == 0) call ATL_mm4x4x2US
   else call generated M cleanup
\end{verbatim}


It is clear that without overloading the flag value to an even more
ludicrous degree, that cleanup will eventually need to have it's own
index file.  For instance, it would be nice to be able to insist that
a particular K-cleanup code be used only when $K > 3$, for instance,
in addition to insisting it be a multiple of a particular value.  The fact
that cleanup does not already have such a seperate file simply represents
a design failure on my part; it was not until I had already produced the
system working as it does now that I saw its shortcomings, and then it
was too late to change for the release.  Subsequent developer releases
will probably address this shortcoming.

\subsubsection{Testing and timing cleanup}
Cleanup is tested in the same way as the normal kernel, but you need to
supply additional parameters.  {\tt M}, {\tt N}, and {\tt K} are the problem
dimensions, and {\tt MB}, {\tt NB}, {\tt KB} are the blocking factors.
If the blocking factors are set to zero, that means they are run-time
parameters to the routine.  {\tt lda}, {\tt ldb}, {\tt ldc} are the leading
dimensions of the operand arrays, and they default to {\tt KB}, {\tt KB},
and zero, respectively.

Here is an example of testing an M-cleanup routine, insisting that
M is a run-time argument:
\begin{verbatim}
make mmutstcase mmrout=CASES/ATL_mm1x1x1.c pre=d M=17 N=40 K=40 \
     mb=0 nb=40 kb=40
\end{verbatim}

Here is timing the same routine, but insisting that the M-loop is fixed at
compile time:
\begin{verbatim}
make ummcase mmrout=CASES/ATL_mm1x1x1.c pre=d M=17 N=40 K=40 \
     mb=17 nb=40 kb=40
\end{verbatim}

Here's testing a K-cleanup routine, taking run-time K and leading dimensions:
\begin{verbatim}
make mmutstcase mmrout=CASES/ATL_mm1x1x1.c pre=d M=40 N=40 K=27 \
     mb=40 nb=40 kb=0 lda=0 ldb=0
\end{verbatim}

The same test taking compile-time K and leading dimensions:
\begin{verbatim}
make mmutstcase mmrout=CASES/ATL_mm1x1x1.c pre=d M=40 N=40 K=27 \
     mb=40 nb=40 kb=27 lda=27 ldb=27
\end{verbatim}

\subsubsection{Importance of cleanup}
\label{sec-cleancost}
In analyzing the importance of good cleanup for performance, it is necessary
to recognize the various types that can occur.  The cleanup that user's can
supply to ATLAS is {\em one dimensional cleanup}, i.e., only one of the three
possible dimensions is less than $N_B$.  There is also 2 and 3 dimensional
cleanup.  To give an idea of the relative importance of various catagories
of computation, it is roughly true that the matmul kernel is a cubic cost,
the one dimensional cleanup is a square cost, the two dimensional cleanup
is a linear cost, and the three dimensional cleanup is $O(1)$.

This is shown more formally below.  Define $M_r = M$ mod $N_B$, let 
$m, n, k$ be the dimensional arguments to the {\it gemmK} and/or cleanup,
and remember that matrix multiplication takes $2 M N K$ flops, and we see
that the flop count for each catagory is:
\begin{itemize}
\item {$\bf m=n=k=N_B$}: 
   $\Wfloor{\frac{M}{N_B}} \Wfloor{\frac{N}{N_B}} \Wfloor{\frac{K}{N_B}} 
   2 {N_B}^3 \Longrightarrow \Wfloor{\frac{N}{N_B}}^3 ~2 {N_B}^3$

\item $\bf m < N_B, n = k = N_B$:
   $\Wfloor{\frac{N}{N_B}} \Wfloor{\frac{K}{N_B}} 2 M_r {N_B}^2
    \Longrightarrow \Wfloor{\frac{N}{N_B}}^2 ~2 N_r {N_B}^2$
\item $\bf n < N_B, m = k = N_B$:
   $\Wfloor{\frac{M}{N_B}} \Wfloor{\frac{K}{N_B}} 2 N_r {N_B}^2
    \Longrightarrow \Wfloor{\frac{N}{N_B}}^2 ~2 N_r {N_B}^2$
\item $\bf k < N_B, m = n = N_B$:
   $\Wfloor{\frac{M}{N_B}} \Wfloor{\frac{N}{N_B}} 2 K_r {N_B}^2
    \Longrightarrow \Wfloor{\frac{N}{N_B}}^2 ~2 N_r {N_B}^2$

\item $\bf m < N_B, n < N_B, k = N_B$:
   $\Wfloor{\frac{K}{N_B}} 2 M_r N_r {N_B} 
    \Longrightarrow \Wfloor{\frac{N}{N_B}} 2 {N_r}^2 N_B$
\item $\bf m < N_B, k < N_B, n = N_B$:
   $\Wfloor{\frac{N}{N_B}} 2 M_r K_r {N_B}
    \Longrightarrow \Wfloor{\frac{N}{N_B}} 2 {N_r}^2 N_B$
\item $\bf n < N_B, k < N_B, m = N_B$:
   $\Wfloor{\frac{M}{N_B}} 2 N_r K_r {N_B}
    \Longrightarrow \Wfloor{\frac{N}{N_B}} 2 {N_r}^2 N_B$
\item $\bf m < N_B, n < N_B, k < N_B$:
   $2 M_r N_r K_r \Longrightarrow 2 {N_r}^3$
\end{itemize}

Note that the simplified equations to the right of $\Longrightarrow$ 
assume the square case, i.e. $M = K = N$.  The above analysis can now
be grouped into the catagories of interest as in:
\begin{itemize}
  \item {\bf kernel} : $\Wfloor{\frac{N}{N_B}}^3 ~2 {N_B}^3$
  \item {\bf 1D cleanup}: $3 \Wfloor{\frac{N}{N_B}}^2 ~2 N_r {N_B}^2
     ~~~~<~~~~ 3  \Wfloor{\frac{N}{N_B}}^2 ~2 {N_B}^3$
  \item {\bf 2D cleanup}: $\Wfloor{\frac{N}{N_B}} 2 {N_r}^2 N_B
     ~~~~~~~~<~~~~~ 3 \Wfloor{\frac{N}{N_B}} ~2 {N_B}^3$
  \item {\bf 3D cleanup}: $2 {N_r}^3 ~~~~~~~~~~~~~~~~~~~<~~~~~ 2 {N_B}^3$
\end{itemize}
The simplified equations to the right of the $<$ above provide a safe
upper bound on cleanup cost by setting $N_r = N_B$
(in reality, $0 < N_r < N_B$, of course).

With this analysis, we can easily see why it is not important for the
user to be able to contribute 2D and 3D cleanup cases: remember that
all of these kernels are for ATLAS's {\em large-case} gemm.  ATLAS has a 
seperate small-case gemm, which is invoked when the problem is so small
that the $2 N^2$ copy cost is significant compared to the $2 N^3$ computational
costs.  So, in the cases where the $O(N)$ 2D cleanup or $O(1)$ 3D cleanup
costs are prohibitive, this large-case gemm will probably not be used anyway.

\subsection{{\it gemmK} usage notes}
The assumptions behind this kernel are that the input operands are
loaded to L1 only one time (i.e., the blocking guarantees that all
of the matrix accessed in the inner loop plus the active panel of
the matrix in the outer loop fits in L1).  For very large caches, all
three operands may fit into cache, but this is typically not the
case.  Because this {\it gemmK} is called by routines that place $K$
as the innermost loop, the output operand $C$ will typically come
from the L2 cache (except, obviously, on the first of the 
$\frac{K}{N_B}$ such calls).  ATLAS uses the JIK loop variant of
on-chip multiply, and thus all of $A$ fits in cache, with nu columns
of $B$.  To take an example, say you are using mu = nu = 4, with
$N_B = 40$, then the idea is that the $40 \times 40$ piece of $A$,
along with the $40 \times 4$ piece of $B$ (the active panel of $B$),
and the $4 \times 4$ section of $C$ all fit into cache at once, with
enough room for the load of the next step, and any junk the algorithm
might have in L1.  That panel of $B$ is applied to all of $A$, and then
a new panel is loaded.  Since the panel has been applied to all $A$, it
will never be reloaded, and thus we see that $B$ is loaded to L1 only
one time.  Since all of $A$ fits in L1, and we keep it there across all
panels of $B$, it is also loaded to L1 only one time. 

If written appropriately, loading all of $B$ with a few rows
of $A$ should theoretically be just as efficient (i.e., the IJK variant
of matmul).  However, the variants where $K$ is not the innermost loop
are unlikely to work well in ATLAS, if for no other reason than the 
transpose settings we have chosen militate against it.

Note that the $\beta = 0$ case must not read $C$, since the memory may
legally be unitialized.

\subsection{Getting ATLAS to use your kernel}
OK, so let's say you've got a kernel that is faster than what ATLAS is
presently using, how do you get ATLAS to use it?  First, of course, you
put the source in the CASES directory, and update the appropriate
{\tt <pre>cases} index file.  Then, you take different steps depending on
how you wish to do the install, as discussed in the following sections.
In all of these discussions, {\tt <pre>} is replaced by your type/precision
modifier (one of {\tt s}, {\tt d}, {\tt c}, {\tt z}).

\subsubsection{Putting it in by hand}
In an already-installed ATLAS, you can make ATLAS reinstall just the
kernel.  From your {\tt BLDdir/tune/blas/gemm/} directory, issue
these commands:
\begin{verbatim}
   rm res/<pre>u*
   rm res/<pre>MMRES
   ./xmmsearch -p <pre>
   make <pre>install
\end{verbatim}

\subsubsection{With a fresh install}
\label{sec-archdefModify}
First, run config as usual.  Then, tail the created {\tt Make.inc}
file, and see if the macro {\tt INSTFLAGS} includes {\tt -a 1}.  If so,
ATLAS has some architectural defaults for your architecture (though
perhaps not for your compiler, if you have forced the use of a non-default
compiler), which won't include
your shiny new kernel.  So, you will need to remove the files indicating
the default {\tt gemmK} kernel for your precision.  To do this, scope
your {\tt ARCH} setting in your {\tt Make.inc}.  For the purposes of
this discussion, let us say it set to {\tt Core2Duo64SSE3} (i.e., in
the below example, substitute the definition of {\tt ARCH} for
{\tt Core2Duo64SSE3}).  Go to {\tt ATLAS/CONFIG/ARCHS}, and issue
the following commands:
%{\tt ARCHDEF} is set.  If so, cd to the
%directory pointed at by {\tt ARCHDEF}, and then issue
\begin{verbatim}
   gunzip -c Core2Duo64SSE3.tgz | tar xvf -
   rm Core2Duo64SSE3/gemm/gcc/<pre>u* 
   rm Core2Duo64SSE3/gemm/gcc/<pre>MMRES
   rm Core2Duo64SSE3.tgz
   tar cvf Core2Duo64SSE3.tar Core2Duo64SSE3
   gzip Core2Duo64SSE3.tar
   mv Core2Duo64SSE3.tar.gz Core2Duo64SSE3.tgz
\end{verbatim}

Now, continue install as normal, and your kernel should be used if it beats
what ATLAS is presently using.  Note that this assumes you are using {\tt gcc}
as the {\tt gemmK} compiler, which is the default on most systems.  If
you are using a different compiler, you would substitute its name instead
of {\tt gcc} in the above lines.  If there is no subdirectory with the
name of your compiler in the tarfile, ATLAS has no architectural defaults
for that compiler, and thus you need to make no changes to the tarfile.

\subsubsection{With an old install, but using full install command}
When rebuilding an old install, the main trap is to avoid having
architectural defaults make it so you don't time your new kernel.
Follow the instructions given in Section~\ref{sec-archdefModify},
but additionally make sure you delete any prexisting directory
that matches your {\tt ARCH} definiton.  Therefore, in the above
example, in the {\tt ATLAS/CONFIG/ARCHS} directory, you would
additionally issue:
\begin{verbatim}
   rm -rf Core2Duo64SSE3
\end{verbatim}
If such a subdirectory existed.

From your {\tt BLDdir} directory, then issue:
\begin{verbatim}
   rm bin/INSTALL_LOG/*
   rm tune/blas/gemm/res/<pre>u*
   rm tune/blas/gemm/res/<pre>MMRES
   make build
\end{verbatim}

\subsection{Contributing a complete GEMM implementation}
\textbf{
This feature has been temporarily disabled in 3.8, though it may be
re-enabled in the 3.9 series if there is user demand.  This section
is therefore kept around solely historical purposes, and will need
to be updated if the feature is added back in.
}

Contributing an L1 kernel is the prefered method of user contribution
for Level 3 BLAS speedup, but it is not the only one supported by ATLAS.
ATLAS also allows a user to contribute a complete system-specific
GEMM implementation.  This method of contribution is far less desirable
than kernel contribution, and thus the standards of acceptance are
correspondingly higher.

When only a kernel is contributed, it is only used when timings indicate
it is superior to the best ATLAS-supplied routine for a given architecture.
Because kernel routines are called in known ways by the ATLAS infrastructure,
the timer can be made to accurately reflect typical usage.  A full GEMM,
which is to all intents called directly by the user, has no ``typical''
usage, and the timer is thus not able to ensure that the user's full
GEMM is superior to that supplied by ATLAS in a system-independent way,
even if the additional installation time required to choose amoung full
GEMM implementations were allowed.  Thus, full GEMM implementations will
be used only when ATLAS's configuration detects a known architecture where
the ATLAS team has certified the full GEMM to be significantly better than
ATLAS's native GEMM, across the entire spectrum of problem shapes and sizes
(with the exception of those shapes and sizes handled by ATLAS's non-copy
code, as explained below).

As explained in Section~\ref{sec-buildMM}, ATLAS has both a small-case
matmul, which does not copy the user's input operands, and a large-case
code that does.  The user contributed GEMM replaces ATLAS's large-case
GEMM, and then timings are used as normal to determine the crossover points at
which the contributed GEMM outperforms ATLAS's small-case code.

\subsubsection{Supplying ATLAS with what it needs}
ATLAS expects that the contributed GEMM will have its own architecture-
specific subdirectory, just as with all other ATLAS source directories.
That directory is indicated to ATLAS by the {\tt UMMdir} macro set in
{\tt Make.inc}.  For instance, on the alpha platform, Mr. Goto's GEMM
is used by ATLAS, and {UMMdir} is therefore set to:
{\tt \$(TOPdir)/src/blas/gemm/GOTO/\$(ARCH)}.

In this directory, there must be a master makefile, called {\tt Makefile},
which minimally contains the following targets:
\begin{itemize}
\item {\tt susermm} : builds the single precision real gemm with
all its dependencies
\item {\tt dusermm} : builds the double precision real gemm with
all its dependencies
\item {\tt cusermm} : builds the single precision complex gemm with
all its dependencies
\item {\tt zusermm} : builds the double precision complex gemm with
all its dependencies
\item {\tt sclean} : deletes all non-library files created by {\tt susermm}
\item {\tt dclean} : deletes all non-library files created by {\tt dusermm}
\item {\tt cclean} : deletes all non-library files created by {\tt cusermm}
\item {\tt zclean} : deletes all non-library files created by {\tt zusermm}
\end{itemize}

For each precision, ATLAS calls the user's GEMM using this API:
\begin{verbatim}
int ATL_U<pre>usergemm(const enum ATLAS_TRANS TA, const enum ATLAS_TRANS TB,
                       const int M, const int N, const int K,
                       const SCALAR alpha, const TYPE *A, const int lda,
                       const TYPE *B, const int ldb, const SCALAR beta,
                       TYPE *C, const int ldc)
\end{verbatim}
where,\\
\begin{tabular}{||l|l|l|l|l||}\hline\hline
\verb+<pre>+ : & {\tt s} & {\tt d} & {\tt c} & {\tt z} \\\hline\hline
SCALAR & {\tt float} & {\tt double} & {\tt float*} & {\tt float*} \\\hline
TYPE   & {\tt float} & {\tt double} & {\tt float} & {\tt float} \\\hline\hline
\end{tabular}\\\\

This routine should return {\tt 0} upon successful invocation, and 
{\tt -1} if unable to malloc enough memory.  Other errors may be
signaled by returning a value of {\tt 2}.  On error in this routine,
ATLAS will call the no-copy code to get the answer, so a return value
of {\tt 1} indicates that ATLAS should do this.  If a fatal error
occurs, or if an error occurs after operands have been modified (i.e.,
calling the no-copy code will no longer produce the correct answer),
then execution should be halted.

ATLAS's interface routines have already done all required error checking,
so the user need not check the input arguments in this routine, or any
of the lower-level user contributed routines.

\subsubsection{What to do if you don't supply all precisions}

Remember that what ATLAS is doing is substituting your GEMM for its own
large-case GEMM.  However, ATLAS's large-case GEMM is still compiled in
the library, it is just not being used.  The following code will call
ATLAS's large case code for all precisions:
\begin{verbatim}
#include "atlas_misc.h"
#include "atlas_lvl3.h"

int Mjoin(Mjoin(ATL_U,PRE),usergemm)
   (const enum ATLAS_TRANS TA, const enum ATLAS_TRANS TB,
    const int M, const int N, const int K, const SCALAR alpha,
    const TYPE *A, const int lda, const TYPE *B, const int ldb,
    const SCALAR beta, TYPE *C, const int ldc)
{
   int ierr;

   if (N >= M)
   {
      if ( ierr = Mjoin(PATL,mmJIK)(TA, TB, M, N, K, alpha, A, lda, B, ldb,
                                    beta, C, ldc) )
         ierr = Mjoin(PATL,mmIJK)(TA, TB, M, N, K, alpha, A, lda, B, ldb,
                                  beta, C, ldc);
   }
   else
   {
      if ( ierr = Mjoin(PATL,mmIJK)(TA, TB, M, N, K, alpha, A, lda, B, ldb,
                                    beta, C, ldc) )
         ierr = Mjoin(PATL,mmJIK)(TA, TB, M, N, K, alpha, A, lda, B, ldb,
                                  beta, C, ldc);
   }
   return (ierr);
}
\end{verbatim}

So, you can use the above code to have ATLAS call it's normal routines
for those precisions you do not supply, and call your own otherwise.
In order to help understand this process, ATLAS includes a 
``user-supplied'' GEMM that simply calls ATLAS's own GEMM for all
precisions, in {\tt ATLAS/src/blas/gemm/UMMEXAMPLE}.  If, for
instance, you have a single precision real GEMM but nothing else,
you can take this directory as a starting point, adding your
own commands for single precision real in the Makefile, etc,
and leaving everything else alone.

In order to do this, you should do the following:
%(where {\tt <arch>} is
%replaced by the {\tt ARCH} string of your {\tt Make.<arch>}; eg, 
%{\tt Linux\_PII}, {\tt AIX\_POWER3}, {\tt OSF\_21264}, etc.):
\begin{enumerate}
\item In {\tt ATLAS/src/blas/gemm/UMMEXAMPLE}:
   \begin{itemize}
   \item {\tt mkdir <arch>}
   \item {\tt cp Makefile <arch>/.}
   \item {\tt ln -s ../../../../../Make.<arch> Make.inc}
   \item Modify {\tt <arch>/Makefile} to compile your single precision routine
   \item Follow the instructions given in Section~\ref{sec-force},
which discusses how to make ATLAS use a contributed GEMM that config
has not setup for you
   \end{itemize}
\end{enumerate}


\subsubsection{Forcing ATLAS to use your GEMM}
\label{sec-force}

If ATLAS detects you are on a platform where a contributed full GEMM is
superior to ATLAS's large-case GEMM, ATLAS will automatically handle
the details of making ATLAS call the user-contributed routine.  If,
however, you wish to force ATLAS to use your GEMM (for instance, you
are testing your code before contribution, or just want to utilize ATLAS
for complete BLAS coverage with your GEMM), you should take
the following steps, after creating the appropriate subdirectory and
API as previously described:
\begin{enumerate}
\item Edit your {\tt Make.inc} file: 
 \begin{itemize}
 \item Change {\tt UMMdir} to point to your full GEMM's architecture
       subdirectory
 \item Add {\tt -DUSERGEMM} to the {\tt CDEFS} macro.
 \end{itemize}
\item In {\tt ATLAS/src/blas/gemm}, touch {\tt ATL\_gemmXX.c} and
{\tt ATL\_AgemmXX.c} to force recompilation
\item In {\tt BLDdir/src/blas/gemm/} type {\tt make lib}
\item In {\tt BLDdir/include/}, issue 
  \begin{itemize}
   \item {\tt rm ?Xover.h atlas\_cacheedge.h}
   \item {\tt touch altas\_cacheedge.h sXover.h dXover.h cXover.h zXover.h}
  \end{itemize}
\item In {\tt BLDdir/tune/blas/gemm/}, issue:
  \begin{itemize}
  \item {\tt rm res/?Xover.h res/atlas\_cacheedge.h}
  \item {\tt make res/atlas\_cacheedge.h}
  \item {\tt make res/sXover.h pre=s}
  \item {\tt make res/dXover.h pre=d}
  \item {\tt make res/cXover.h pre=c}
  \item {\tt make res/zXover.h pre=z}
  \end{itemize}
\end{enumerate}


\section{Speeding up the Level 2 BLAS}

ATLAS presently empirically tunes four kernels to optimize the various
L2BLAS, as shown in Table~\ref{tab-l2k}.
This table shows matvec used to tune SYMV and HEMV, and this is mostly
true.  For essentially any modern x86 or ARM, GEMV will be used speed up
HEMV and SYMV, but on other systems, ATLAS will simply use the reference
implementation, which may be faster.  The problem is that to support
HEMV/SYMV optimally, we need a kernel which we have not yet
empirically tuned.  You can build SYMV and HEMV out of GEMV, but in
doing so you bring the matrix $A$ into registers twice (once from
memory, and once from cache).  Since the load of $A$ is the dominant
cost in these operations, that is not good news.  However, for places
where vector instructions speed up memory access (modern x86) and where
the compiler can't do a great job (ARM), using a tuned GEMV in this
fashion is still faster than just calling a reference version.  So,
for most systems, speeding up the GEMV kernels will cause a corresponding
speedup in SYMV and GEMV, but these operations are the least well-tuned
BLAS that ATLAS supports (as a percentage of achievable peak, not raw
MFLOP, of course).  All other L2BLAS operations should be well optimized,
particularly for large problems.


\begin{table}[thb]
\begin{center}
\begin{tabular}{lll}
Mneum & operations & Used to support \\\hline\hline
\kernk{mvn} & $y \leftarrow Ax$, $y \leftarrow Ax + y$ & GEMV, TRMV, TRSV, 
HEMV, SYMV\\\hline
\kernk{mvt} & $y \leftarrow A^Tx$, $y \leftarrow A^Tx + y$ & GEMV, TRMV, TRSV,
HEMV SYMV\\\hline
\kernk{ger} & $A \leftarrow xy^T + A$  & GER, GERU, GERC, SYR, HER \\\hline
\kernk{ger2}& $A \leftarrow xy^T + wz^T + A$, & GER2, GER2U, GER2C, SYR2, HER2
\\\hline
\end{tabular}
\end{center}
\caption{Level-2 BLAS kernels, and the L2BLAS routines they improve}
\label{tab-l2k}
\end{table}

In this section, we use some macros that are automatically defined
by the ATLAS build system.  The first is {\tt ATL\_CINT} which is
presently an alias for {\tt const int}.  The macros {\tt SCALAR}
and {TYPE} are defined according the precision being being compiled: \\
\begin{tabular}{||l|l|l|l|l||}\hline\hline
\verb+<pre>+ : & {\tt s} & {\tt d} & {\tt c} & {\tt z} \\\hline\hline
SCALAR & {\tt float} & {\tt double} & {\tt float*} & {\tt float*} \\\hline
TYPE   & {\tt float} & {\tt double} & {\tt float} & {\tt float} \\\hline\hline
\end{tabular}

\subsection{Testing and timing \kernk{mvn}}
The API of this routine is given by:
\begin{verbatim}
void ATL_UGEMV(ATL_CINT M, ATL_CINT N, const TYPE *A, ATL_CINT lda,
               const TYPE *X, TYPE *Y)
\end{verbatim}
If the routine is compiled with the macro {\tt BETA0} defined, then
it should perform the operation $y \leftarrow Ax$; if this macro is
not defined, then it should perform $y \leftarrow Ax + y$.
$A$ is a column-major $lda \times N$ contiguous array where $lda \ge M$.
$M$ specifies both the number of rows of $A$ and the length of the vector $X$.
$N$ provides the number of columns in $A$, and the length of the vector $Y$.

Since this is a $O(MN)$ operation, and A is $M \times N$ in size,
this algorithm is dominated by the load of $A$.  No reuse of $A$ is
possible, so the best we can do is reuse the vectors (through operations
like register and cache blocking).

Each element of $Y$ can obviously be computed by doing a dot product of
the corresponding row of $A$ with $X$.  However, this accesses the
matrix along the non-contiguous dimension, which leads to terrible memory
heirarchy usage for any reasonably sized matrix.  Therefore, \kernk{mvn}
cannot be written in this fashion.

In order to optimize the access of $A$ (dominant algorithmic cost) we must
access it in a column-major fashion.  Therefore, instead of writing it as
a series of dot products ({\tt ddot}s), we can write it as a series of 
{\tt axpy}s (recall that {\tt axpy} does the operation 
{\tt $y \leftarrow \alpha x + y$}).

Therefore, instead of being composed of a series of {\tt ddot}s, 
\kernk{mvn} should be implemented as a series of {\tt axpy}s.  In this
formulation, we access columns of $A$ (the contiguous dimension) at a time,
and we write update all elements of $Y$ 

A simple implementation of this kernel (for real precisions) is:
\begin{verbatim}
#include "atlas_misc.h"  /* define TYPE macros */
void ATL_UGEMV(ATL_CINT M, ATL_CINT N, const TYPE *A, ATL_CINT lda,
               const TYPE *X, TYPE *Y)
{
   register int j, i;

/*
 * Peel 1st iteration of N-loop if BETA0 is defined
 */
   #ifdef BETA0
   {
      const register TYPE x0 = *X;
      for (i=0; i != M; i++)
         Y[i] = A[i] * x0;
      j=1;
      A += lda;   /* done with this column of A */
   }
   #else
      j=0;
   #endif
   for (; j != N; j++)
   {
      const TYPE register x0 = X[j];
      register TYPE y0 = Y[i];

      for (i=0; i != M; i++) 
         Y[i] += A[i] * x0;
      A += lda;   /* done with this column of A */
   }
}
\end{verbatim}

To time and test \kernk{mvn} kernel, its implementation must be stored in 
the\\ {\tt OBJdir/tune/blas/gemv/MVNCASES} directory.  Assuming I saved the
above implementation to {\tt mvn.c} in the above directory, I can test
the kernel from the {\tt OBJdir/tune/blas/gemv} directory with the
command: \verb+make <pre>mvnktest+, where \verb+<pre>+ specifies the
type/precision and is one of : {\tt s}, {\tt d}, {\tt c}, or {\tt z}.

This target uses the following {\tt make} variables which you can change
to vary the type of testing done; number in parens are the default values
that will be used if no command-line override is given:
\begin{itemize}
\item \verb+mu+ (1): Unrolling on M dimension
\item \verb+nu+ (1): Unrolling on N dimension
\item \verb+mvnrout+: the filename of your kernel implementation
\item \verb+Mt+ (297): Number of rows of $A$ (elements of $y$)
\item \verb+Nt+ (177): Number of columns of $A$ (elements of $x$)
\item \verb+ldat+ (\verb+Mt+): leading dimension for the matrix $A$
\item \verb+align+ (\verb+-Fx 16 -Fy 16 -Fa 16+): alignments that
      matrix and vectors must adhere to.
\item \verb+<pre>MVCC+ (ATLAS default compiler) : compiler required
      to compile your kernel
\item \verb+<pre>MVCFLAGS+ (ATLAS default flags) : compiler flags required
      to compile your kernel
\item \verb+beta+ (1) : 0 or 1 (scale of $y$)
\end{itemize}

Therefore, to test the $\beta=1.0$ case of double precision real:
\begin{verbatim}
>make dmvnktest mu=1 nu=1 mvnrout=mvn.c
.... bunch of compilation, etc ....
   TEST M=997, N=177, lda=1111, STARTED
   TEST M=997, N=177, be=0.00, lda=1111, incXY=1,1 PASSED
\end{verbatim}

To test single precision $\beta=0.0$ with a different shape:
\begin{verbatim}
>make smvnktest mu=1 nu=1 mvnrout=mvn.c beta=0 Mt=801 Nt=55 ldat=1000
.... bunch of compilation, etc ....
   TEST M=801, N=55, lda=1000, STARTED
   TEST M=801, N=55, be=0.00, lda=1000, incXY=1,1 PASSED
\end{verbatim}

Now we are ready to time this masterpiece.  We have two choices for timers.
The first such timer simply calls your newly-written kernel directly.
It does no cache flushing at all: it initializes the operands
(bringing them into the fastest level of heirarchy into which they fit),
and times the operations.  This is the timer to use when you want to preload
the problem to a given level of cache and see how fast your kernel is in
isolation.  The make target for this methodology is \verb+<pre>mvnktime+.

The second make target calls ATLAS's GEMV driver that builds the full
GEMV from the kernels.  This timer does cache flushing, and is what you
should use to estimate how fast the complete GEMV will be.  This make target
is \verb+<pre>mvntime+.

Both of these targets take largely the same make macros:
\begin{itemize}
\item \verb+mu+ (1): Unrolling on M dimension
\item \verb+nu+ (1): Unrolling on N dimension
\item \verb+mvnrout+: the filename of your kernel implementation
\item \verb+M+ (1000): Number of rows of $A$ (elements of $y$)
\item \verb+N+ (1000): Number of columns of $A$ (elements of $x$)
\item \verb+lda+ (\verb+M+): leading dimension for the matrix $A$
\item \verb+align+ (\verb+-Fx 16 -Fy 16 -Fa 16+): alignments that
      matrix and vectors must adhere to.
\item \verb+<pre>MVCC+ (ATLAS default compiler) : compiler required
      to compile your kernel
\item \verb+<pre>MVCFLAGS+ (ATLAS default flags) : compiler flags required
      to compile your kernel; pass "-x assembler-with-cpp" (along with any
      architecture-specific flags) if your kernel is written in assembly
      (assuming your compiler is \texttt{gcc}).
\item \verb+beta+ (1) : 0 or 1 (scale of $y$)
\item \verb+flushKB+ : for {\tt mvntime}, kilobytes of memory to flush.
\end{itemize}

Therefore, to time the probable speed of a complete GEMV of size 1000x32 
while flushing 16MB of memory, I would issue:
\begin{verbatim}
>make dmvntime mu=1 nu=1 mvnrout=mvn.c M=1000 N=32 flushKB=16384
GEMV: M=1000, N=32, lda=1008, AF=[16,16,16], AM=[0,0,0], beta=1.000000e+00, alpha=1.000000e+00:
   M=1000, N=32, lda=1008, nreps=57, time=2.011856e-05, mflop=3230.85
   M=1000, N=32, lda=1008, nreps=57, time=2.011042e-05, mflop=3232.15
   M=1000, N=32, lda=1008, nreps=57, time=2.006722e-05, mflop=3239.11
NREPS=3, MAX=3239.11, MIN=3230.85, AVG=3234.04, MED=3232.15
\end{verbatim}

\subsection{Testing and timing \kernk{mvt}}
The API of this routine is given by:
\begin{verbatim}
void ATL_UGEMV(ATL_CINT M, ATL_CINT N, const TYPE *A, ATL_CINT lda,
               const TYPE *X, TYPE *Y)
\end{verbatim}
If the routine is compiled with the macro {\tt BETA0} defined, then
it should perform the operation $y \leftarrow A^Tx$; if this macro is
not defined, then it should perform $y \leftarrow A^Tx + y$.
$A$ is a column-major $lda \times N$ contiguous array where $lda \ge M$.
$M$ specifies both the number of rows of $A$ and the length of the vector $X$.
$N$ provides the number of columns in $A$, and the length of the vector $Y$.

Since this is a $O(MN)$ operation, and A is $M \times N$ in size,
this algorithm is dominated by the load of $A$.  No reuse of $A$ is
possible, so the best we can do is reuse the vectors (through operations
like register and cache blocking).

Each element of $Y$ can obviously be computed by doing a dot product of
the corresponding row of $A$ with $X$, and this gives us the simplist
implementation possible:
\begin{verbatim}
#include "atlas_misc.h"  /* define TYPE macros */
void ATL_UGEMV(ATL_CINT M, ATL_CINT N, const TYPE *A, ATL_CINT lda,
               const TYPE *X, TYPE *Y)
{
   register int j, i;
   
   for (j=0; j < N; j++)
   {
      #ifdef BETA0
         register TYPE y0 = 0.0;
      #else
         register TYPE y0 = Y[j];
      #endif
      for (i=0; i < M; i++)
         y0 += A[i] * X[i];
      Y[j] = y0;
      A += lda;  /* done with this column */
   }
}
\end{verbatim}

To time and test a \kernk{mvt} kernel, its implementation must be stored in 
the\\ {\tt OBJdir/tune/blas/gemv/MVTCASES} directory.  Assuming I saved the
above implementation to {\tt mvt.c} in the above directory, I can test
the kernel from the {\tt OBJdir/tune/blas/gemv} directory with the
command: \verb+make <pre>mvtktest+, where \verb+<pre>+ specifies the
type/precision and is one of : {\tt s}, {\tt d}, {\tt c}, or {\tt z}.

This target uses the following {\tt make} variables which you can change
to vary the type of testing done; number in parens are the default values
that will be used if no command-line override is given:
\begin{itemize}
\item \verb+mu+ (1): Unrolling on M dimension
\item \verb+nu+ (1): Unrolling on N dimension
\item \verb+mvtrout+: the filename of your kernel implementation
\item \verb+Mt+ (297): Number of rows of $A$ (elements of $y$)
\item \verb+Nt+ (177): Number of columns of $A$ (elements of $x$)
\item \verb+ldat+ (\verb+Mt+): leading dimension for the matrix $A$
\item \verb+align+ (\verb+-Fx 16 -Fy 16 -Fa 16+): alignments that
      matrix and vectors must adhere to.
\item \verb+<pre>MVCC+ (ATLAS default compiler) : compiler required
      to compile your kernel
\item \verb+<pre>MVCFLAGS+ (ATLAS default flags) : compiler flags required
      to compile your kernel
\item \verb+beta+ (1) : 0 or 1 (scale of $y$)
\end{itemize}

Therefore, to test the $\beta=1.0$ case of double precision real:
\begin{verbatim}
>make dmvtktest mu=1 nu=1 mvtrout=mvt.c
.... bunch of compilation, etc ....
   TEST M=997, N=177, lda=1111, STARTED
   TEST M=997, N=177, be=1.00, lda=1111, incXY=1,1 PASSED
\end{verbatim}

We have two choices for timers.
The first such timer simply calls your newly-written kernel directly.
It does no cache flushing at all: it initializes the operands
(bringing them into the fastest level of heirarchy into which they fit),
and times the operations.  This is the timer to use when you want to preload
the problem to a given level of cache and see how fast your kernel is in
isolation.  The make target for this methodology is \verb+<pre>mvtktime+.

The second make target calls ATLAS's GEMV driver that builds the full
GEMV from the kernels.  This timer does cache flushing, and is what you
should use to estimate how fast the complete GEMV will be.  This make target
is \verb+<pre>mvttime+.

Both of these targets take largely the same make macros:
\begin{itemize}
\item \verb+mu+ (1): Unrolling on M dimension
\item \verb+nu+ (1): Unrolling on N dimension
\item \verb+mvtrout+: the filename of your kernel implementation
\item \verb+M+ (1000): Number of rows of $A$ (elements of $y$)
\item \verb+N+ (1000): Number of columns of $A$ (elements of $x$)
\item \verb+lda+ (\verb+M+): leading dimension for the matrix $A$
\item \verb+align+ (\verb+-Fx 16 -Fy 16 -Fa 16+): alignments that
      matrix and vectors must adhere to.
\item \verb+<pre>MVCC+ (ATLAS default compiler) : compiler required
      to compile your kernel
\item \verb+<pre>MVCFLAGS+ (ATLAS default flags) : compiler flags required
      to compile your kernel; pass "-x assembler-with-cpp" (along with any
      architecture-specific flags) if your kernel is written in assembly
      (assuming your compiler is \texttt{gcc}).
\item \verb+beta+ (1) : 0 or 1 (scale of $y$)
\item \verb+flushKB+ : for {\tt mvntime}, kilobytes of memory to flush.
\end{itemize}

Therefore, to time the probable speed of a complete GEMV of size 1000x32 
while flushing 16MB of memory, I would issue:
\begin{verbatim}
>make dmvttime mu=1 nu=1 mvtrout=mvt.c M=1000 N=32 flushKB=16384
GEMV: M=1000, N=32, lda=1008, AF=[16,16,16], AM=[0,0,0], beta=1.000000e+00, alpha=1.000000e+00:
   M=1000, N=32, lda=1008, nreps=57, time=2.809835e-05, mflop=2313.30
   M=1000, N=32, lda=1008, nreps=57, time=2.812956e-05, mflop=2310.74
   M=1000, N=32, lda=1008, nreps=57, time=2.807517e-05, mflop=2315.21
NREPS=3, MAX=2315.21, MIN=2310.74, AVG=2313.08, MED=2313.30
\end{verbatim}

\subsection{Testing and timing \kernk{ger}}
The API of \kernk{ger} is given by:
\begin{verbatim}
void ATL_UGERK(ATL_CINT M, ATL_CINT N, const TYPE *X, const TYPE *Y, 
               TYPE *A, ATL_CINT lda)
\end{verbatim}
This kernel performs the outer product operation $A \leftarrow xy^T + A$.
$A$ is a column-major $lda \times N$ contiguous array where $lda \ge M$.
$M$ specifies both the number of rows of $A$ and the length of the vector $X$.
$N$ provides the number of columns in $A$, and the length of the vector $Y$.

Since this is a $O(MN)$ operation, and A is $M \times N$ in size,
this algorithm is dominated by the load of $A$.  No reuse of $A$ is
possible, so the best we can do is reuse the vectors (through operations
like register and cache blocking).

The simplist implementation of this operation is done by doing a {\tt axpy}
operation for each column of the matrix:
\begin{verbatim}
#include "atlas_misc.h"  /* define TYPE macros */
void ATL_UGERK(ATL_CINT M, ATL_CINT N, const TYPE *X, const TYPE *Y, 
               TYPE *A, ATL_CINT lda)
{
   register int j, i;
   
   for (j=0; j < N; j++)
   {
      const register TYPE y0 = Y[j];
      for (i=0; i < M; i++)
         A[i] += X[i] * y0;
      A += lda;  /* done with this column */
   }
}
\end{verbatim}

To time and test a \kernk{ger} kernel, its implementation must be stored in 
the\\ {\tt OBJdir/tune/blas/gemv/R1CASES} directory.  Assuming I saved the
above implementation to {\tt ger.c} in the above directory, I can test
the kernel from the {\tt OBJdir/tune/blas/ger} directory with the
command: \verb+make <pre>r1ktest+, where \verb+<pre>+ specifies the
type/precision and is one of : {\tt s}, {\tt d}, {\tt c}, or {\tt z}.

This target uses the following {\tt make} variables which you can change
to vary the type of testing done; number in parens are the default values
that will be used if no command-line override is given:
\begin{itemize}
\item \verb+mu+ (1): Unrolling on M dimension
\item \verb+nu+ (1): Unrolling on N dimension
\item \verb+r1rout+: the filename of your kernel implementation
\item \verb+Mt+ (297): Number of rows of $A$ (elements of $x$)
\item \verb+Nt+ (177): Number of columns of $A$ (elements of $y$)
\item \verb+ldat+ (\verb+Mt+): leading dimension for the matrix $A$
\item \verb+align+ (\verb+-Fx 16 -Fy 16 -Fa 16+): alignments that
      matrix and vectors must adhere to.
\item \verb+<pre>R1CC+ (ATLAS default compiler) : compiler required
      to compile your kernel
\item \verb+<pre>R1CFLAGS+ (ATLAS default flags) : compiler flags required
      to compile your kernel
\end{itemize}

To test this kernel with a 511x220 matrix, I would issue:
\begin{verbatim}
>make dr1ktest mu=1 nu=1 r1rout=ger.c Mt=511 Nt=220
.... bunch of compilation, etc ....
   TEST CONJ=0, M=511, N=220, lda=511, incY=1, STARTED
   TEST CONJ=0, M=511, N=220, lda=511, incY=1, PASSED
\end{verbatim}

We have two choices for timers.
The first such timer simply calls your newly-written kernel directly.
It does no cache flushing at all: it initializes the operands
(bringing them into the fastest level of heirarchy into which they fit),
and times the operations.  This is the timer to use when you want to preload
the problem to a given level of cache and see how fast your kernel is in
isolation.  The make target for this methodology is \verb+<pre>r1ktime+.

The second make target calls ATLAS's GER driver that builds the full
GER from the kernels.  This timer does cache flushing, and is what you
should use to estimate how fast the complete GER will be.  This make target
is \verb+<pre>r1time+.

Both of these targets take largely the same make macros:
\begin{itemize}
\item \verb+mu+ (1): Unrolling on M dimension
\item \verb+nu+ (1): Unrolling on N dimension
\item \verb+r1rout+: the filename of your kernel implementation
\item \verb+M+ (1000): Number of rows of $A$ (elements of $x$)
\item \verb+N+ (1000): Number of columns of $A$ (elements of $y$)
\item \verb+lda+ (\verb+M+): leading dimension for the matrix $A$
\item \verb+align+ (\verb+-Fx 16 -Fy 16 -Fa 16+): alignments that
      matrix and vectors must adhere to.
\item \verb+<pre>R1CC+ (ATLAS default compiler) : compiler required
      to compile your kernel
\item \verb+<pre>R1CFLAGS+ (ATLAS default flags) : compiler flags required
      to compile your kernel; pass "-x assembler-with-cpp" (along with any
      architecture-specific flags) if your kernel is written in assembly
      (assuming your compiler is \texttt{gcc}).
\item \verb+beta+ (1) : 0 or 1 (scale of $y$)
\item \verb+flushKB+ : for {\tt mvntime}, kilobytes of memory to flush.
\end{itemize}

Therefore, to time the probable speed of a complete GER of size 1000x220
while flushing 16MB of memory, I would issue:
\begin{verbatim}
>make dr1time mu=1 nu=1 r1rout=ger.c M=800 Nt=220 flushKB=16384
.... bunch of compilation, etc ....
GER1: M=800, N=1000, lda=800, AF=[16,16,16], AM=[0,0,0], alpha=1.000000e+00:
   M=800, N=1000, lda=800, nreps=3, time=8.589835e-04, mflop=1863.60
   M=800, N=1000, lda=800, nreps=3, time=8.340690e-04, mflop=1919.27
   M=800, N=1000, lda=800, nreps=3, time=8.746753e-04, mflop=1830.16
NREPS=3, MAX=1919.27, MIN=1830.16, AVG=1871.01, MED=1863.60
\end{verbatim}

To time the kernel alone without cache flushing:
\begin{verbatim}
>make dr1ktime mu=1 nu=1 r1rout=ger.c M=800 Nt=220
.... bunch of compilation, etc ....
GER1: M=800, N=1000, lda=800, AF=[16,16,16], AM=[0,0,0], alpha=1.000000e+00:
   M=800, N=1000, lda=800, nreps=3, time=6.657494e-04, mflop=2404.51
   M=800, N=1000, lda=800, nreps=3, time=6.664957e-04, mflop=2401.82
   M=800, N=1000, lda=800, nreps=3, time=6.842208e-04, mflop=2339.60
NREPS=3, MAX=2404.51, MIN=2339.60, AVG=2381.97, MED=2401.82
\end{verbatim}

\subsection{Testing and timing \kernk{ger2}}
The rank-2 update kernel \kernk{ger2} performs the operation
$A \leftarrow xy^T + wz^T + A$, and has the API:
\begin{verbatim}
void ATL_UGER2K
   (ATL_CINT M, ATL_CINT N, const TYPE *X, const TYPE *Y,
    const TYPE *W, const TYPE *Z, TYPE *A, ATL_CINT lda)
\end{verbatim}
The rank-2 update kernel \kernk{ger2} uses the exact same testing
and timing methodology as described for \kernk{ger} in the previous
section, except the kernel must be stored in the {\tt R2CASES/} subdirectory,
and you substute ``r2'' for ``r1'' in the testing and timing commands,
and ``R2'' for ``R1'' in the compiler and flag macros.

A simple GER2 implememtation would be:
\begin{verbatim}
#include "atlas_misc.h"  /* define TYPE macros */
void ATL_UGER2K
   (ATL_CINT M, ATL_CINT N, const TYPE *X, const TYPE *Y,
    const TYPE *W, const TYPE *Z, TYPE *A, ATL_CINT lda)
{
   register ATL_INT i, j;

   for (j=0; j < N; j++)
   {
      const register TYPE y0=Y[j], z0=Z[j];
      for (i=0; i < M; i++)
         A[i] += X[i]*y0 + W[i]*z0;
      A += lda;  /* finished with this column */
   }
}
\end{verbatim}

Assuming I save the above file to {\tt R2CASES/r2k.c}, I would test:
\begin{verbatim}
>make sr2ktest mu=1 nu=1 r2rout=r2k.c
.... bunch of compilation, etc ....
   TEST CONJ=0, M=297, N=177, lda=297, incY=1, STARTED
   TEST CONJ=0, M=297, N=177, lda=297, incY=1, PASSED
\end{verbatim}

And time the single precision real kernel without cache flushing with:
\begin{verbatim}
>make sr2ktime mu=1 nu=1 r2rout=r2k.c
.... bunch of compilation, etc ....
GER2: M=1000, N=1000, lda=1000, AF=[16,16,16], AM=[0,0,0], alpha=1.000000e+00:
   M=1000, N=1000, lda=1000, nreps=1, time=9.489282e-04, mflop=4217.39
   M=1000, N=1000, lda=1000, nreps=1, time=9.714776e-04, mflop=4119.50
   M=1000, N=1000, lda=1000, nreps=1, time=9.486141e-04, mflop=4218.79
NREPS=3, MAX=4218.79, MIN=4119.50, AVG=4185.22, MED=4217.39
\end{verbatim}
\Wskip{
\subsection{OLDSTUFF}
All Level 2 BLAS are written in terms of 3 kernel routines:
\begin{enumerate}
\item Notranspose GEMV
\item Transpose GEMV
\item GER
\end{enumerate}

For complex codes, these kernels must also supply conjugate cases.

\subsection{Speeding Up GEMV, HEMV, SYMV, TRMV and TRSV}
These routines are all based on GEMV.  Therefore, to speed them up, the user
needs to supply a more efficient GEMV primitive.  The hand-coded GEMV
primitives may be found in ATLAS/tune/blas/gemv/CASES.

\subsubsection{The Kernel Description File}
\label{sec-gemvdesc}

The most important file in {\tt ATLAS/tune/blas/gemv/CASES}
is the primitive description
file, \verb+<pre>cases.dsc+.  Each precision has its own description file (as
indicated by \verb+<pre>+), and this file describes all of the routines to
time in order to find the best.  For instance, for double precision, we see:
\begin{verbatim}
speedy. cat CASES/dcases.dsc 
9
  1  8  0  0 ATL_gemvN_mm.c     "R. Clint Whaley"
  2  0  1  1 ATL_gemvN_1x1_1.c  "R. Clint Whaley"
  3 16 32  1 ATL_gemvN_1x1_1a.c "R. Clint Whaley"
  4  0  4  2 ATL_gemvN_4x2_0.c  "R. Clint Whaley"
  5  0  4  4 ATL_gemvN_4x4_1.c  "R. Clint Whaley"
  6  0  8  4 ATL_gemvN_8x4_1.c  "R. Clint Whaley"
  7  0 16  2 ATL_gemvN_16x2_1.c "R. Clint Whaley"
  8  0 16  4 ATL_gemvN_16x4_1.c "R. Clint Whaley"
  9 16 32  4 ATL_gemvN_32x4_1.c "R. Clint Whaley"
 6
101 8  0  0 ATL_gemvT_mm.c      "R. Clint Whaley"
102 0  2  8 ATL_gemvT_2x8_0.c   "R. Clint Whaley"
103 0  4  8 ATL_gemvT_4x8_1.c   "R. Clint Whaley"
104 0  4 16 ATL_gemvT_4x16_1.c  "R. Clint Whaley"
105 0  2 16 ATL_gemvT_2x16_1.c  "R. Clint Whaley"
106 0  1  1 ATL_gemvT_1x1_1.c   "R. Clint Whaley"
\end{verbatim}

The first number (in this case 9) is the number of NoTranspose primitives
to time.  This is followed by that number of primitive lines describing those
NoTrans primitives, and then we supply the number of Transpose primitives to
time (in this example, 6), followed by that number of primitive lines
describing the Transpose primitives.

As you can see, each line supplies four integers and a filename to the
search routine.  The filename is the filename of the primitive to time.
The first integer provides a unique integer ID (must be greater than zero)
for each primative line, and the other
three supply information necessary in order for the higher
level routines to do blocking.

This is the first piece of important information about these primitive
routines: no blocking should be done in them.  The appropriate blocking
is done by higher level ATLAS routines.  Most primitives
employ some kind of loop unrolling, and when these higher level routines
block in order to reuse vectors or matrices, it is important that this
blocking does not conflict with the primitives' unrolling factors (for instance,
if the primitive unrolls a given dimension by 8, but ATLAS blocks that
dimension to 3, ATLAS would always call the cleanup code).  So this is the
information conveyed by these three integers.

The form of a GEMV primitive line is:
\begin{verbatim}
<ID> <flag> <Yunroll> <Xunroll> <filename> "<author(s)>"
\end{verbatim}

As mentioned previously, \verb+<filename>+ is the primitive source file.  
\verb+<Yunroll>+
is the unrolling used for the loop that loops over the $Y$ vector, and
\verb+<Xunroll>+ is the unrolling used for the loop that loops over the
$X$ vector.  \verb+<flag>+ is a less obvious parameter which is used
to tell the search script about special properties of a kernel.

It is assumed that the user has supplied a "inner-product" based GEMV
implementation (i.e., an implementation which basically does \verb+<Yunroll>+
simultaneous dot products).  This default state is expressed to the search
by a \verb+<flag>+ value of 0.  However, since the inner product formulation of
NoTranspose GEMV loops across the non-contiguous dimension of the matrix,
some architectures need to employ an "outer-product" based NoTranspose GEMV
(i.e., a GEMV which is performed by doing \verb+<Xunroll>+ simultaneous axpy's).
This is indicated by a \verb+<flag>+ value of 16.  Finally, since ATLAS's
GEMM has
a code generator which allows it to achieve very good portable performance,
it is always worth seeing how optimal a GEMV can be obtained by simply
making the appropriate call to GEMM.  \verb+<flag>+ of 8 indicates
that this is what the kernel is doing.

In summary:
\begin{table}[h]
\begin{tabular}{||r|l||}\hline\hline
FLAG & MEANING \\\hline\hline
   0 & Normal \\\hline
   8 & GEMM-based primitive \\\hline
  16 & Outer-product or AXPY-based primitive (only valid for Notranspose GEMV) \\\hline \hline
\end{tabular}
\end{table}

\subsubsection{Writing a GEMV kernel}

There are several assumptions that need to hold true for a user-supplied GEMV
primitive.  First, the loop ordering must be that implied by the 
\verb+<flag>+ setting
the user supplies in the primitive description file, as discussed in
Section~\ref{sec-gemvdesc}.  
Each primitive makes assumptions about the arguments it handles,
and these assumptions are reflected in the routine name.  The function name of
a GEMV primitive is:
\begin{verbatim}
   ATL_<pre>gemv<Trans>_a1_x1_<betanam>_y1
\end{verbatim}
where:
\begin{itemize}
\item \verb+<pre>+ is replaced by the precision prefix:  
      {\tt s}, {\tt d}, {\tt c}, or {\tt z}.
\item \verb+<Trans>+ is replaced by transpose specifier:
  \begin{itemize}
   \item {\tt N} : NoTranspose
   \item {\tt T} : Transpose
   \item {\tt C} : Conjugate Transpose
   \item {\tt Nc} : NoTranspose, with conjugation
  \end{itemize}
\item \verb+<betanam>+ is replaced by the beta specifier this kernel
supplies.  All GEMV kernels must supply the following beta specifiers and
names:
  \begin{itemize}
     \item {\tt b0} : $\beta = 0$
     \item {\tt b1} : $\beta = 1$
     \item {\tt bXi0} : for complex GEMV only, specifies when 
$\beta \neq 0$ and $\beta \neq 1$, but the imaginary component is zero.
     \item {\tt bX} : beta is a input variable without known characteristics.
  \end{itemize}
\end{itemize}

For a given gemv primitive (either NoTranspose or Transpose), if the cpp macro
\verb+Conj_+
is defined we want the conjugate form of that transpose setting
(i.e., {\tt Nc} or {\tt C}).

Each file is further compiled with differing cpp settings to generate the
various beta cases.  The beta macro settings and their meanings are:

\begin{tabular}{||l|l||}\hline\hline
CPP MACRO & MEANING \\\hline\hline
{\tt BETA0}   & Primitive should provide $y \leftarrow A  x$ \\\hline
{\tt BETA1}   & Primitive should provide $y \leftarrow y + A x$ \\\hline
{\tt BETAX}   & Primitive should provide $y \leftarrow \beta y + A  x$ \\\hline
{\tt BETAXI0} & For complex only, primitive should provide $y \leftarrow \beta y + A x$, \\
        & where the imaginary component of beta is zero. \\\hline\hline
\end{tabular}

In terms of the BLAS API, the GEMV kernels additionally assume
\begin{itemize}
 \item $\alpha = 1$
 \item \verb+incX = 1+
 \item \verb+incY = 1+
 \item Column-major storage of A
\end{itemize}

Higher level ATLAS routines ensure these assumptions are true before calling
the primitive.

Therefore, the routine:
\begin{verbatim}
   ATL_dgemvN_a1_x1_b0_y1
\end{verbatim}
supplies a primitive doing notranspose gemv, on a column-major array with
$\alpha=1$, $\beta=0$, {\tt incX} = 1 and {\tt incY} = 1.  while:
\begin{verbatim}
   ATL_cgemvNc_a1_x1_bXi0_y1:
\end{verbatim}
supplies a primitive doing notranspose gemv, on a column-major array whose
elements should be conjugated before the multiplication, with
$\alpha=1$, {\tt incX} = 1, {\tt incY} = 1, and $\beta$ whose real component
is unknown, but whose imaginary component is known to be zero.

For greater understanding of how these CPP macros are used to compile multiple
primitives from one file, examine the provided CASES files.

The API of the primitive is:
\begin{verbatim}
   ATL_<pre>gemv<Trans>_a1_x1_<betanam>_y1
   (
      const int M,       /* length of Y vector */
      const int N,       /* length of X vector */
      const SCALAR alpha,/* ignored, assumed to be one */
      const TYPE *A,     /* pointer to column-major matrix */
      const int lda,     /* leading dimension of A, or row-stride */
      const TYPE *X,     /* vector to multiply A by */
      const int incX,    /* ignored, assumed to be one */
      const SCALAR beta, /* value of beta */
      TYPE *Y,           /* output vector */
      const int incY     /* ignored, assumed to be one */
   );
\end{verbatim}

where,
\begin{tabular}{||l|l|l|l|l||}\hline\hline
\verb+<pre>+ : & {\tt s} & {\tt d} & {\tt c} & {\tt z} \\\hline\hline
SCALAR & {\tt float} & {\tt double} & {\tt float*} & {\tt float*} \\\hline
TYPE   & {\tt float} & {\tt double} & {\tt float} & {\tt float} \\\hline\hline
\end{tabular}

Note that the meaning of M and N are slightly different than that used by the
Fortran77 API, in that they give the vector lengths, not array dimensions.

\subsubsection{GEMV examples}

Probably the simplest, notranspose GEMV kernel implementation is:
\begin{verbatim}
#ifdef BETA0
void ATL_dgemvN_a1_x1_b0_y1
#elif defined (BETA1)
void ATL_dgemvN_a1_x1_b1_y1
#else
void ATL_dgemvN_a1_x1_bX_y1
#endif
   (const int M, const int N, const double alpha, const double *A, 
    const int lda, const double *X, const int incX, const double beta,
    double *Y, const int incY)
{
   int i, j;
   register double y0;

   for (i=0; i != M; i++)
   {
      #ifdef BETA0
         y0 = 0.0;
      #elif defined(BETA1)
         y0 = Y[i];
      #elif defined(BETAX)
         y0 = Y[i] * beta;
      #endif
      for (j=0; j != N; j++) y0 += A[i+j*lda] * X[j];
      Y[i] = y0;
   }
}
\end{verbatim}

Saving this file to 
{\tt ATLAS/tune/blas/gemv/CASES/ATL\_dgemvN\_1x1\_1.c}, \\from the
{\tt BLDdir/tune/blas/gemv/} directory, we can test and time
the implementation by:
\begin{verbatim}
make dmvtstcaseN mvrout=CASES/ATL_dgemvN_1x1_1.c yu=1 xu=1
BEGINNING GEMV PRIMITIVE TESTING:

   TEST TA=N, M=997, N=177, lda=1004, beta=0.000000 STARTED
   TEST TA=N, M=997, N=177, lda=1004, beta=0.000000 PASSED
   TEST TA=N, M=997, N=177, lda=1004, beta=1.000000 STARTED
   TEST TA=N, M=997, N=177, lda=1004, beta=1.000000 PASSED
   TEST TA=N, M=997, N=177, lda=1004, beta=0.800000 STARTED
   TEST TA=N, M=997, N=177, lda=1004, beta=0.800000 PASSED


GEMV PRIMITIVE PASSED ALL TESTS

speedy. make dmvcaseN mvrout=CASES/ATL_dgemvN_1x1_1.c yu=1 xu=1
      gemvN_0 : 49.484536 MFLOPS
      gemvN_0 : 49.484536 MFLOPS
      gemvN_0 : 49.230769 MFLOPS
   gemvN_0 : 49.40 MFLOPS
\end{verbatim}

In the above examples, we pass {\tt yu}, the unrolling along the $y$ vector,
and {\tt xu}, the unrolling along the $x$ vector, so that when ATLAS
blocks the operation, it knows the correct unrolling to use to avoid
cleanup code.

A similarly sophisticated transpose primitive is:
\begin{verbatim}
#ifdef BETA0
void ATL_dgemvT_a1_x1_b0_y1
#elif defined (BETA1)
void ATL_dgemvT_a1_x1_b1_y1
#else
void ATL_dgemvT_a1_x1_bX_y1
#endif
   (const int M, const int N, const double alpha, const double *A,
    const int lda, const double *X, const int incX, const double beta,
    double *Y, const int incY)

{
   int i, j;
   register double y0;

   for (j=0; j != M; j++)
   {
      #ifdef BETA0
         y0 = 0.0;
      #elif defined(BETA1)
         y0 = Y[j];
      #else
         y0 = Y[j] * beta;
      #endif
      for (i=0; i != N; i++) y0 += A[i+j*lda] * X[i];
      Y[j] = y0;
   }
}
\end{verbatim}

Which we could test and time by:
\begin{verbatim}
speedy. make dmvtstcaseT mvrout=CASES/ATL_dgemvT_1x1_1.c xu=1 yu=1
BEGINNING GEMV PRIMITIVE TESTING:

   TEST TA=T, M=997, N=177, lda=1004, beta=0.000000 STARTED
   TEST TA=T, M=997, N=177, lda=1004, beta=0.000000 PASSED
   TEST TA=T, M=997, N=177, lda=1004, beta=1.000000 STARTED
   TEST TA=T, M=997, N=177, lda=1004, beta=1.000000 PASSED
   TEST TA=T, M=997, N=177, lda=1004, beta=0.800000 STARTED
   TEST TA=T, M=997, N=177, lda=1004, beta=0.800000 PASSED


GEMV PRIMITIVE PASSED ALL TESTS

speedy. make dmvcaseT mvrout=CASES/ATL_dgemvT_1x1_1.c xu=1 yu=1
      gemvT_0 : 37.647059 MFLOPS
      gemvT_0 : 37.647059 MFLOPS
      gemvT_0 : 37.647059 MFLOPS
   gemvT_0 : 37.65 MFLOPS
\end{verbatim}

An unsophisticated notranspose GEMV implementation for double precision
complex would be:
\begin{verbatim}
#ifdef BETA0
   #ifdef Conj_
      void ATL_dgemvNc_a1_x1_b0_y1
   #else
      void ATL_dgemvN_a1_x1_b0_y1
   #endif
#elif defined (BETA1)
   #ifdef Conj_
      void ATL_dgemvNc_a1_x1_b1_y1
   #else
      void ATL_dgemvN_a1_x1_b1_y1
   #endif
#elif defined (BETAXI0)
   #ifdef Conj_
      void ATL_dgemvNc_a1_x1_bXi0_y1
   #else
      void ATL_dgemvN_a1_x1_bXi0_y1
   #endif
#else
   #ifdef Conj_
      void ATL_dgemvNc_a1_x1_bX_y1
   #else
      void ATL_dgemvN_a1_x1_bX_y1
   #endif
#endif
   (const int M, const int N, const double *alpha,
    const double *A, const int lda, const double *X, const int incX,
    const double *beta, double *Y, const int incY)
{
   int i, j;
   const int M2 = M<<1, N2 = N<<1;
   #if defined(BETAX)
      const double rbeta = *beta, ibeta = beta[1];
   #elif defined(BETAXI0)
      const double rbeta = *beta;
   #endif
   register double ra, ia, rx, ix, ry, iy;

   for (i=0; i != M2; i += 2)
   {
      #ifdef BETA0
         ry = iy = 0.0;
      #elif defined(BETAX)
         rx = rbeta; ix = ibeta;
         ra = Y[i]; ia = Y[i+1];
         ry = ra * rx - ia * ix;
         iy = ra * ix + ia * rx;
      #else
         ry = Y[i];
         iy = Y[i+1];
         #ifdef BETAXI0
            rx = rbeta;
            ry *= rx;
            iy *= rx;
         #endif
      #endif
      for(j=0; j != N2; j += 2)
      {
         ra = A[i+j*lda]; ia = A[i+j*lda+1];
         rx = X[j]; ix = X[j+1];
         ry += ra * rx;
         iy += ra * ix;
         #ifdef Conj_
            ry += ia * ix;
            iy -= ia * rx;
         #else
            ry -= ia * ix;
            iy += ia * rx;
         #endif
      }
      Y[i] = ry;
      Y[i+1] = iy;
   }
}
\end{verbatim}

Which, when saved to {\tt ATLAS/tune/blas/gemv/CASES/ATL\_zgemvN\_1x1\_1.c},
we could test and time by:
\begin{verbatim}
 make zmvtstcaseN mvrout=CASES/ATL_zgemvN_1x1_1.c xu=1 yu=1

BEGINNING GEMV PRIMITIVE TESTING:

   TEST TA=N, M=997, N=177, lda=1004, beta=(0.000000,0.000000) STARTED
   TEST TA=N, M=997, N=177, lda=1004, beta=(0.000000,0.000000) PASSED
   TEST TA=-, M=997, N=177, lda=1004, beta=(0.000000,0.000000) STARTED
   TEST TA=-, M=997, N=177, lda=1004, beta=(0.000000,0.000000) PASSED
   TEST TA=N, M=997, N=177, lda=1004, beta=(1.000000,0.000000) STARTED
   TEST TA=N, M=997, N=177, lda=1004, beta=(1.000000,0.000000) PASSED
   TEST TA=-, M=997, N=177, lda=1004, beta=(1.000000,0.000000) STARTED
   TEST TA=-, M=997, N=177, lda=1004, beta=(1.000000,0.000000) PASSED
   TEST TA=N, M=997, N=177, lda=1004, beta=(0.800000,0.000000) STARTED
   TEST TA=N, M=997, N=177, lda=1004, beta=(0.800000,0.000000) PASSED
   TEST TA=-, M=997, N=177, lda=1004, beta=(0.800000,0.000000) STARTED
   TEST TA=-, M=997, N=177, lda=1004, beta=(0.800000,0.000000) PASSED
   TEST TA=N, M=997, N=177, lda=1004, beta=(0.800000,0.300000) STARTED
   TEST TA=N, M=997, N=177, lda=1004, beta=(0.800000,0.300000) PASSED
   TEST TA=-, M=997, N=177, lda=1004, beta=(0.800000,0.300000) STARTED
   TEST TA=-, M=997, N=177, lda=1004, beta=(0.800000,0.300000) PASSED


GEMV PRIMITIVE PASSED ALL TESTS

speedy. make zmvcaseN mvrout=CASES/ATL_zgemvN_1x1_1.c xu=1 yu=1
      gemvN_0 : 78.688525 MFLOPS
      gemvN_0 : 78.688525 MFLOPS
      gemvN_0 : 78.688525 MFLOPS
   gemvN_0 : 78.69 MFLOPS

\end{verbatim}

\subsubsection{GEMV kernel notes}
All routines except SYMV call the GEMV kernel in the same fashion.
Other than SYMV, all routines cannot reduce the load of $A$ from $O(N^2)$,
but can reduce the memory access of both $X$ and $Y$ from $O(N^2)$ to
$O(N)$.  In general, the $Y$ access is reduced by register blocking in
the GEMV kernel.  Therefore, the higher level routines block $X$ such
it is reused across kernel invocations in L1 (if you write a axpy-based
notranspose GEMV kernel, $Y$ is blocked instead of $X$).  What this amounts
to is partitioning $X$ via:
$N_p = \frac{S_1 - R_y}{R_y+1}$, where $S_1$ is the size, in elements,
of the Level 1 cache, $N_p$ is the partitioning of $X$, and $R_y$ corresponds
to the {\tt Yunroll} of your input file.  The equation is actually a little
more complicated than this, as ATLAS may want to use less than the full 
$S_1$ to avoid cache thrashing and throwing useful sections away between
kernel calls.  However, this gives the user some idea of the importance of
these parameters.  In particular, it shows that {\tt Yunroll} should not
be allowed to grow too large, for fear of causing the $X$ loop to be
too short to support good optimization.

Also, note that after the first invocation of the kernel, $X$ will come
from L1, leaving $A$ the dominant data cost.

At present, SYMV does a different blocking that blocks both $X$ {\bf and}
$Y$ (all other routines block only one dimension), so that $A$ is reused
between calls to the Transpose and NoTranspose kernels.  This may
eventually change as greater sophistication is achieved (as you might
imagine, you get two very different GEMV kernels if one is expecting
$A$ from main memory, and the other expects $A$ to come from L1, as
in this case; this means we may at some time generate a specialized
L1-contained GEMV kernel).

Note that the $\beta = 0$ case must not read $Y$, since the memory may
legally be unitialized.

\subsection{Speeding Up GER, GERU, GERC, HER, HER2, SYR and SYR2}
All of these routines rely on the GER primitive for their performance.  The
hand-written primitives tried by ATLAS may be found in
\begin{verbatim}
   ATLAS/tune/blas/ger/CASES.
\end{verbatim}

Most of the discussion of the GEMV primitives applies to the GER primitives
as well, so I assume you have read and are familiar with the concepts
discussed above.  As before, the routines to be timed are given
in a kernel description file, \verb+<pre>cases.dsc+.  GER does not have a
transpose case, so this file first lists the number of GER primitives to search,
followed by that many primitive lines describing them.

GER primitive lines are of the form:
\begin{verbatim}
<ID> <flag> <Xunroll> <Yunroll> <filename> "<author(s)>"
\end{verbatim}

\begin{itemize}
\item \verb+<ID>+: Integer greater than 0 uniquely identifying this entry
\item \verb+<flag>+: is an integer flag which is ignored at the moment
\item \verb+<Xunroll>+: is the unrolling of the loop over the X vector
      (i.e. the M-loop)
\item \verb+<Yunroll>+: is the unrolling of the loop over the Y vector
      (i.e. the N-loop)
\item \verb+<filename>+: is the name of the C source file for the primitive.
\item \verb+<author(s)>+: author(s) name(s)
\end{itemize}

The API for the ger primitive is:
\begin{verbatim}
#if defined(SCPLX) || defined(DCPLX)
   #ifdef Conj_
      ATL_<pre>ger1c_a1_x1_yX
   #else
      ATL_<pre>ger1u_a1_x1_yX
   #endif
#else
   ATL_<pre>ger1_a1_x1_yX
#endif
   (
      const int M,       /* length of X vector */
      const int N,       /* length of Y vector */
      const SCALAR alpha,/* ignored, assumed to be one */
      const TYPE *X,     /* pointer to X vector */
      const int incX,    /* ignored, assumed to be one */
      const TYPE *Y,     /* pointer to Y vector */
      const int incY     /* increment of Y vector; NOTE: NOT IGNORED */
      TYPE *A,     /* pointer to column-major matrix */
      const int lda,     /* leading dimension of A, or row-stride */
   );
\end{verbatim}

Assumptions:
\begin{itemize}
 \item $\alpha = 1$
 \item \verb+incX = 1+
 \item Column-major storage of A
\end{itemize}

\subsubsection{GER examples}

A simple double precision real implementation of GER would then be:
\begin{verbatim}
void ATL_dger1_a1_x1_yX(const int M, const int N, const double alpha,
                        const double *X, const int incX, const double *Y,
                        const int incY, double *A, const int lda)
{
   int i, j;
   register double y0;

   for (j=0; j < N; j++)
   {
      y0 = Y[j*incY];
      for (i=0; i < M; i++) A[i+j*lda] += X[i] * y0;
   }
}
\end{verbatim}
Which we can test and time with:
\begin{verbatim}
speedy. make dr1tstcase r1rout=CASES/ATL_dger1_1x1_1.c xu=1 yu=1
   TEST CONJ=0, M=997, N=177, lda=1006, incY=1, STARTED
   TEST CONJ=0, M=997, N=177, lda=1006, incY=1, PASSED
   TEST CONJ=0, M=997, N=177, lda=1006, incY=3, STARTED
   TEST CONJ=0, M=997, N=177, lda=1006, incY=3, PASSED
   TEST CONJ=0, M=997, N=177, lda=1006, incY=-3, STARTED
   TEST CONJ=0, M=997, N=177, lda=1006, incY=-3, PASSED

speedy. make dr1case r1rout=CASES/ATL_dger1_1x1_1.c xu=1 yu=1
      dger_0 : 31.168831 MFLOPS
      dger_0 : 31.788079 MFLOPS
      dger_0 : 31.578947 MFLOPS
   dger_0 : 31.51 MFLOPS
\end{verbatim}

A simple double precision complex implementation would be:
\begin{verbatim}
#ifdef Conj_
   void ATL_zger1c_a1_x1_yX
#else
   void ATL_zger1u_a1_x1_yX
#endif
   (const int M, const int N, const double *alpha, const double *X, 
    const int incX, const double *Y, const int incY, double *A, const int lda)
{
   const int M2 = M<<1, N2 = N<<1;
   int i, j;
   register double ry, iy, ra, ia, rx, ix;

   for (j=0; j < N2; j += 2)
   {
      ry = Y[incY*j];
      iy = Y[incY*j+1];
      for (i=0; i < M2; i += 2)
      {
         rx = X[i];
         ix = X[i+1];
         ra = rx * ry;
         ia = ix * ry;
         #ifdef Conj_
            ra += ix * iy;
            ia -= rx * iy;
         #else
            ra -= ix * iy;
            ia += rx * iy;
         #endif
         A[i+j*lda] += ra;
         A[i+j*lda+1] += ia;
      }
   }
}
\end{verbatim}

Which we test in time by:
\begin{verbatim}
speedy. make zr1tstcase r1rout=CASES/ATL_zger1_1x1_1.c xu=1 yu=1
   TEST CONJ=0, M=997, N=177, lda=1006, incY=1, STARTED
   TEST CONJ=0, M=997, N=177, lda=1006, incY=1, PASSED
   TEST CONJ=0, M=997, N=177, lda=1006, incY=3, STARTED
   TEST CONJ=0, M=997, N=177, lda=1006, incY=3, PASSED
   TEST CONJ=0, M=997, N=177, lda=1006, incY=-3, STARTED
   TEST CONJ=0, M=997, N=177, lda=1006, incY=-3, PASSED
   TEST CONJ=1, M=997, N=177, lda=1006, incY=1, STARTED
   TEST CONJ=1, M=997, N=177, lda=1006, incY=1, PASSED
   TEST CONJ=1, M=997, N=177, lda=1006, incY=3, STARTED
   TEST CONJ=1, M=997, N=177, lda=1006, incY=3, PASSED
   TEST CONJ=1, M=997, N=177, lda=1006, incY=-3, STARTED
   TEST CONJ=1, M=997, N=177, lda=1006, incY=-3, PASSED

speedy. make zr1case r1rout=CASES/ATL_zger1_1x1_1.c xu=1 yu=1
      zger_0 : 39.024390 MFLOPS
      zger_0 : 38.709677 MFLOPS
      zger_0 : 39.024390 MFLOPS
   zger_0 : 38.92 MFLOPS
\end{verbatim}

\subsubsection{GER kernel notes}

As in GEMV, GER blocks $X$ so that it is reused in L1.  Since the
dominant direction of the loop is expected to be down the columns of
$A$, {\tt incY} is not restricted to 1, while {\tt incX} is.  As in GEMV,
all routines except SYR2 block only the the $X$ dimension, while
SYR2 blocks both, so that $A$ can be reused in L1.
}

\section{Speeding up the Level 1 BLAS}

\subsection{General comments for Level 1 optimization}
All Level 1 optimizations are carried on in the 
{\tt BLDdir/tune/blas/level1} directory and its subdirectories.  Under this
directory are subdirs with names corresponding to the generic name of the
routine in question (eg. AXPY, IAMAX, DOT, etc).  It is in these subdirs that
the user should place the routines to test and time.

A great deal of the performance win to be had on the Level 1 BLAS, 
particularly for long vectors, comes from using data prefetch.  
ATLAS now includes a prefetch header file (described in
Section~\ref{sec-prefetch}), which makes prefetch instructions for 
various systems available for C programmers.

\subsubsection{No general kernels here}
The Level 1 BLAS are in general too basic to be written in terms of simpler
kernels.  Therefore, each Level 1 routine must be pretty much optimized
individually.  The only real reuse of kernels comes from either
complex-to-real reuse, or one BLAS routine simplifying to another.

For an example of complex-to-real reuse, consider ZNRM2 which, when called with
{\tt incX = 1}, can be simply implemented as a call to DNRM2 with {\tt 2*N}.
An example of a routine simplifying to another would be calling DSCAL with
{\tt alpha = 0.0}, which devolves to a call to the primitive {\tt ATL\_dset}.

Therefore, if you are planning to optimize a particular case, be sure to read
the appopriate section below to make sure that the case you want to optimize
is not implemented by a call to another routine.

As with other ATLAS optimizations, each routine has its own kernel index file,
one for each precision (eg., {\tt AXPY/dcases.dsc} indexes the various DAXPY 
implementations that should be tested and timed during the install process).
All of these index files follow the format below, though they leave out
unneeded parameters (eg., SCAL, which operates on only one vector, will not
have an entry for {\tt incY} or {\tt BETA}).

\subsubsection{General index file description}
The general form of the index files are:
\begin{verbatim}
<Number of kernels>
<ID> <alpha> <beta> <incX> <incY> <source file> <author name>
 .
 .
<ID> <alpha> <beta> <incX> <incY> <source file> <author name>
\end{verbatim}

Here's an explanation of these values:
\begin{itemize}
\item {\bf ID}: Strictly positive ($>0$) integer which must be unique in
the file (eg., two lines should not begin with the same ID).  This ID is
used to distinguish between kernels, so reusing one will result in
confusion.
\item{\bf ialpha}:  Integer flag describing special conditions of the scalar
{\tt alpha}.  Possible values are:
   \begin{itemize}
   \item {\bf 1}: alpha = 1.0 (for complex, (1.0, 0.0))
   \item {\bf -1}: alpha = -1.0 (for complex, (-1.0, 0.0))
   \item {\bf 0}: For complex only, this flag means that the imaginary component
                 is zero (i.e., scalar is actually real, not complex).
   \item {\bf 2}: or anything else, is assumed to mean that nothing special is
      known about the scalar.
\end{itemize}
\item{\bf ibeta}: Same as {\tt ialpha}, but for second scalar.
\item{\bf incX}: Stride on the X vector your kernel is specialized for
                 (eg., {\tt incX=1} means the kernel works only for X vectors
                 with unit stride).
\item{\bf incY}: Vector stride for Y vector.
\item{\bf source file}: The name of the source file appearing in the BLAS
    subdir (eg, {\tt AXPY/myaxpy.c} would mean an entry of {\tt myaxpy.c}).
\item{\bf author name}: Name of the author, enclosed in double quotes.
\end{itemize}

For example, here is an example {\tt AXPY/dcases.dsc}:
\begin{verbatim}
3
1  2  0  0 axpy1_x0y0.c       "R. Clint Whaley"
2  2  1  1 axpy1_x1y1.c       "R. Clint Whaley"
2 -1  1  1 axpy1_an1x1y1.c    "R. Clint Whaley"
<ID> <ialpha> <incX> <incY> <rout> <author> [\
\end{verbatim}

So, to make this work, all three filenames mentioned above must appear in
your {AXPY/} subdir.  The first kernel routine is the general case:
general incX, incY, and alpha.  The second specializes both vector increments
to unit stride, but takes any alpha.  Finally, the last routine specializes
even further, with unit strides and requiring alpha be negative one.  
Note that all index files must contain at least one routine that handles
the most general cases, and passes the tester.

If you need a particular compiler and flag combo to compile your kernel
you end the line with a \verb+\+, and put the compiler on the following line,
and the flags on a line after that.  So, if {\tt axpy\_x1y1.c} needed {\tt gcc},
with {\tt -O1 -fschedule-insns -DGOODPERFORMANCE} for flags, the above
file would be changed to:
\begin{verbatim}
3
1  2  0  0 axpy1_x0y0.c       "R. Clint Whaley"
2  2  1  1 axpy1_x1y1.c       "R. Clint Whaley" \
gcc
-O1 -fschedule-insns -DGOODPERFORMANCE
2 -1  1  1 axpy1_an1x1y1.c    "R. Clint Whaley"
<ID> <ialpha> <incX> <incY> <rout> <author> [\
\end{verbatim}
\subsubsection{Things you can assume, and/or need to know when writing kernels}
First, ATLAS will ensure that {\tt N > 0} for all routines.  Vector
strides will never be zero, and for those routines taking only one vector,
the vector stride will always be positive.

For routines taking more than one vector, this is not so simple.  The first
thing that needs to be clear is that internally, ATLAS does not use the
pointer conventions that the BLAS do when it comes to handling vectors
with negative increments.  In the BLAS, if you want to operate on a vector
with a negative increment, you pass in the {\tt end} of the negatively
incremented vector (i.e, you always pass the part of the vector closest
to 0 in memory, regardless of the increment).  ATLAS will instead pass
the beginning of the vector regardless of sign, so that the following
sample code would correctly step through a vector regardless of the sign:
\begin{verbatim}
   for (i=0; i < N; i++, X += incX, Y += incY) ...
\end{verbatim}.

ATLAS also fiddles with the vector increments so that:
\begin{enumerate}
 \item Both increments are never negative
 \item {\tt incY} will be negative only if~$|${\tt incX}$|$~==~1 {\bf and}
       $|${\tt incY}$|$~!=~1.
\end{enumerate}

There may be additional constraints, on an individual routine basis.

\subsection{Testing a kernel}
Testing is straightforward.  If your BLAS routine is {\tt <blas>},
and your file in the {\tt /<blas>} subdir is {\tt <rout>},
then you compile and test it by:
\begin{verbatim}
   make <pre><blas>test urout=<rout>
\end{verbatim}

On the makefile line, you can also pass {\tt N=<N>} to test a particular
length vector.  Each tester has various flags that can be passed to them,
and these can be passed to make via the {\tt opt="flags"} macro.

To find out what flags a tester takes, compile it, and run the resulting
executable with {\tt -help} (or just look in the source, you slacker).

Alright, here's a specific example.  Let's say I have just written a new
AXPY implementation and put it in {\tt AXPY/myaxpy.c}.  I can then test
it with:
\begin{verbatim}
   make daxpytest urout=myaxpy.c
\end{verbatim}

When I fire this off, it does a bunch of compilation, and then spits out
something like:
\begin{verbatim}
  ITST         N     alpha  incX  incY    TEST
======  ========  ========  ====  ====  ======
     0       777      1.00     1     1  PASSED
     1       777     -1.00     1     1  PASSED
     2       777      0.90     1     1  PASSED
ALL AXPY SANITY TESTS PASSED.
\end{verbatim}

Now, let's say I've written a complex routine (cmyaxpy), that can handle
multiple increments.  To test various increments, I do:
\begin{verbatim}
speedy. make zaxpytest urout=cmyaxpy.c opt="-X 4 -1 1 -2 3 -Y 4 1 -1 2 -3"
  ITST         N    ralpha   ialpha  incX  incY    TEST
======  ========  ======== ========  ====  ====  ======
     0       777      1.00     0.00    -1     1  PASSED
     1       777     -1.00     0.00    -1     1  PASSED
     2       777      1.30     0.00    -1     1  PASSED
     3       777      0.90     1.10    -1     1  PASSED
     .....
     .....
    61       777     -1.00     0.00     3    -3  PASSED
    62       777      1.30     0.00     3    -3  PASSED
    63       777      0.90     1.10     3    -3  PASSED
ALL AXPY SANITY TESTS PASSED.
\end{verbatim}

\subsection{Timing a kernel}
{\bf IMPORTANT NOTE}: At present, all of ATLAS Level 1 timing is
completely inaccurate for short vectors.  Both the level 1 timing
in {\tt ATLAS/bin}, and the kernel timers described here screw this
up.  Essentially, our portable cache flushing mechanisms are not complete enough
to get things completely out of cache, so you see that short vectors appear
to get better performance than long vectors, a patent impossibility if all
caches were correctly flushed.  The only ``solution'' we have at the moment
is to time vectors that themselves overflow the cache (i.e., N=1000000).

Timing works a lot like testing.  If your BLAS routine is {\tt <blas>},
and your file in the {\tt /<blas>} subdir is {\tt <rout>},
then you compile and time it by:
\begin{verbatim}
   make <pre><blas>case urout=<rout>
\end{verbatim}

So, to time the previously meantioned {\tt myaxpy.c} with a million
length vector, you'd have the following:
\begin{verbatim}
speedy. make daxpycase urout=myaxpy.c N=1000000
      N=1000000, tim=5.000000e-02
      N=1000000, tim=5.000000e-02
      N=1000000, tim=5.000000e-02
   N=1000000, time=5.000000e-02, mflop=40.000000
N=1000000, incX=1, incY=1, mflop = 40.000000
\end{verbatim}

So, we got 40Mflop out of that implementation.
All the timers take the flag {\tt -C <cache flush size in bytes>}, which
you can use to get the L1 (or L2) contained performance.  Eg, the timer
is usually doing it's best to flush the caches, but I want to see what
performance I get when in the L1 cache.  What I do is:
\begin{verbatim}
make daxpycase urout=myaxpy.c N=500 opt="-C 512"
      N=500, tim=4.296302e-06
      N=500, tim=3.938276e-06
      N=500, tim=4.296302e-06
   N=500, time=4.176960e-06, mflop=239.408571
N=500, incX=1, incY=1, mflop = 239.408571
\end{verbatim}
Flushing 512 bytes is not going to do anything, and N=500 will not overflow
cache, so we see that L1-contained operations come in at 240Mflop . . .

When it comes to timing codes with varying increments, the timer is not as
flexible as the tester.  It can time only one given increment value at a
time.  So, to time axpy with incX=3 and incY=4, we'd do:
\begin{verbatim}
   make daxpycase urout=myaxpy N=500 opt="-X 3 -Y 4"
\end{verbatim}

\subsection{What to do if you are writing in assembler}
If your kernel is written in gas assembler, you can tell the tester
and timer that by setting the \verb+<pre>UCC+ and \verb+<pre>UCCFLAGS+ 
appropriately.
For instance, to test the {\tt DDOT} code given in Section~\ref{sec-DDOT-ass},
you would issue:
\begin{verbatim}
   make ddottest dUCC=gcc dUCCFLAGS="-x assembler-with-cpp" urout=ddot.c
\end{verbatim}

\subsection{Ramblings on special cases}

When presented with these options, you may be tempted to ask what cases
you should optimize.  On the other hand, you might also validly ask: 
isn't it obvious that it is always {\tt incX = incY = 1}, and why did
you bother building in this special case flexibility?

First, it must be acknowledged that on most systems, bandwidth contraints
will make any non-unit optimization tough.  Doesn't mean it can't be done,
though, at least to some degree.

For an example of why you want some flexibility, consider the COPY routine.
Of course, the most optimizable routine is {\tt incX = incY = 1}.  However,
one big use of the routine I make myself is to copy from noncontiguous
storage to unit stride, so that more efficient access is possible.  This
suggests that the ability to make {\tt incX} arbitrary and {\tt incY = 1}
might be useful in this routine.

For an example of a non-unit fixed stride, think of doing conjugation on
a complex vector.  That is essentially a real SCAL, with {\tt incX = 2} and
{\tt alpha = -1.0}, and if this was important to you, you could optimize that
exact case.

Ultimately, one can never foresee when flexibility will be needed, anyway.
With the present case, a user that knew he was accessing a vector with
increments of 50, 100, 500 all the time, could create special cases for
them . . .

All that said, {incX=incY=1} is often the only case where optimization will
have a noticable effect, so it's where I'd concentrate my efforts as long
as there's not some reason to do otherwise.

\subsection{Notes for ROTG, ROTMG, ROT, ROTM}
\begin{enumerate}
\item
ROTG is essentially a scalar routine, so I didn't feel it was worth
optimizing beyond just not writing it in a non-braindead way in the first
place.

\item
As for the rest of the routines, OK, so I haven't supported speeding up
any of them yet, sue me.
I myself don't use'em, and I couldn't write the testers without scoping
the code, so they got a pass.  Let me know if you call these guys, maybe
I'll get off my duff . . .
\end{enumerate}

\subsection{Notes for ROT}
\begin{enumerate}
\item Applies plane rotation.
\item
An index file input line looks like:
\begin{verbatim}
<ID> <C> <S> <incX> <incY> <source file> <author name>
\end{verbatim}
\item Will never be called with $C = 1, S = 0$ (ROT is no-op for this case)
\item For complex, if both vectors have unit stride, the real routine is
      called with {\tt 2*N}.
\end{enumerate}

\subsection{Notes for ASUM}
\begin{enumerate}
\item Performs $asum \leftarrow || re( x ) || _ {1}  + || im( x ) || _ {1} $
\item
An index file input line looks like:
\begin{verbatim}
<ID> <incX> <source file> <author name>
\end{verbatim}
\item For complex, if both vectors have unit stride, the real routine is
      called with {\tt 2*N}.
\end{enumerate}

\subsection{Notes for AXPBY}
\begin{enumerate}
\item Performs $ y \leftarrow \alpha x + \beta y $
\item An index file input line looks like:
\begin{verbatim}
<ID> <alpha> <beta> <incX> <incY> <source file> <author name>
\end{verbatim}

\item The kernel will never be called with {\tt beta == 1.0} (AXPY is called
      instead).
\item The kernel will never be called with {\tt beta == 0.0} (CPSC is called
      instead).
\item The kernel will never be called with {\tt alpha == 0.0} (SCAL is called
      instead).
\item For complex, if both vector strides are 1, and the imaginary
      components of alpha and beta are both zero, the real kernel is
      called with {\tt 2*N}.

\end{enumerate}

\subsection{Notes for AXPY}
\begin{enumerate}
\item Performs $ y \leftarrow \alpha x + y $
\item An index file input line looks like:
\begin{verbatim}
<ID> <alpha> <incX> <incY> <source file> <author name>
\end{verbatim}
\item The kernel will never be called with {\tt alpha == 0.0}.
\item For complex, if both vector strides are 1, and the imaginary component
      of alpha is zero, the real kernel is called with {2*N}.
\end{enumerate}

\subsection{Notes for COPY}
\begin{enumerate}
\item Performs $ y \leftarrow x $
\item
An index file input line looks like:
\begin{verbatim}
<ID> <incX> <incY> <source file> <author name>
\end{verbatim}
\item For complex with both vector strides 1, the real kernel is called with
      {2*N}.
\item {\tt incX} arbitrary {\tt incY=1} is interesting since it represents
      copying to contiguous storage.
\end{enumerate}


\subsection{Notes for CPSC}
\begin{enumerate}
\item Performs $ y \leftarrow \alpha x $
\item An index file input line looks like:
\begin{verbatim}
<ID> <alpha> <incX> <incY> <source file> <author name>
\end{verbatim}
\item Never called with $\alpha$ of zero (SET is called instead).
\item Never called with $\alpha$ of one  (COPY is called instead).
\item For complex, if  both vector strides are 1, and the imaginary component of
      $\alpha$ is zero, the real kernel is called with {2*N}.
\item {\tt incX} arbitrary {\tt incY=1} is interesting since it represents
      copying to contiguous storage.
\end{enumerate}

\subsection{Notes for DOT}
\begin{enumerate}
\item Performs $dot \leftarrow x ^ {T} y $ or $ dot \leftarrow x ^ {H} y $
\item
An index file input line looks like:
\begin{verbatim}
<ID> <incX> <incY> <source file> <author name>
\end{verbatim}

\item
For complex, {\tt DOTU}'s index file is {\tt [c,z]cases.dsc}, while
{\tt DOTC}'s is {\tt [c,z]casesc.dsc}.  

\item
During compilation of the {DOTC} kernels, ATLAS throws the compilation
flag {\tt -DConj\_} (this is so both nonconjugate and conjugate cases
can come from the same file).

\end{enumerate}

\subsection{Notes for IAMAX}
\begin{enumerate}
\item Performs 
      $ amax \leftarrow 1^{st} k \ni | re( x _ {k} ) |  + | im( x _ {k} ) | $
\item
An index file input line looks like:
\begin{verbatim}
<ID> <incX> <source file> <author name>
\end{verbatim}
\end{enumerate}

\subsection{Notes for NRM2}
\begin{enumerate}
\item Performs $nrm2 \leftarrow || x || _ {2} $
\item
An index file input line looks like:
\begin{verbatim}
<ID> <incX> <source file> <author name>
\end{verbatim}

\item
For complex, if both vectors have unit stride, the real routine is called
with {\tt 2*N}.

\item
The user should be aware that this routine was originally added to the BLAS
mostly for {\em accuracy} concerns, not for optimization.  The simple
implementation of this routine:
\begin{verbatim}
   for (nrm2=0.0, i=0; i < N; i++, X += incX) nrm2 += *X * *X;
   return(sqrt(nrm2));
\end{verbatim}
will needlessly overflow for half of the exponent range.  Thus, a good
implementation must either use extended precision or a scaling algorithm
(such as the sum of squares used in the reference BLAS) to prevent overflow
in the squaring process.

\end{enumerate}

\subsection{Notes for SCAL}
\begin{enumerate}
\item Performs $ x \leftarrow \alpha x $
\item An index file input line looks like:
\begin{verbatim}
<ID> <alpha> <incX> <source file> <author name>
\end{verbatim}
\item If {\tt alpha} is zero, the ATLAS internal routine {\tt ATL\_set}
      is called.
\item For complex, if {\tt incX == 1}, and the imaginary component of
      {\tt alpha} is zero, the real routine is called with {\tt N*2}.
\item The {\tt incX == 2}, {\tt alpha == -1.0} real cases can be used
      for complex conjugation.
\end{enumerate}

\subsection{Notes for SET}
\begin{enumerate}
\item Performs $ x \leftarrow \alpha $
\item An index file input line looks like:
\begin{verbatim}
<ID> <alpha> <incX> <source file> <author name>
\end{verbatim}
\item For complex, if {\tt incX == 1}, and the imaginary component of
      {\tt alpha} is the same as the real component, the real routine
      is called with {\tt N*2}.
\item {\tt alpha == 0.0} is heavily used in some codes.
\end{enumerate}

\subsection{Notes for SWAP}
\begin{enumerate}
\item Performs $x \leftrightarrow y$
\item An index file input line looks like:
\begin{verbatim}
<ID> <incX> <incY> <source file> <author name>
\end{verbatim}
\item For unit stride case, the complex routine calls the real equivalent
      with {\tt N*2}.
\item The main use of this routine that I am aware of is row-swapping in 
      factorizations, so this is one routine that it would make sense to
      optimize the arbitrary increment cases, if it were possible.
\end{enumerate}

\subsection{Getting your new kernel used}
Once you have a kernel that beats the present offerings, and you have
updated the appropriate index file(s), nothing could be easier.  
If you are starting from a fresh install without Level 1 arch defaults,
you need do nothing further: the install process will find your kernel.

Things are almost as simple if you have already installed, and need to
force a redo.  First, get rid of old search results by issuing
{\tt rm BLDdir/tune/blas/level1/res/<pre><blas>\_SUMM}.
Then, go to {\tt BLDdir/src/blas/level1/}, and type :
\begin{verbatim}
   rm Make_<pre><blas>
   make Make_<pre><blas>
   make <pre>lib
\end{verbatim}
and that should do it.

So, to do that for the omnipresent DAXPY, on my PIII, I'd do:
\begin{verbatim}
   cd BLDdir
   rm tune/blas/level1/res/dAXPY_SUMM.
   cd ATLAS/src/blas/level1/
   rm Make_daxpy
   make Make_daxpy
   make dlib
\end{verbatim}

\subsubsection{Details I doubt you care about}
The executable that tests and times all input kernels and chooses a winner
to be included in ATLAS is {\tt x<blas>srch}.  So, for instance, the
search executable for AXPY can be built by typing
{\tt make xaxpysrch} in {\tt BLDdir/tune/blas/level1/}.

Each BLAS's search routine creates a file {\tt res/<pre><blas>\_SUMM}.
All summation file have the same format, which is:
\begin{verbatim}
<# of specific cases>
<ID> <ialpha> <ibeta> <incX> <incY> <rout> <auth>
....
<ID> <ialpha> <ibeta> <incX> <incY> <rout> <auth>
\end{verbatim}

For instance, if you want to see what went into making your DAXPY, you'd
\begin{verbatim}
speedy. cat res/dAXPY_SUMM 
2
  3   2   2   1   1 axpy32_x1y1.c "R. Clint Whaley"
  1   2   2   0   0 axpy1_x0y0.c "R. Clint Whaley"
\end{verbatim}

So, we see that the search has found two routines to build DAXPY out of.
For {\tt incX = incY = 1}, it will call {\tt axpy32\_x1y1.c}, and for
all other cases, it will call {\tt axpy1\_x0y0.c}

\section{Using special compilers and flags for kernel compilation}

\subsection{Specifying all new compilers and flags}
For all BLAS kernels supported by ATLAS, you can specify a particular
compiler and flags to be used in compiling your kernel.  Normally, the
default compiler specified by config is used.  To override this behavior,
you simply end a primitive line in the particular description file with
the backslash character `\verb+\+', and then the next two lines are
assumed to contain your compiler, and the flags to use, respectively.
For instance, Let us say you start with a simple gemm description file like:
\begin{verbatim}
2
300 480    4    4    1 1 1 4 4 2 ATL_mm4x4x2US.c "V. Nguyen & P. Strazdins" 
301   8    4    4    2 1 1 4 4 2 ATL_mm4x4x2US_NB.c "V. Nguyen & P. Strazdins"
\end{verbatim}

You then decide you want the first routine compiled with gcc, and some
ultrasparc-specific flags.  This file would change to:
\begin{verbatim}
2
300 480    4    4    1 1 1 4 4 2 ATL_mm4x4x2US.c "V. Nguyen & P. Strazdins" \
gcc
-O -mcpu=ultrasparc -mtune=ultrasparc -fno-schedule-insns -fno-schedule-insns2
301   8    4    4    2 1 1 4 4 2 ATL_mm4x4x2US_NB.c "V. Nguyen & P. Strazdins"
\end{verbatim}

\subsection{Specifying additional flags for the default compiler}
At present, only the Level 2 kernels have this option.  It's the same as
when you specify a new compiler and flags, but you specify the compiler
as `+', and the flags you give on the following flagline
are {\em added} to its default flags.  This can be useful for compiling
the same routine multiple times with differing macros (eg., prefetch
distance, etc.).
Let us say you had this simple gemv description file:
\begin{verbatim}
2
  1  8  0  0 ATL_gemvN_mm.c     "R. Clint Whaley"
  2  0  1  1 ATL_gemvN_1x1_1.c  "R. Clint Whaley"
2
101 8  0  0 ATL_gemvT_mm.c      "R. Clint Whaley"
102 0  2  8 ATL_gemvT_2x8_0.c   "R. Clint Whaley"
\end{verbatim}

You now want to compile {\tt ATL\_gemvT\_2x8\_0.c} two different ways:
one with a default prefetch distance, and once with a prefetch distance of
16.  This would be simply:
\begin{verbatim}
2
  1  8  0  0 ATL_gemvN_mm.c     "R. Clint Whaley"
  2  0  1  1 ATL_gemvN_1x1_1.c  "R. Clint Whaley"
2
101 8  0  0 ATL_gemvT_mm.c      "R. Clint Whaley"
102 0  2  8 ATL_gemvT_2x8_0.c   "R. Clint Whaley"
103 0  2  8 ATL_gemvT_2x8_0.c   "R. Clint Whaley" \
+
-DPREFETCH_DIST=16
\end{verbatim}

\subsection{Using a binary kernel}
All the ATLAS machinery assumes you are using a C source, or something
that can be made to look like one (i.e. assembler done as in-line assembler
in C).  Not all ways of programming can be made to work this way, so ATLAS
has the ability to take an object-only kernel.  The level 1 kernels do not
have this option, but the Level 2 and 3 do.

This ability is an ugly hack on the pre-existing machinery, so don't expect
either elegance or ease of use with this option: it is a last-resort kind
of thing at best.

ATLAS has a routine that it builds called {\tt ./xccobj}, which is a 
``compiler'' for object files.  Essentially, what it does is fake compilation,
but instead it simply moves a pre-compiled .o into place during the 
install/tuning process.  You can see the source for this routine in
{\tt ATLAS/bin/ccobj.c}.

This routine reads in some special ``compiler flags'' that you pass it in
order to figure out what to do.  The most important of these are:
\begin{itemize}
\item \verb+-o <outfile>+: Same as with a compiler.
\item \verb+--objdir <objdir>+ The directory where the object file to be
      copied resides.  The default is {\tt ATLAS/blas/gemm/CASES/objs}.
\item \verb+--name <name>+ The base name of the object file to be copied.
This name is suffixed by the appropriate beta suffix, as determined by the
usual BETA macro definition.  Name has no usuable default, so it must be
specified. 
\item \verb+-DBETA+ or \verb+-DATL_BETA+.  These are standard macros used
to specify which BETA case a particular kernel can handle.  According to
this definition, a beta suffix will be added to the \verb+--name+ basename
to find the object file to copy (this is to simulate all the beta cases
coming from the same C code, as in the normal case).  The values, along with
their suffixes are:
   \begin{itemize}
      \item \verb+-DATL_BETA0+ $\Rightarrow$ \verb+_b0+
      \item \verb+-DATL_BETA1+ $\Rightarrow$ \verb+_b1+
      \item \verb+-DATL_BETAX+ $\Rightarrow$ \verb+_bX+
      \item \verb+-DATL_BETAXI0+ $\Rightarrow$ \verb+_bXi0+
   \end{itemize}
The default is \verb+ATL_BETA1+.
\item \verb+-DConj_+:  If this macro definition is found, the name of the
routine to be copied is actually \verb+<name>c_<beta>+, and that routine
should supply the conjugate version of the kernel (simulates recompiling the
same source in order to get both normal and conjugate versions of routines).
The default is no conjugation.  
Note that this macro is set automatically by the build process,
so the user would only set it manually when testing and timing by hand.
\end{itemize}

\subsubsection{Using a binary kernel from the command line}
OK, let us say you have a kernel object file, called {\tt new\_b1.o}
which you have
compiled/assembled/created using the magnetic field of the earth.  You want to
test and time it on the command line.  For the DGEMM kernel, this would be
simply:
\begin{verbatim}
   make ummcase mmrout=CASES/ATL_objdummy.c DMC=./xccobj \
        DMCFLAGS="--name new" nb=30 beta=1
   make mmutstcase mmrout=CASES/ATL_objdummy.c DMC=./xccobj \
        DMCFLAGS="--name new" nb=30 beta=1
\end{verbatim}
Note that {\tt ATL\_objdummy.c} could be any file that exists in the
indicated directory (does not need to be a compilable file), and that 
we do not specify the beta case (done using the tester/timer machinery)
or the {\tt objdir} (because we placed the kernel in the default directory).

\subsubsection{Using a binary kernel from input file}
Here's an example of an object kernel primitive line from a real gemm kernel
descriptor file:
\begin{verbatim}
315   8 -30 -30 -30 0 4 30 30 30 ATL_objdummy.c "Julian Ruhe" \
./xccobj
--name julian2
\end{verbatim}

This guy assumes that in the directory {\tt ATLAS/tune/blas/gemm/CASES/objs},
you have these three files:
\begin{enumerate}
 \item {\tt julian2\_b0.o}
 \item {\tt julian2\_b1.o}
 \item {\tt julian2\_bX.o}
\end{enumerate}

\newpage
\section{A quick reference to ATLAS programming resources}
ATLAS has a quite a bit of programming infrastructure
which can be used.
The routines in {\tt ATLAS/src/auxil} represent low-level routines
which can be called from anywhere in the ATLAS code (other than the Level 1
kernel routines), and are prototyped
by including either {\tt atlas\_misc.h} or {\tt atlas\_aux.h}.

ATLAS makes fairly heavy use of macros in order to achieve something
approaching precision (and in some cases type) independent code. 
Any routine can include the
general macro file {\tt atlas\_misc.h} in order to get access to these
macros.  Low level routines are compiled multiple times with differing
makefile-controlled cpp macros (call these {\em index macros})
in order to produce differing implementations.  The index macros used
in ATLAS presently include:
\begin{itemize}
 \item Type/precision selection macros, choices are:
 \begin{enumerate}
 \item {\tt SREAL}: single precision real is expected
 \item {\tt DREAL}: double precision real is expected
 \item {\tt SCPLX}: single precision complex is expected
 \item {\tt DCPLX}: double precision complex is expected
 \end{enumerate}
 \item Scalar alpha case selection macros, choices are:
 \begin{enumerate}
 \item {\tt ALPHA1}: scalar $\alpha$ should be assumed to be 1
 \item {\tt ALPHAXI0}: only valid for complex types;
       real component is unknown, but imaginary component is zero
 \item {\tt ALPHAX}: scalar alpha is unknown, and must be applied as a variable
 \end{enumerate}
 \item Scalar $\beta$ case selection macros, choices are:
 \begin{enumerate}
 \item {\tt BETA0}: scalar beta should be assumed to be 0
 \item {\tt BETA1}: scalar beta should be assumed to be 1
 \item {\tt BETAXI0}: only valid for complex types;
       real component is unknown, but imaginary component is zero
 \item {\tt BETAX}: scalar beta is unknown, and must be applied as a variable
 \end{enumerate}
 \item For complex types only, ATLAS defines the index variable {\tt Conj\_}
       when the conjugation of the transpose setting is needed
\end{itemize}

Each of these index macros define a number of helper macros that go with
them.  {\tt atlas\_misc.h} should be examined for full details.  
For instance, if {\tt SREAL} or {\tt DREAL} are defined, the
type-determinant macro {\tt TREAL} is defined.  Similarly,
if {\tt SCPLX} or {\tt DCPLX} are defined, the
type-determinant macro {\tt TCPLX} is defined.
For a simpler example, the precision macro defines \verb+ATL_rone+, which
corresponds to {\tt 1.0f} in single precision, and {\tt 1.0} in double.
A great many of these helper macros are designed to be used to help
in resolving names independant of type.  In order to use these naming
macros, we use the macro-joining function {\tt Mjoin(Mac1, Mac2)},
which results in the joining of the {\tt Mac1} and {\tt Mac2}.

You can examine the GEMV kernel implementations for examples of how
this works.  In particular, ATLAS uses the same implementation for
single and double precision by using these macros, and recompiling
the same source with differing index macros.  Just to give a quick example,
the index macro controlling $\beta$ defines a helper macro {\tt BNM} which
corresponds to the correct beta name for the gemv and ger kernels.  Using
this trick, we can reduce:
\begin{verbatim}
#ifdef BETA0
   #ifdef Conj_
      void ATL_dgemvNc_a1_x1_b0_y1
   #else
      void ATL_dgemvN_a1_x1_b0_y1
   #endif
#elif defined (BETA1)
   #ifdef Conj_
      void ATL_dgemvNc_a1_x1_b1_y1
   #else
      void ATL_dgemvN_a1_x1_b1_y1
   #endif
#elif defined (BETAXI0)
   #ifdef Conj_
      void ATL_dgemvNc_a1_x1_bXi0_y1
   #else
      void ATL_dgemvN_a1_x1_bXi0_y1
   #endif
#else
   #ifdef Conj_
      void ATL_dgemvNc_a1_x1_bX_y1
   #else
      void ATL_dgemvN_a1_x1_bX_y1
   #endif
#endif
\end{verbatim}
{\samepage
to:
\begin{verbatim}
#include "atlas_misc.h"

#ifdef Conj_
   void Mjoin(Mjoin(ATL_dgemvNc_a1_x1,BNM),_y1)
#else
   void Mjoin(Mjoin(ATL_dgemvN_a1_x1,BNM),_y1)
#endif
\end{verbatim}
}

\subsection{ATLAS's prefetch header file \label{sec-prefetch}}
ATLAS has an include file ({\tt ATLAS/include/atlas\_prefetch.h}),
which defines the following macros:
\begin{itemize}
\item {\tt ATL\_GOT\_L1PREFETCH}: This macro is only defined if ATLAS has
a way of doing Level 1 cache prefetch for your architecture.
\item {\tt ATL\_pfl1R(mem)} This macro prefetches the address pointed at
by {\tt mem}.  It prefetches one cacheline into L1.  The cacheline length
will vary by platform, and the include file does not define it at this
time (so sue me).  This prefetch is optimized for read access.  If your
system does not possess level 1 prefetch, this macro will define to nothing
(i.e., it will not be an error to call it).
\item {\tt ATL\_pfl1W(mem)} This macro prefetches the address pointed at
by {\tt mem}.  It prefetches one cacheline into L1.  The cacheline length
will vary by platform, and the include file does not define it at this
time (so sue me).  This prefetch is optimized for write access.  If your
system does not possess level 1 prefetch, this macro will define to nothing
(i.e., it will not be an error to call it).
\end{itemize}

These prefetch instructions are not the same as user read, and do not
cause access errors (i.e., you can prefetch the NULL pointer).

\section{Conclusion}
As you have seen, this note and the protocols it describes have plenty of
room for improvement.  Now, as the end-user of this function, you may have
a naturally strong and negative reaction to these crude mechanisms, tempting
you to send messages decrying my lack of humanity, decency, and legal
parentage to the atlas or developer mailing lists.  While this is quite
understandable, keep in mind that I'm not a very understanding guy, and
am also a gigantic baby that pouts when my tender feelings are hurt.  So,
the proper bitch format involves 
\begin{itemize}
\item {\em First} thanking me for spending time in hell getting things
to their present crude state
\item {\em Then}, supplying your constructive ideas
\end{itemize}

For a higher overview of things, scope out the paper
{\tt ATLAS/doc/atlas\_over.ps}, which describes
the ideas behind ATLAS, including several that are touched upon
only lightly in this note.

\newpage
\appendix
\section{Some notes on using assembly}
This
appendix contains useful information for people wanting to supply assembly
kernels.  Presently, ATLAS config detects seven styles of assembler, all
using gnu's gcc/gas:
\begin{enumerate}
\item {\bf x86-32}: Classic x86 assembler (AT\&T syntax)
\item {\bf x86-64}: Assembler for 64-bit mode of x86-64 (AT\&T syntax).
\item {\bf Sparc}:  Sparc assembler.
\item {\bf OS~X PowerPC}:  PowerPC assembly using OS~X's ABI.
\item {\bf Linux PowerPC}:  PowerPC assembly using Linux's ABI.
\item {\bf HP-UX PA-RISC}:  PA-RISC assembly using HP-UX's commands.
\item {\bf Linux PA-RISC}:  PA-RISC assembly using Linux's commands.
\end{enumerate}

The following sections give some hints that may be helpful in writing
assembly for each of these variants.  One trick to keep in mind, when
you are unsure of syntax or how to do something, is to write the code
in C, and use gcc to convert your C implementation to assembler 
({\tt gcc -S}), and see how gcc does it.

\subsection{Some notes on x86-32 assembler}

All assembly kernels presently in ATLAS utilize gcc for compilation.  This
is because gcc is required by the architectural defaults for all x86 archs,
and is freely and widely available.  Gcc/gas uses an AT\&T style assembler
syntax, which may be a bit confusing for people used to the MASM/NASM style.
People tend to talk like this is a big deal, but I learned x86 assembler from
a MASM-style book, and was working in AT\&T immediately.  The main difference
is that MASM has a style of {\tt DEST, SOURCE}, while AT\&T uses 
{\tt SOURCE, DEST}.  For concise description of MASM/AT\&T differences, scope:
\begin{verbatim}
   http://www.gnu.org/manual/gas-2.9.1/html_mono/as.html#SEC198
\end{verbatim}

There are numerous examples of CPP-augmented assembler in ATLAS's kernels.
Look in the relavant description file (eg., 
{\tt ATLAS/tune/blas/gemm/dcases.SSE}, \\
{\tt ATLAS/tune/blas/level1/IAMAX/dcases.dsc}, etc.), and any kernel that
mandates {\tt gcc} as the compiler, with the flags of 
{\tt -x assembler-with-cpp} is an example.

A x86-32 assembler kernel should contain the following CPP lines at the
beginning of the file:
\begin{verbatim}
#ifndef ATL_GAS_x8632
   #error "This kernel requires gas x86-32 assembler!"
#endif
\end{verbatim}
This quick error exit keeps a non-x86-32 assembler from generating hundreds
or thousands of spurious error messages during install.

\newpage
\subsubsection{Register usage}
x86-32 provides a grand total of eight integer registers, a full 7 of which
are usable as integer registers!!  Whatever will we do with such bounty.
Table~\ref{tab-x86-32-regs} summarizes these registers.
\begin{table}[htb]
\begin{center}
\begin{tabular}{||l|l|l||}\hline\hline
         &       & CALLEE\\
REGISTER & USAGE & SAVE\\\hline\hline
\%eax    & integer return val & NO\\\hline
\%edx    & dividend reg  & NO\\\hline
\%ecx    & count reg     & NO\\\hline
\%ebp    & optional frame pointer& YES\\\hline
\%ebx    & local reg             & YES\\\hline
\%esi    & local reg             & YES\\\hline
\%edi    & local reg             & YES\\\hline
\%esp    & Stack pointer         & YES\\\hline\hline
\%st     & fp return val, aliased with mmx         & NO\\\hline
\%st1-7  & scratch regs, aliased with mmx & NO\\\hline\hline
\%mmx0-7 & scratch MMX regs, aliased with fp stack & NO\\\hline\hline
\%xmm0-7 & scratch SSE/SSE2 regs & NO\\\hline\hline
\end{tabular}
\end{center}
\caption{Possible x86-32 registers\label{tab-x86-32-regs}}
\end{table}
Note that the MMX registers are only available on architectures possing
the MMX ISA extensions, and that the {\tt xmm} registers are only available
on architectures implementing SSE or SSE2.  It is also worth mentioning that
all of the integer registers with the exception of the stack pointer 
({\tt \%esp}) may be used as general purpose registers, in addition to their
specialized usages.  Finally, please note that the MMX registers
and floating point are aliased, so they cannot be referenced at the same time.


Registers marked as CALLEE~SAVE~=~YES must be explicitly saved by any
functions which write them.  Registers marked NO are scratch registers
whose values need not be saved.  EAX is used by intergral functions to
return the value, and the floating point top-of-stack, \verb+%st+ returns
floating point values.  EBP is documented as base pointer, but I think
most sane people just do all referencing from ESP (the stack pointer),
and utilize EBP as an additional integer register.

As for floating point registers, the x86-32 has eight of them arranged
in a pseudo-stack.  A document this brief cannot explain this travesty to
you if you don't already grok it.  Go read the x87 FPU description in an
assembler book, and come back when you are through crying.

\subsubsection{The calling sequence and stack frame}
When your function first gets control, the stack frame looks like: \\
\begin{center}
\begin{tabular}{l|l|}
        & Caller's frame \\\hline
        & last arg \\\hline
        & \vdots \\\hline
\%esp+4 & 1st arg \\\hline
\%esp & return address \\\hline
\end{tabular}
\end{center}

To get an idea of how this can be used, let us say we have the following
function:
\begin{verbatim}
   int isum(int N, int *v);
\end{verbatim}

So, we want to sum up the vector {\tt v}, and return the sum.  Let us say
that we are going to use all 7 integer registers, and put {\tt N} into
{\tt eax} and {\tt v} into {\tt edx}.  Our function prologue would then look
like:
\begin{verbatim}
#   int isum(int N, int *v)
.global isum
        .type   isum,@function
isum:
#
#       Save non-scratch registers
#
        subl    $16, %esp
        movl    %ebx, (%esp)
        movl    %ebp, 4(%esp)
        movl    %esi, 8(%esp)
        movl    %edi, 12(%esp)
#
#       Load N and v
#
        movl    20(%esp), %eax
        movl    24(%esp), %edx
\end{verbatim}

Assuming we accumulated our integer sum into {\tt ecx}, the function epilogue
would then consist of:
\begin{verbatim}
#
#       Set return value
#
        movl    %ecx, %eax
#
#       Restore registers
#
        movl    (%esp), %ebx
        movl    4(%esp), %ebp
        movl    8(%esp), %esi
        movl    12(%esp), %edi
        addl    $16, %esp
        ret
\end{verbatim}

\subsubsection{A simple example\label{sec-DDOT-ass}}
Here is the complete code for the simplist assembler implementation of a
{\tt DDOT} primitive (we are implementing the case where {\tt incX} and
{\tt incY} are known to be {\tt 1}):
\begin{verbatim}
#
#  These macros show integer register usage
#
#define N       %eax
#define X       %edx
#define Y       %ecx

#
#double ATL_UDOT(const int N, const double *X, const int incX,
#                             const double *Y, const int incY)
.global ATL_UDOT
        .type   ATL_UDOT,@function
ATL_UDOT:
#
#       Load parameters
#
        movl    4(%esp), N
        movl    8(%esp), X
        movl    16(%esp), Y
#
#       Dot product starts at 0
#
        fldz    
LOOP:
        fldl    (X)
        fldl    (Y)
        fmulp   %st, %st(1)
        addl    $8, X
        addl    $8, Y
        faddp   %st, %st(1)
        dec     N
        jnz     LOOP

        ret
\end{verbatim}

Notice that because we are able to confine ourselves to the three scratch
registers, we have an empty function prologue and epilogue (we do not save
any registers or move the stack pointer).

\newpage
\subsection{Some notes on x86-64 assembler}
x86-64 assembler is AMD's extension of the classic IA32 into 64 bits.
It's expanded register set and associated calling sequence improvements
make it a good deal easier to code. 

A x86-64 assembler kernel should contain the following CPP lines at the
beginning of the file:
\begin{verbatim}
#ifndef ATL_GAS_x8664
   #error "This kernel requires gas x86-64 assembler!"
#endif
\end{verbatim}
This quick error exit keeps a non-x86-64 assembler from generating hundreds
or thousands of spurious error messages during install.


\subsubsection{Register usage}
x86-64 has a much less claustrophobic 16 integer registers, as well
as 16 SSE/SSE2 registers.  It has the usual complement of 8 x87 registers,
aliased with 8 MMX registers.  The available x86-64 registers is summarized
in Table~\ref{tab-x86-64-regs}.
\begin{table}[hbt]
\begin{center}
\begin{tabular}{||l|l|l||}\hline\hline
         &       & CALLEE\\
REGISTER & USAGE & SAVE\\\hline\hline
\%rsp    & Stack pointer         & YES\\\hline
\%rbx    & optional base pointer & YES\\\hline
\%rbp    & optional frame pointer& YES\\\hline
\%rax    & integer return val & NO\\\hline
\%rdi    & 1st int arg  & NO\\\hline
\%rsi    & 2nd int arg  & NO\\\hline
\%rdx    & 3rd int arg  & NO\\\hline
\%rcx    & 4th int arg  & NO\\\hline
\%r8     & 5th int arg  & NO\\\hline
\%r9     & 6th int arg  & NO\\\hline
\%r10    & used to pass static chain pointer & NO\\\hline
\%r11    & scratch reg  & NO\\\hline
\%r12-15 & callee-saved regs & YES \\\hline\hline
\%xmm0-1  & pass \& return fp args & NO \\\hline
\%xmm2-7  & pass fp args & NO \\\hline
\%xmm8-15 & scratch regs & NO\\\hline\hline
\%mmx0-7   & scratch regs, aliased to fp stack & NO\\\hline\hline
\%st      & returns {\tt long double} args & NO\\
          & aliased with mmx regs          & \\\hline
\%st1-7   & scratch regs, aliased with mmx & NO\\\hline\hline
\end{tabular}
\end{center}
\caption{x86-64 register summary\label{tab-x86-64-regs}}
\end{table}

\subsubsection{The calling sequence and stack frame}
When your function first gets control, the stack frame looks like: \\
\begin{center}
\begin{tabular}{l|l|}
        & Caller's frame \\\hline
        & last overflow arg \\\hline
        & \vdots \\\hline
8(\%rsp) & 1st overflow arg \\\hline
0(\%rsp) & return address \\\hline
-8(\%rsp)& begin red zone (16-byte aligned)\\\hline
-128(\%rsp) & end of red zone \\\hline
\end{tabular}
\end{center}

The ``red zone'' is an reserved area that is not modified by signal or 
interrupt handlers, and so may be used as the temporary area for leaf
functions.

When arguments are passed, they are first passed in registers as
summarized in Table~\ref{tab-x86-64-regs}.  Only when all registers of
a given type are used up are they passed on the stack (the ``overflow''
args above).  
Note that the 7th and later integral arguments will overflow, as will the 
9th and later floating point arguments.
All argument lengths are rounded up to 8 bytes (i.e., a
4-byte integer is passed in the {\tt \%edi} portion of the {\tt \%rdi}
register, for instance), both in register passing and in stack passing.

\subsection{Some notes on Sparc assembler}
A Sparc assembler kernel should contain the following CPP lines at the
beginning of the file:
\begin{verbatim}
#ifndef ATL_GAS_SPARC
   #error "This kernel requires gas SPARC assembler!"
#endif
\end{verbatim}
This quick error exit keeps a non-SPARC assembler from generating hundreds
or thousands of spurious error messages during install.

\subsubsection{Register usage}
Table~\ref{tab-sparc-iregs} shows the registers available on the
sparc.  For floating point registers, double precision only uses even 
numbered registers, with the next odd numbered register holding the second
half of the 64-bit value.  Single precision instructions can't use
\%f32-\%f62.  However, one trick to keep in mind is anytime you need to load
two consecutive single precision values, these can be loaded to an even and
next odd register with one {\tt ldd}, which is more efficient than two {\tt ld}.

\begin{table}[hbt]
\begin{center}
\begin{tabular}{||l|l|l||}\hline\hline
         &       & CALLEE\\
REGISTER & USAGE & SAVE\\\hline\hline
\multicolumn{3}{||c||}{\bf Global Registers}\\\hline\hline
\%g0 (\%r0) & Always zero& NO \\\hline
\%g1-\%g4 (\%r1-\%r4) & global regs & NO \\\hline\hline
\%g5-\%g7 (\%r5-\%r7) & system regs - no use & NO USE\\\hline
\multicolumn{3}{||c||}{\bf Out Registers} \\\hline\hline
\%o0 (\%r8) & Return value from callee & NO\\
            & outgoing arg 1 to caller &\\\hline
\%o1-\%o5 (\%r9-\%r13) & outgoing arg 2-6 to caller & NO \\\hline
\%o6/\%sp (\%r14) & Stack pointer & YES \\\hline
\%o7 (\%r15) & scratch/@ of CALL inst. & NO \\\hline\hline
\multicolumn{3}{||c||}{\bf Local Registers} \\\hline\hline
\%l0-\%l7 (\%r16-\%r23) & scratch registers & YES \\\hline\hline
\multicolumn{3}{||c||}{\bf In Registers} \\\hline\hline
\%i0-\%i5 (\%r24-\%r29) & incoming arg 1-6  & YES \\\hline
\%i6/\%fp (\%r30) & frame pointer & YES \\\hline
\%i7 (\%r31) & return address - 8 & YES \\\hline\hline
\multicolumn{3}{||c||}{\bf Floating Point Registers} \\\hline\hline
\%f0-\%f31   & single prec regs   & NO  \\
             & even \#'s double prec &     \\\hline
\%f32-\%f62  & double prec regs   & NO  \\\hline\hline
\end{tabular}
\end{center}
\caption{Sparc register summary\label{tab-sparc-iregs}}
\end{table}

The sparc has only eight globally visable integer registers, the {\bf Global}
registers.  Applications are not allowed to use the system registers
(\%g5-\%g7), \underline{not even if they are saved and restored}.
Note that in the 64-bit variants of the ABI, \%g5 \underline{can} be
used, and need not be saved for {\bf v9}, but must be saved for {\bf v8plus}.
The other 24 registers ({\bf In}, {\bf Out}, {\bf Local})
are in a moving register window, which is moved by the {\tt save} and
{\tt restore} commands.  Note that {\tt \%g0} {\em can} be the target of
an operation, but it's value will not be effected (eg., its value will
remain 0, even if you were to {\tt subcc \%i2, 8, \%g0}).

Note that as long as you use the {\tt save} statement, all callee-saved
registers are implicitly saved by the register window.

The caller's {\bf Out} registers are the callee's {\bf In} registers.

\subsubsection{The calling sequence and stack frame}
After the callee performs the {\tt save} statement, the stack frame looks
like:
%         & \vdots \\

\begin{minipage}{2.7in}
\begin{small}
\begin{tabular}{l|l|}
[\%fp]   & Caller's frame \\\hline\hline
         & variable size local scratch \\\hline
         & (if needed) strg for wrd8 \\
$\lbrack$\%sp+92$\rbrack$& (if needed) strg for wrd7 \\\hline
[\%sp+68]& storage for words 1-6 \\\hline
[\%sp+64]& struct/union return ptr\\\hline
[\%sp]   & ireg window save \\\hline
\end{tabular}
\centerline{\bf 32 bit ABI}
\end{small}
\end{minipage}
%
\begin{minipage}{3.0in}
\begin{small}
\begin{tabular}{l|l|}
[\%fp+2047]   & Caller's frame \\\hline\hline
         & variable size local scratch \\\hline
         & (if needed) strg for wrd8 \\
$\lbrack$\%sp+176+2047$\rbrack$& (if needed) strg for wrd7 \\\hline
[\%sp+128+2047]& storage for words 1-6 \\\hline
[\%sp+2047]  & ireg window save \\\hline
\end{tabular}
\centerline{\bf 64 bit ABI}
\end{small}
\end{minipage}

The callee's arguments are stored in the {\em caller}'s stack frame.
Note that the address of {\tt \%sp} must be kept 8-byte aligned for the
32 bit ABI, and 16-byte aligned for 64 bit.  

For the 64-bit frame, the 2047
offset is called the BIAS, and it allow system libs to know which ABI is
in effect (if 1 in last binary digit, 64 bit).  Integers are stored in
64 bit slots, with the last 4 bytes holding the good value.  Therefore, to
load an int stored at 176 you do a \verb|ldsw  [%sp+2047+176+4], reg|.

For the 32-bit ABIs,
floating point args are passed in the integer regs (2 regs for double) and
floating point functions return their value in {\%f0}.

For the 64 bit v9 ABI, there is a complicated memory-slot/register mapping
which determines how things are passed in registers, as shown
in~\ref{fig-v9argpass} (note that \underline{after} the callee executes
the {\tt save} statement, the \%o registers will of course be the 
\%i registers).  Note that v9a is just v9 with VIS extensions, and v9b
also includes some UltraSPARC-III extensions, so all v9 ABIs are the same
for the information discussed here.
\begin{figure}[h]
\begin{ttfamily}
\begin{center}
\begin{tabular}{||l|l|l|l|l||}\hline\hline
{\bf Memory} & {\bf Intergral} & {\bf Float} & {\bf Double} & {\bf Quad} 
\\\hline\hline
\verb|%sp+BIAS+128| & \%o0 & \%f1 & \%f0 & \%f0 \\\hline
\verb|%sp+BIAS+136| & \%o1 & \%f3 & \%f2 &      \\\hline
\verb|%sp+BIAS+144| & \%o2 & \%f5 & \%f4 & \%f4 \\\hline
\verb|%sp+BIAS+152| & \%o3 & \%f7 & \%f6 &      \\\hline
\verb|%sp+BIAS+160| & \%o4 & \%f9 & \%f8 & \%f8 \\\hline
\verb|%sp+BIAS+168| & \%o5 & \%f11& \%f10&      \\\hline
\verb|%sp+BIAS+176| & n/a  & \%f13& \%f12& \%f12\\\hline
\verb|%sp+BIAS+184| & n/a  & \%f15& \%f14&      \\\hline
\verb|%sp+BIAS+192| & n/a  & \%f17& \%f16& \%f16\\\hline
\verb|%sp+BIAS+200| & n/a  & \%f19& \%f18&      \\\hline
\verb|%sp+BIAS+208| & n/a  & \%f21& \%f20& \%f20\\\hline
\verb|%sp+BIAS+216| & n/a  & \%f23& \%f22&      \\\hline
\verb|%sp+BIAS+224| & n/a  & \%f25& \%f24& \%f24\\\hline
\verb|%sp+BIAS+232| & n/a  & \%f27& \%f26&      \\\hline
\verb|%sp+BIAS+240| & n/a  & \%f29& \%f28& \%f28\\\hline
\verb|%sp+BIAS+248| & n/a  & \%f31& \%f30&      \\\hline
\hline
\end{tabular}
\end{center}
\end{ttfamily}
\caption{Argument passing for 64-bit SPARC v9 ABI}\label{fig-v9argpass}
\end{figure}


\subsection{Some notes on PowerPC assembler}
There are three OSes that I know something about that do PowerPC assembler:
AIX, OS~X, and Linux.  All three of these OSes use the same ABI for 64-bit
assembly (64-bit PowerPC ELF ABI Supplement 1.7).  For 32 bits, the ABIs
for AIX and OS~X are essentially the same,
but Linux differs very slightly in register usage and substantially in
the way the stack is defined.

The standard ATLAS include file defines the macros {\tt ATL\_AS\_OSX\_PPC},
{\tt ATL\_AS\_AIX\_PPC} and {\tt ATL\_GAS\_LINUX\_PPC}, which can be used like:
\begin{verbatim}
#ifndef ATL_AS_OSX_PPC
   #error "This kernel requires OS X PPC assembler!"
#endif
\end{verbatim}
to guard against invalid compilation.

\subsubsection{Register usage for 64 bit PowerPC}
The register usage for PPC64 is given in Table~\ref{tab-PPC64-regs}.

\begin{table}[hbt] 
\begin{center}
\begin{tabular}{||l|l|l||}\hline\hline
         &       & CALLEE\\
REGISTER & USAGE & SAVE\\\hline\hline
\multicolumn{3}{||c||}{\bf Integer Registers}\\\hline\hline
r0\footnotemark[1]
         & Used in prolog/epilog & NO \\\hline
r1       & Stack pointer         & YES \\\hline
r2       & TOC pointer (reserved)& YES \\\hline
r3       & 1st para/return val   & NO  \\\hline
r4-r10   & 3-8th  para           & NO  \\\hline
r11      & Environment pointer   & NO  \\\hline
r12      & Used by global linkage& NO  \\\hline
r13      & {\bf reserved} system thread ID& N/A \\\hline
r14-31   & Global int registers  & YES  \\\hline\hline
\multicolumn{3}{||c||}{\bf Floating Point Registers}\\\hline\hline
f0       & Scratch reg           & NO \\\hline
f1-13    & 1-13th fp para        & NO \\\hline
f14-f31  & Global fp regs        & YES \\\hline\hline
\multicolumn{3}{||c||}{\bf Special Registers}\\\hline\hline
LR       & Link register         & YES \\\hline
CTR      & Count register        & NO  \\\hline
XER      & Fixed pt exception    & NO  \\\hline
FPSCR    & fp status \& ctrl     & NO  \\\hline\hline
CR0-CR7  & Condition reg fields, each 4 bits wide & 2, 3, 4 : YES\\\hline
\multicolumn{3}{||c||}{\bf Vector Registers}\\\hline\hline
v0-v1    & scratch regs          & NO  \\\hline
v2-v13   & vec para regs         & NO  \\\hline
v14-v19  & scratch regs          & NO  \\\hline
v20-v31  & global vregs          & YES \\\hline
vrsave   & (32 bits)             & YES \\\hline\hline
\end{tabular}
\caption{Register usage for 64-bit PowerPC assembly}
\label{tab-PPC64-regs}.
\end{center}
\end{table}
\footnotetext[1]{r0 interpretated as $0$ for many instructions, so don't use
unless you are really sure of what you are doing!}

\subsubsection{The calling sequence and stack frame for 64 bit PowerPC}
The 64-bit PowerPC ELF ABI defines a 288-byte red zone beneath the stack
pointer which can be used by leaf functions in lieu of allocating their
own stack frame.  The stack frame is:
\begin{center}
\begin{tabular}{l|l}
       & fp reg save area (optional) \\\hline
       & ireg save area (optional) \\\hline
       & VRSAVE save word (32 bits, optional) \\\hline
       & padding (optional) \\\hline
       & Local storage (optional) \\\hline
48(r1) & Parameter area ($>= 8$ words) \\\hline
40(r1) & TOC save area \\\hline
32(r1) & Link editor doubleword \\\hline
24(r1) & Compiler doubleword \\\hline
16(r1) & Link register (LR) save \\\hline
 8(r1) & Condition register (CR) save \\\hline
 0(r1) & ptr to callee's stack \\\hline
\end{tabular}
\end{center}

If the LR is changed, it is first saved to the LR save area, and similarly
if any of the callee-saved condition register fields are modified, it must
be saved to the CR save area.

\subsubsection{Register usage for 32-bit OS X or AIX}
\begin{table}[hbt] 
\begin{center}
\begin{tabular}{||l|l|l||}\hline\hline
         &       & CALLEE\\
REGISTER & USAGE & SAVE\\\hline\hline
\multicolumn{3}{||c||}{\bf Integer Registers}\\\hline\hline
r0\footnotemark[1]
         & Used in prolog/epilog & NO \\\hline
r1       & Stack pointer         & YES \\\hline
r2       & TOC pointer (reserved)& YES \\\hline
r3       & 1st  para/return   & NO  \\\hline
r4       & 2nd  para/return   & NO  \\\hline
r5-r10   & 3-8th  para        & NO  \\\hline
r11      & Environment pointer   & NO  \\\hline
r12      & Used by global linkage& NO  \\\hline
r13-31   & Global int registers  & YES  \\\hline\hline
\multicolumn{3}{||c||}{\bf Floating Point Registers}\\\hline\hline
f0       & Scratch reg           & NO \\\hline
f1-13    & 1-13th fp para        & NO \\\hline
f14-f31  & Global fp regs        & YES \\\hline\hline
\multicolumn{3}{||c||}{\bf Special Registers}\\\hline\hline
LR       & Link register         & YES \\\hline
CTR      & Count register        & NO  \\\hline
XER      & Fixed pt exception    & NO  \\\hline
FPSCR    & fp status \& ctrl     & NO  \\\hline\hline
CR0-CR7  & Condition reg fields, each 4 bits wide & 2, 3, 4 : YES\\\hline
\end{tabular}
\end{center}
\end{table}
\footnotetext[1]{r0 interpretated as $0$ for many instructions, so don't use
unless you are really sure of what you are doing!}

Note that if an OS~X/AIX routine accepts a floating point register, the
appropriate number of integer registers are skipped, even though the value
is passed in a fp reg.  Linux does not waste registers in this way.

\subsubsection{The calling sequence and stack frame for 32-bit OS~X/AIX}
When control is passed to the called routine, the stack pointer points to
the callee's frame, which looks like:
\begin{center}
\begin{tabular}{l|l}
       & fp reg save area (optional) \\\hline
       & ireg save area (optional) \\\hline
       & padding (optional) \\\hline
       & Local storage (optional) \\\hline
24(r1) & Parameter area ($>= 8$ words) \\\hline
20(r1) & TOC save area \\\hline
16(r1) & Link editor doubleword \\\hline
12(r1) & Compiler doubleword \\\hline
 8(r1) & Link register (LR) save \\\hline
 4(r1) & Condition register (CR) save \\\hline
 0(r1) & ptr to callee's stack \\\hline
\end{tabular}
\end{center}

\begin{itemize}
\item Stack pointer must be quadword aligned
\item Start of fp \& ireg save areas always double word aligned
\item Before calling another func, must save the LR in LR save area
\item Have 224-byte (OSX) / 220-byte (AIX) red zone below stack ptr so leaf func frame
\end{itemize}

\subsubsection{Register usage for 32 bit Linux}
\begin{table}[hbt] 
\begin{center}
\begin{tabular}{||l|l|l||}\hline\hline
         &       & CALLEE\\
REGISTER & USAGE & SAVE\\\hline\hline
\multicolumn{3}{||c||}{\bf Integer Registers}\\\hline\hline
r0\footnotemark[1]
         & Used in prolog/epilog & NO \\\hline
r1       & Stack pointer         & YES \\\hline
r2       & TOC pointer (reserved)& YES \\\hline
r3-r4    & 1/2 para and return   & NO  \\\hline
r5-r10   & 3-8th  integer para   & NO  \\\hline
r11-r12  & Func linkage regs     & NO  \\\hline
r12      & Used by global linkage& NO  \\\hline
r13      & Small data area ptr reg& NO \\\hline
r14-30   & Global int registers  & YES  \\\hline
r31      & Global/environment ptr & YES \\\hline\hline
\multicolumn{3}{||c||}{\bf Floating Point Registers}\\\hline\hline
f0       & Scratch reg           & NO \\\hline
f1       & 1st para / return     & NO \\\hline
f2-8     & 2-8th fp para         & NO \\\hline
f9-f13   & Scratch reg           & NO \\\hline
f14-f31  & Global fp regs        & YES \\\hline\hline
\multicolumn{3}{||c||}{\bf Special Registers}\\\hline\hline
CR0-CR7  & Condition reg fields, each 4 bits wide & 2, 3, 4 : YES\\\hline
LR       & Link register         & YES \\\hline
CTR      & Count register        & NO  \\\hline
XER      & Fixed pt exception    & NO  \\\hline
FPSCR    & fp status \& ctrl     & NO  \\\hline\hline
\end{tabular}
\end{center}
\end{table}
\footnotetext[1]{r0 interpretated as $0$ for many instructions, so don't use
unless you are really sure of what you are doing!}

\subsubsection{The calling sequence and stack frame for Linux}
When control is passed to the called routine, the stack pointer points to
the callee's frame, which looks like:
\begin{center}
\begin{tabular}{l|l}
       & fp reg save area (optional) \\\hline
       & ireg save area (optional) \\\hline
       & CR save area (optional) \\\hline
       & Local storage (optional) \\\hline
 8(r1) & Parameter area (optional) \\\hline
 4(r1) & Link register (LR) save \\\hline
 0(r1) & ptr to callee's stack \\\hline
\end{tabular}

\begin{itemize}
\item Stack pointer is always 16-byte (quad-word) aligned.
\item Stack pointer (sp) updated atomically via ``store word with update''.
\item Any non-scratch reg f\# must be saved to the fp reg save area at location
      8*(32-\#) from the previous frame (i.e., the register f31 is saved
      adjacent to previous ptr to callee's stack).
\item Any non-scratch reg r\# must be saved in the ireg save area 4*(32-\#)
      bytes before the low-addressed end of the fp reg save area.
\item Minimum stack frame callee save and LR save.
\item No red zone is specifically mandated.
\item If number if ipara is $\leq 8$ and the number of fppara $\leq 8$, then
      no values are stored in the parameter area, and if this is true for
      all calls made by the routine, the parameter area will be of size 0.
\end{itemize}
\end{center}

\begin{itemize}
\item Stack pointer must be quadword aligned
\item If parameters are contained in registers, no space allocated in frame
\end{itemize}

\subsubsection{Mixing OS X, AIX and Linux PPC assembler}
Other than register usage and stack frame issues (which can be isolated
into the prologue and epilogue sections of code), the only real difference
between these assemblies lies in the way registers are addressed.  Under
Linux, all registers are addressed by their number only, whereas under
OS~X, integer registers are prefixed by {\tt r}, and floating point registers
are prefixed by {\tt f}.  Therefore, I usually write a routine with
two different prologues and epilogues, and at the start of the file I
do something like:
\begin{verbatim}
#if defined(ATL_GAS_LINUX_PPC) || defined(ATL_AS_AIX_PPC)
   #define r0 0
   #define r1 1
   ...
   #define r31 31

   #define f0 0
   #define f1 1
   ...
   #define f31 31
#endif
\end{verbatim}
and then use the OS~X-style of register naming.

Also, since I'm using cpp anyway, I use C style comments, to avoid 
problems with OS~X and Linux having different comment characters.

\subsection{Some notes on HP PA-RISC assembler}
ATLAS presently supports only 32 bit PA-RISC assembly.  Such routines should
include:
\begin{verbatim}
#if !defined(ATL_LINUX_PARISC) && !defined(ATL_HPUX_PARISC)
   #error "This kernel requires PA-RISC assembler!"
#endif
\end{verbatim}
This quick error exit keeps a non-parisc assembler from generating hundreds
or thousands of spurious error messages during install.

\subsubsection{Register usage}
Table~\ref{tab-hppa-iregs} shows the registers available on the
PA-RISC.  For floating point registers, single precision registers get
additional modifier 'L' or 'R' for left or right 32 bits.

\begin{table}[hbt]
\begin{center}
\begin{tabular}{||l|l|l||}\hline\hline
         &       & CALLEE\\
REGISTER & USAGE & SAVE\\\hline\hline
\%r0 & Always zero& NO \\\hline
\%r1 & general reg & NO \\\hline
\%r2 (\%rp) & return ptr/address & NO \\\hline
\%r3-\%r18 & general regs & YES \\\hline
\%r19      & shared lib link reg & NO \\\hline
\%r20-\%r22& general regs & NO \\\hline
\%r23 (\%arg3) & 4th iarg & NO  \\\hline
\%r24 (\%arg2) & 3rd iarg & NO  \\\hline
\%r25 (\%arg1) & 2nd iarg & NO  \\\hline
\%r26 (\%arg0) & 1st iarg & NO  \\\hline
\%r27 (\%dp)   & RESERVED: global data pointer & NA \\\hline
\%r28 (\%ret0)  & func ret reg & NO\\\hline
\%r29   & static link reg & NO\\\hline
\%r30 (\%sp)   & stack pointer & YES\\\hline
\%r31          & general reg & NO\\\hline\hline
\multicolumn{3}{||c||}{\bf Floating Point Registers} \\\hline\hline
\%fr0 / always zero   & fp status reg & NO  \\\hline
\%fr1-\%fr3  & exceptions regs & NO \\\hline
\%fr4-\%fr7  & fp arg regs & NO \\\hline
\%fr8-\%fr11 & fp regs     & NO \\\hline
\%fr12-\%fr21 & fp regs   & YES  \\\hline
\%fr22-\%fr31 & fp regs   & NO  \\\hline\hline
\end{tabular}
\end{center}
\caption{HPPA register summary\label{tab-hppa-iregs}}
\end{table}

\begin{itemize}
\item Non-leaf functions must save \%r2 before calling a func.
\item single prec fp args are passed in right side of fp arg ptrs
\item If single prec fp arg has no 'L' or 'R' suffix, 'L' is assumed
\end{itemize}

\subsubsection{Calling sequence and stack frame}
Stack pointer always kept 64-bit aligned, and it grows {\em upward}.
\begin{center}
\begin{tabular}{l|l}
              & \vdots previous stack frame \\\hline
              & reg save \\\hline
              & local vars \\\hline
-4*(N+9)(r30) & arg word N (opt) \\\hline
-52(r30) & arg word 4 (optional) \\\hline
-48(r30) & arg word 3 (required) \\\hline
-44(r30) & arg word 2 (required) \\\hline
-40(r30) & arg word 1 (required) \\\hline
-36(r30) & arg word 0 (required) \\\hline
-32(r30) & External Data/LT ptr\\\hline
-28(r30) & External SR4/LT\\\hline
-24(r30) & External RP\\\hline
-20(r30) & Current RP\\\hline
-16(r30) & Static Link\\\hline
-12(r30) & Cleanup\\\hline
 -8(r30) & Reloc stub RP\\\hline
 -4(r30) & ptr to callee's stack \\\hline
\end{tabular}
\end{center}

\clearpage
\subsection{Some notes on MIPS assembler}
ATLAS presently supports only the 64-bit ISAs (MIPS III and  MIPS IV)
using the ABIs discussed below, which seem to work for Linux and IRIX.
This information is presently priliminary, but as far as I can tell,
both IRIX and Linux use the same ABIs.  The only difference appears to
be in their comment character, which is easily worked around by using
C-style comments (assuming you use cpp, as I do).  
%Note that {\tt gcc} is
%invoked differently on the platforms.  For Linux, I had to add the flag
%{\tt -mmips64}, whereas IRIX wanted {\tt -mabi=64} in order to assemble
%the same {\.S} file.

There seem to be a lot of MIPS ABIs, and different archs use different
ones.  As far as I know, Linux follows the SGI/IRIX ABIs, which can
be sorted into the following catagories, based on the IRIX {\bf cc}
flag or {\bf gcc}'s {\bf -mabi=} flag:
\begin{description}
\item [-32]: this is the classic 32 ABI build for old MIPS I MIPS II 32 bit
             hardware.  Has only 16 fp regs, and integer regs are 32 bits long.
             Not presently supported by ATLAS.
\item [-o64]: Seems to be a 64-bit extension of {\bf -32}, which is also
              not presently supported by ATLAS.
\item [-64]: 64 bit ABI for MIPS III and MIPS IV ISA machines.
             Integer regs are 64 bits long, and 32 fp regs are allowed.
             Supported by ATLAS.  
\item [-n32]: 32 bit ABI for MIPS III and MIPS IV ISA machines.  
              Integer regs are 64 bits long, and 32 fp regs are allowed.
              Has same register usage as {\bf -64}, but integers, longs, and
              pointers all 32 bit. 
\end{description}

{\bf NOTE:} rest of this document describes my understanding of {\bf -64}
and {\bf -n32} ABIs only.

MIPS assembler routines in ATLAS should include:
\begin{verbatim}
#if !defined(ATL_GAS_MIPS)
   #error "This kernel requires MIPS assembler!"
#endif
\end{verbatim}
This quick error exit keeps a non-MIPS assembler from generating hundreds
or thousands of spurious error messages during install.

\subsubsection{Register usage}
Table~\ref{tab-mips-iregs} shows the registers available on MIPS.

\begin{table}[hbt]
\begin{center}
\begin{tabular}{||l|l|l||}\hline\hline
         &       & CALLEE\\
REGISTER & USAGE & SAVE\\\hline\hline
\$0 & Always zero& n/a \\\hline
\$1/\$at & assemb temp& NO \\\hline
\$2-\$3 & integer return & NO \\\hline
\$4-\$11 & 1$^{\rm st}$ 8 iargs & NO \\\hline
\$12-\$15 & scratch & NO \\\hline
\$16-\$23& saved regs & YES \\\hline
\$24-\$25 & scratch & NO  \\\hline
\$26-\$27 & reserved for kernel & n/a  \\\hline
\$28 (\$gp)  & global pointer & YES \\\hline
\$29 (\$sp)  & stack pointer & YES\\\hline
\$30   & optional frame ptr & YES\\\hline
\$31   & return address     & NO \\\hline\hline
\multicolumn{3}{||c||}{\bf Floating Point Registers for -64} \\\hline\hline
\$f0,\$f2    & fp func return vals & NO  \\\hline
\$f1,\$f3    & scratch             & NO \\\hline
\$f4-\$f11   & scratch             & NO \\\hline
\$f12-\$f19  & 1$^{\rm st}$ 8 fp args & NO \\\hline
\$f20-\$f23  & scratch             & NO \\\hline
\$f24-\$f31  & scratch             & YES \\\hline\hline
\multicolumn{3}{||c||}{\bf Floating Point Registers for -n32} \\\hline\hline
\$f0,\$f2    & fp func return vals & NO  \\\hline
\$f1,\$f3    & scratch             & NO \\\hline
\$f4-\$f11   & scratch             & NO \\\hline
\$f12-\$f19  & 1$^{\rm st}$ 8 fp args & NO \\\hline
\$f21,23,25,27,29,31  & scratch             & NO \\\hline
\$f20,22,24,26,28,30  & scratch  & YES \\\hline
\end{tabular}
\end{center}
\caption{MIPS register summary\label{tab-mips-iregs}}
\end{table}

Note that in parameter passing, you can only pass 8 args in registers, 
regardless of what type, and args always consume a register of the correct
type, and cause a skip of the other type.  So, here's the register passing
of a simple routine with mixed arguments:
\begin{verbatim}
//          $f12    $5       $f14     $7
void bob(float s, int i, double d, int k);
\end{verbatim}

\subsubsection{Calling sequence and stack frame}
Stack pointer always kept 16-byte aligned (8-byte aligned for the old
unsupported {\bf -32} ABI), and it grows downward:
\begin{center}
\begin{tabular}{l|l|}
        & Caller's frame \\\hline
        & \vdots~~ remaining overflow args \\\hline
16(\$sp)& 1$^{\rm st}$ overflow arg \\\hline
        & 16 bytes reserved \\
\$sp    & for args passed in regs \\\hline
\end{tabular}
\end{center}

Note that integers are promoted to 64 bits, and all optional parameters
are allocated at least 64 bits, even if they are short.

\end{document}
